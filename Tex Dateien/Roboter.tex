% !TeX root = Roboter.tex
% !TeX root = ./Roboter.tex
\documentclass[a4paper,12pt, oneside]{scrartcl}
\usepackage[ngerman]{babel}
\usepackage[utf8]{inputenc}
\setlength{\parindent}{0em}
\usepackage{eurosym}
\usepackage{amsmath}
\usepackage{amssymb}
\usepackage{dsfont}
\usepackage{polynom}
\usepackage{graphicx}
\usepackage{caption}
\usepackage{tikz}
\usepackage[a4paper,portrait,left=1.0cm,right=1.0cm,top=2cm,bottom=2cm]{geometry}
\usepackage{pgf} % Zur Einbindung der PGF Files
\usepackage{bm}

% Light Mode Customization
\usepackage{xcolor}
%\definecolor{backgroundgray}{RGB}{255,255,255}  % Weiß als Seitenhintergrund
\definecolor{textwhite}{RGB}{0,0,0}            % Schwarz als globale Schriftfarbe
\definecolor{boxgray}{RGB}{245,245,245}        % Sehr helles Grau für Box-Hintergründe
\definecolor{boxframe}{RGB}{200,200,200}       % Helles Grau für Box-Rahmen
\pagecolor{boxgray}  % Hintergrund auf Weiß
\color{textwhite}           % Text auf Schwarz

% Eigene Farben definiert (nun optimiert für hellen Hintergrund)
\definecolor{rot}{RGB}{204,0,0}
\definecolor{blau}{RGB}{0,102,204}
\definecolor{orange}{RGB}{204,85,0}
\definecolor{gruen}{RGB}{0,153,0}
\definecolor{dunkelgruen}{RGB}{0,153,0}
\definecolor{dunkelblau}{RGB}{0,128,128}
\definecolor{backcolor}{RGB}{235,245,255}
\definecolor{frontcolor}{RGB}{0,153,0}
\definecolor{hellblau}{RGB}{180,210,235}
\definecolor{lila}{RGB}{153,0,153}
\definecolor{gelb}{RGB}{204,153,0}
\definecolor{dunkelrot}{RGB}{153,0,51}
\definecolor{dunkelgelb}{RGB}{153,153,0}

% Section colors (Headings)
%\usepackage{sectsty}
%\sectionfont{\color{textwhite}}
%\subsectionfont{\color{textwhite}}
%\subsubsectionfont{\color{textwhite}}

% Box customization mit mdframed
\usepackage{mdframed}
\mdfdefinestyle{mystyle}{
    leftmargin=0pt,
    linecolor=boxframe,
    linewidth=1pt,
    backgroundcolor=boxgray,
    fontcolor=textwhite,
    %frametitlefont={\color{gruen}\bfseries}
}

% Custom theorem environments
\newmdtheoremenv[style=mystyle]{satz}{\color{rot}Satz}[section] % linecolor=rot
\newmdtheoremenv[style=mystyle]{definition}[satz]{\color{gelb}Definition}
\newmdtheoremenv[style=mystyle,leftmargin=-10pt]{beispiel}[satz]{\color{lila}Beispiel}
\newmdtheoremenv[style=mystyle,leftmargin=-10pt]{bem}[satz]{Anmerkung}
\newmdtheoremenv[style=mystyle,leftmargin=-10pt]{ex}{Aufgabe} % \color{gruen}

% Proof- und Lösungs-Kommandos
\newcommand{\proof}[1]{
    \begingroup
    \textit{\hfill \\ \gruen{\textbf{Beweis:}} }#1 \hfill $\Box$
    \endgroup
}
\newcommand{\loesung}[1]{
    \begingroup
    \renewcommand{\textsc}[1]{{\rmfamily\scshape##1}}
    \renewcommand{\emph}[1]{{\normalfont##1}}
    \addtokomafont{subsection}{\color{gruen}}
    \addtokomafont{subsubsection}{\color{gruen}}
    \addtokomafont{caption}{\color{gruen}}
    \addtokomafont{captionlabel}{\color{gruen}}
    \sffamily\slshape\color{textwhite}
    \textbf{\hfill \\ Lösung: }#1 
    \endgroup
}

% Listings für Python
\usepackage[german,tworuled]{algorithm2e}
\usepackage{listings}
\lstset{
    commentstyle=\color{gruen},
    keywordstyle=\color{blau},
    language=Python,
    backgroundcolor=\color{boxgray},
    basicstyle=\ttfamily\color{textwhite},
    stringstyle=\color{rot},
    tabsize=2,
    literate={@}{{@}}1,
    title=\lstname
}

% Hyperref Settings
\usepackage[pdfstartview=Fit,plainpages=false,colorlinks=true,linkcolor=gruen,linktocpage=true,hyperfootnotes=false,bookmarksopen=false]{hyperref}

% Eigene Kommandos
\newcommand{\e}[1]{\text{e}^{#1}}
\newcommand{\rot}[2][rot]{\textcolor{#1}{#2}}
\newcommand{\gruen}[2][gruen]{\textcolor{#1}{#2}}
\newcommand{\frontcolor}[2][frontcolor]{\textcolor{#1}{#2}}
\newcommand{\gelb}[2][gelb]{\textcolor{#1}{#2}}
\newcommand{\orange}[2][orange]{\textcolor{#1}{#2}}
\newcommand{\blau}[2][blau]{\textcolor{#1}{#2}}
\newcommand{\hellblau}[2][hellblau]{\textcolor{#1}{#2}}
\newcommand{\lila}[2][lila]{\textcolor{#1}{#2}}
\newcommand{\dunkelrot}[2][dunkelrot]{\textcolor{#1}{#2}}
\newcommand{\dunkelgelb}[2][dunkelgelb]{\textcolor{#1}{#2}}

%%%%%%%%%%%%%%%%%%%%%%%%%%%%%%%%%%%%%%%%%%%%%%%%%%%%%%%%%%%%%%%%%%%%%%%
\newcommand{\norm}[1]{\left\lVert#1\right\lVert}
\newcommand{\dv}{\thinspace \mathrm{d}v}
\newcommand{\dw}{\thinspace \mathrm{d}w}
\newcommand{\dx}{\thinspace \mathrm{d}x}
\newcommand{\dy}{\thinspace \mathrm{d}y}
\newcommand{\dt}{\thinspace \mathrm{d}t}
\newcommand{\dr}{\thinspace \mathrm{d}r}
\newcommand{\ds}{\thinspace \mathrm{d}s}
\newcommand{\du}{\thinspace \mathrm{d}u}
\newcommand{\dz}{\thinspace \mathrm{d}z}
\newcommand{\dW}{\thinspace \mathrm{d}W}
\newcommand{\dX}{\thinspace \mathrm{d}X}
\newcommand{\dP}{\thinspace \mathrm{d}P}
\DeclareMathOperator*{\esssup}{ess\,sup}
\DeclareMathOperator*{\argmin}{arg\,min}
%% adjust title page to be centered
% \typearea[18mm]{11}
\usepackage[backend=biber,style=numeric]{biblatex}
\addbibresource{references.bib}
\usepackage{listings}
\definecolor{codebg}{rgb}{0.92,0.92,0.92}
\definecolor{keywordcolor}{rgb}{0.2,0.2,0.7}
\definecolor{stringcolor}{rgb}{0.8,0.1,0.1}
\definecolor{commentcolor}{rgb}{0.4,0.4,0.4}


\lstset{
  backgroundcolor=\color{codebg},
  %frame=single,
  %numbers=left,
  %numberstyle=\tiny\color{gray},
  basicstyle=\ttfamily\small,
  keywordstyle=\color{keywordcolor}\bfseries,
  stringstyle=\color{stringcolor},
  commentstyle=\color{gruen}\itshape,
  language=Python,
  tabsize=2,
  showstringspaces=false
}



\begin{document}

\begin{titlepage}

\newcommand{\TitleLineI}{Modellierung eines Roboters} 
\newcommand{\Author}{Nicolas Schäfer}
\newcommand{\SubmissionDate}{Saarbrücken, \today}

\thispagestyle{empty}



%\begin{center}
	%\includegraphics[scale=0.3]{cover.jpg}
%	\includegraphics[scale=0.3]{cover2.jpg}
%\end{center}


\rule{\textwidth}{0.4pt}

\begin{center}
	\begin{Large}
		Modellierung eines Roboters
	\end{Large}
\end{center}

\vspace{2.0cm}
\begin{center}
		\begin{tabular}{c}
			\Author \\
			 \SubmissionDate 
			%%Uncomment the following line for a third title line 
%			\TitleLineIII \\
		\end{tabular}
\end{center}
\vspace{1.5cm}

\end{titlepage}


\pagenumbering{roman}

\tableofcontents

\newpage

\pagenumbering{arabic}

\raggedbottom

\section{Modellierung eines Roboters}
 Wir halten uns an \cite{spong2004} und \cite{siciliano2010}
% Herleitung der ELE aus Newton
\begin{align*}
    \underbrace{f - mg}_{\textsf{Kräftebilanz}} 
    = \underbrace{m \ddot{x}}_{\textsf{Newtons 2nd Law}} 
    &= \frac{\mathrm{d}}{\dt} \left[m \rot{\dot{x}} \right] \\
    f - \frac{\mathrm{\partial}}{\partial x} 
    \underbrace{
        \left[ mg x(t) \right]
        }_{\textsf{Potenzielle Energie }V} 
    &= \frac{\mathrm{d}}{\dt} \rot{\frac{\partial}{\partial \dot{x}}} 
    \underbrace{
        \left[m \rot{ \frac{1}{2} \dot{x}(t)^2} \right]
        }_{\textsf{kinetische Energie } T} \\
    f \dunkelgelb{- \frac{\partial V}{\partial x}} 
    = \frac{\mathrm{d}}{\dt} \rot{\frac{\partial T}{\partial \dot{x}}}
    &\Rightarrow
    f = \frac{\mathrm{d}}{\dt} \rot{\frac{\partial L}{\partial \dot{x}}} 
    - \dunkelgelb{\frac{\partial L}{\partial x}} 
\end{align*}

Wobei $L = T - V = \frac{1}{2}m \dot{x}(t)^2 -mg x(t)$ gilt.
Für generalisierte Koordinaten $\textbf{q} = \theta \in \mathbb{R}^n$
%
\begin{align*}
\frac{d}{dt} \left( \frac{\partial T}{\partial \dot{\textbf{q}}_{i}}  \right) 
- \frac{\partial T}{\partial \textbf{q}_i}  
+ \frac{\partial V}{\partial \textbf{q}_i} 
= \tau_i \quad i=1, \cdots , n
\end{align*}

% Herleitung kinetische Energie und Massenmatrix
Für die kinetische Energie eines starren Körpers gilt:
%
\begin{align*}
    T 
    &= \frac{1}{2} m \textbf{v}^T \textbf{v} 
    + \frac{1}{2} \boldsymbol{\omega}^T I \boldsymbol{\omega}
\end{align*}
%
Wobei $\textbf{v}$ die Geschwindigkeit des Schwerpunkts, 
$\boldsymbol{\omega}$ die Winkelgeschwindigkeit 
und $I$ der Trägheitstensor bezüglich des Schwerpunkts ist.
Für den Trägheitstensor gilt:
% Trägheitstensor Definition
\begin{align}
    I = \int_V \rho(\textbf{r}) 
    [(\textbf{r} \cdot \textbf{r}) \textbf{1} 
    - \textbf{r} \otimes \textbf{r}] \mathrm{d}V \label{eq:traegheitstensor}
\end{align}

Für ein Mehrkörpersystem mit generalisierten 
Koordinaten $\textbf{q}$ lässt sich die Gesamtenergie in 
kompakter Form schreiben als:
%
\begin{align*}
T &= \frac{1}{2} \dot{\textbf{q}}^T M(\textbf{q}) \dot{\textbf{q}}
\end{align*}
%
wobei $M(\textbf{q})$ die konfigurationsabhängige Massenmatrix ist, 
die alle Massen und Trägheiten der einzelnen Glieder erfasst.
Für die Geschwindigkeiten gilt mittels der Kettenregel:
% Definition Jacobi J_v, J_omega
\begin{align*}
    \textbf{v} = J_v(\textbf{q}) \dot{\textbf{q}} \quad \text{und} \quad
    \boldsymbol{\omega} = J_\omega(\textbf{q}) \dot{\textbf{q}}
    \quad
    J_v = \frac{\partial \textbf{r}}{\partial \textbf{q}} 
    \quad \text{und} \quad  
    J_\omega = \frac{\partial \boldsymbol{\omega}}{\partial \dot{\textbf{q}}}
\end{align*}

Einsetzen in die kinetische Energie liefert:
% Herleitung T = 1/2 q'M(q)q
\begin{align*}
    T 
    &= \frac{1}{2} m (J_v \dot{\textbf{q}})^T (J_v \dot{\textbf{q}}) 
    + \frac{1}{2} (J_\omega \dot{\textbf{q}})^T I (J_\omega \dot{\textbf{q}}) \\
    &= \frac{1}{2} \dot{\textbf{q}}^T (m J_v^T J_v) \dot{\textbf{q}} 
    + \frac{1}{2} \dot{\textbf{q}}^T (J_\omega^T I J_\omega) \dot{\textbf{q}} \\
    &= \frac{1}{2} \dot{\textbf{q}}^T 
    \underbrace{\left( m J_v^T J_v + J_\omega^T I J_\omega \right)}_{M(\textbf{q})} 
    \dot{\textbf{q}}
\end{align*}

Die Gesamtenergie setzt sich zusammen 
aus der Summe der Energien aller $n$ Glieder:
% Formulierung M(q)
\begin{align*}
    M(\textbf{q}) 
    &=
    \sum_{i=1}^n m_i J_{v_i}^T J_{v_i} + J_{\omega_i}^T I_i J_{\omega_i} 
\end{align*}

\subsection{Two Link Revolute Manipulator}

Wir bezeichnen $\textbf{q} = 
\begin{bmatrix} \gruen{\theta_1} \\ \dunkelrot{\theta_2} \end{bmatrix}$ 
als die generalisierten Koordinaten des Systems. Die Schwerpunkte der beiden Links
bezeichnen wir mit $(x_1, y_1)$ und $(x_2, y_2)$. Die Abstände der Schwerpunkte
von den Gelenken werden mit $a_1$ und $a_2$ bezeichnet.

% Graphik des Roboterarms
%--------------------------------------------------------------------------------------------------------
\begin{center}
\begin{tikzpicture}[scale = 1.75, every node/.style={transform shape}]
    % Koordinatensystem
    %\draw[very thin, gray] (-5,-5) grid (5,5); % Raster
    %\draw[->] (-5,0) -- (5,0); % x-Achse
    %\draw[->] (0,-5) -- (0,5); % y-Achse
    
    % Achsen und Hilfslinien
    \draw[white, dashed, ->] (0.0,-1.0) -- (3.5,-1.0);
    \draw[white, dashed, ->] (0.0,-1.0) -- (0.0,3.0);
    
    % Schwerpunkte
    \draw[dashed] (1.0,-0.5) -- (1.0,-1.0);
    \draw[dashed] (1.0,-0.5) -- (0,-0.5);
    \node at (-0.2, -0.5) {\tiny $y_1$};  % Punktbeschriftung
    \node at (1, -1.2) {\tiny $x_1$}; 
    \draw[dashed] (2.5,1) -- (0.0,1);
    \draw[dashed] (2.5,1) -- (2.5,-1);
    \node at (-0.2, 1.0) {\tiny $y_2$};  % Punktbeschriftung
    \node at (2.5, -1.2) {\tiny $x_2$}; 

    % Erste Zeichnung (Original)
    \fill[hellblau] (-4,-1) circle (0.25);  % Füllt das Objekt mit hellblau
    \draw[white, line width = 0.2mm] (-4,-1) circle (0.25);  % Weißer Rand um das Objekt
    
    \draw[white, line width = 0.2mm] (-4.25,-1.4) rectangle (-3.75,-1);  % Weißer Rand um das Rechteck
    \fill[hellblau] (-4.25,-1.4) rectangle (-3.75,-1);  % Füllt das Rechteck mit hellblau
    \draw[thick, white] (-4.5,-1.4) -- (-3.5,-1.4);
    
    % Grüne Linie angepasst (Endpunkt auf (-2.0, 0.0))
    \draw[thick, gruen] (-4,-1) -- (-2.0,0.0);
    
    % Rote Linie angepasst (von (-2.0, 0.0) nach (-1.0, 2.0))
    \draw[thick, dunkelrot] (-2.0,0.0) -- (-1.0,2.0);
    
    % Glieder
    \fill[blau] (-2.0,0.0) circle (0.1);
    \draw[white, thin] (-2.0,0.0) circle (0.1);  % Weißer Rand um das Objekt
    \fill[blau] (-4.0,-1.0) circle (0.1);
    \draw[white, thin] (-4.0,-1.0) circle (0.1);  % Weißer Rand um das Objekt
    
	% Beschriftungen
	\node at (-3, -0.75) {\tiny $\gruen{l_1}$};  % 
	\node at (-1.5, 1.5) {\tiny $\dunkelrot{l_2}$};  % 
	
	% Schwerpunkte 
	\fill[white] (-3.0,-0.5) circle (0.05);
	\fill[white] (-1.5,1) circle (0.05);
	
	\draw[white, <->] (-1.85,0.15) -- (-1.5,0.85);
	\draw[white, <->] (-3.85,-0.85) -- (-3.1,-0.45);
	
	% Beschriftungen
	\node at (-3.5, -0.4) {\tiny $a_1$};  % 
	\node at (-1.5, 0.5) {\tiny $a_2$};  % 
    
    % Zweite Zeichnung (verschoben um 4 Einheiten entlang der x-Achse)
    \fill[hellblau] (0,-1) circle (0.25);  % Füllt das Objekt mit hellblau
    \draw[white, line width = 0.2mm] (0,-1) circle (0.25);  % Weißer Rand um das Objekt
    
    \draw[white, line width = 0.2mm] (-0.25,-1.4) rectangle (0.25,-1);  % Weißer Rand um das Rechteck
    \fill[hellblau] (-0.25,-1.4) rectangle (0.25,-1);  % Füllt das Rechteck mit hellblau
    \draw[thick, white] (-0.5,-1.4) -- (0.5,-1.4);
    
    % Grüne Linie (verschoben)
    \draw[thick, gruen] (0,-1) -- (2.0,0.0);
    \draw[thick, gruen, dotted] (0,-1) -- (3.5,0.75);
    
    % Rote Linie (verschoben)
    \draw[thick, dunkelrot] (2.0,0.0) -- (3.0,2.0);
    
    % Glieder
    \fill[blau] (2.0,0.0) circle (0.1);
    \draw[white, thin] (2.0,0.0) circle (0.1);  % Weißer Rand um das Objekt
    \fill[blau] (0.0,-1.0) circle (0.1);
    \draw[white, thin] (0.0,-1.0) circle (0.1);  % Weißer Rand um das Objekt
    \fill[blau] (3.0,2.0) circle (0.1);
    \draw[white, thin] (3.0,2.0) circle (0.1);  % Weißer Rand um das Objekt
    
    % Winkel
    \draw [gruen, thick] (1,-1)  arc[start angle=0,end angle=45,radius=0.5cm];
	\node at (0.75, -0.85) {\tiny $\gruen{\theta_1}$};  % Winkelbeschriftung
	
	\draw [dunkelrot, thick] (3,0.5)  arc[start angle=0,end angle=80,radius=0.5cm];
	\node at (2.7, 0.6) {\tiny $\dunkelrot{\theta_2}$};  % Winkelbeschriftung
\end{tikzpicture}
\end{center}
%--------------------------------------------------------------------------------------------------------

Für die Koordinaten der Schwerpunkte gilt:
% Koordinaten der Schwerpunkte
\begin{align*}
    \begin{pmatrix}
        x_1 \\
        y_1 \\
        z_1
    \end{pmatrix}
    =
    \begin{pmatrix}
        a_1 \cdot \cos(\gruen{\theta_1}) \\
        a_1 \cdot \sin(\gruen{\theta_1}) \\
        0
    \end{pmatrix}
    \quad
    \begin{pmatrix}
        x_2 \\
        y_2 \\  
        z_2
    \end{pmatrix}
    =
    \begin{pmatrix}
        l_1 \cdot \cos(\gruen{\theta_1}) 
        + a_2 \cos(\gruen{\theta_1} + \dunkelrot{\theta_2}) \\
        l_1 \cdot \sin(\gruen{\theta_1}) 
        + a_2 \sin(\gruen{\theta_1} + \dunkelrot{\theta_2}) \\
        0
    \end{pmatrix}
\end{align*}

\subsubsection{Formulierung der Massenmatrix}

Für die Massenmatrix gilt hier:
% M(q)
\begin{align*}
    M(\textbf{q}) 
    = m_1 J_{v_{\gruen{1}}}^T J_{v_{\gruen{1}}} 
    + J_{\omega_{\gruen{1}}}^T I_1 J_{\omega_{\gruen{1}}}  
    + m_2 J_{v_{\dunkelrot{2}}}^T J_{v_{\dunkelrot{2}}} 
    + J_{\omega_{\dunkelrot{2}}}^T I_2 J_{\omega_{\dunkelrot{2}}} 
\end{align*}

Hierbei bezeichnet $J$ die Jacobi-Matrix, $I_i$ das Trägheitstensor 
und $m_i$ die Masse des $i$-ten Links.
% Berechnung der Jacobi-Matrizen J_v
\begin{align*}
    J_{v_1} &=
    \frac{\partial \textbf{r}_1}{\partial \textbf{q}}
    =
    \frac{\partial (x_1,y_1,z_1)}{\partial (\gruen{\theta_1}, \dunkelrot{\theta_2})}
    =
    \begin{pmatrix}
        \frac{\partial x_1}{\partial \gruen{\theta_1}} 
        & \frac{\partial x_1}{\partial \dunkelrot{\theta_2}} \\
        \frac{\partial y_1}{\partial \gruen{\theta_1}} 
        & \frac{\partial y_1}{\partial \dunkelrot{\theta_2}} \\
        \frac{\partial z_1}{\partial \gruen{\theta_1}} 
        & \frac{\partial z_1}{\partial \dunkelrot{\theta_2}} 
    \end{pmatrix}
    =
    \begin{pmatrix}
        -a_1 \sin(\gruen{\theta_1}) & 0 \\
        a_1 \cos(\gruen{\theta_1}) & 0 \\
        0 & 0
    \end{pmatrix}
    \\
    J_{v_2} &=
    \frac{\partial \textbf{r}_2}{\partial \textbf{q}}
    =
    \frac{\partial (x_2,y_2,z_2)}{\partial (\gruen{\theta_1}, \dunkelrot{\theta_2})}
    =
    \begin{pmatrix}
        \frac{\partial x_2}{\partial \gruen{\theta_1}} 
        & \frac{\partial x_2}{\partial \dunkelrot{\theta_2}} \\
        \frac{\partial y_2}{\partial \gruen{\theta_1}} 
        & \frac{\partial y_2}{\partial \dunkelrot{\theta_2}} \\
        \frac{\partial z_2}{\partial \gruen{\theta_1}}
        & \frac{\partial z_2}{\partial \dunkelrot{\theta_2}}
    \end{pmatrix} 
    =
    \begin{pmatrix}
        -l_1 \sin(\gruen{\theta_1}) - a_2 \sin(\gruen{\theta_1} + \dunkelrot{\theta_2}) 
        & -a_2 \sin(\gruen{\theta_1} + \dunkelrot{\theta_2}) \\
        l_1 \cos(\gruen{\theta_1}) + a_2 \cos(\gruen{\theta_1} + \dunkelrot{\theta_2}) 
        & a_2 \cos(\gruen{\theta_1} + \dunkelrot{\theta_2}) \\
        0 & 0
    \end{pmatrix}
\end{align*}

Für das Matrizenprodukt $J_{v_1}^T J_{v_1}$ gilt:
% Berechnung der Matrizenprodukte J_v1^T * J_v1
\begin{align*}
    J_{v_1}^T J_{v_1} 
    &=
    \begin{pmatrix}
        -a_1 \sin(\gruen{\theta_1}) & a_1 \cos(\gruen{\theta_1}) & 0 \\
        0 & 0 & 0
    \end{pmatrix}
    \cdot
    \begin{pmatrix}
        -a_1 \sin(\gruen{\theta_1}) & 0 \\
        a_1 \cos(\gruen{\theta_1}) & 0 \\
        0 & 0
    \end{pmatrix}
    \\
    &=
    \begin{pmatrix}
        a_1^2 \sin^2(\gruen{\theta_1}) + a_1^2 \cos^2(\gruen{\theta_1}) & 0 \\
        0 & 0
    \end{pmatrix}
    \\
    &=
    \begin{pmatrix}
        a_1^2 & 0 \\
        0 & 0
    \end{pmatrix}
\end{align*}

Für das Matrizenprodukt $J_{v_2}^T J_{v_2}$ gilt:
% Berechnung der Matrizenprodukte J_v2^T * J_v2
\begin{align*}
    J_{v_2}^T J_{v_2} 
    &=
    \begin{pmatrix}
        -l_1 \sin(\gruen{\theta_1}) - a_2 \sin(\gruen{\theta_1} + \dunkelrot{\theta_2}) 
        & l_1 \cos(\gruen{\theta_1}) + a_2 \cos(\gruen{\theta_1} + \dunkelrot{\theta_2}) 
        & 0 \\
        -a_2 \sin(\gruen{\theta_1} + \dunkelrot{\theta_2}) 
        & a_2 \cos(\gruen{\theta_1} + \dunkelrot{\theta_2})
        & 0
    \end{pmatrix}
    \\
    &\quad \cdot
    \begin{pmatrix}
        -l_1 \sin(\gruen{\theta_1}) - a_2 \sin(\gruen{\theta_1} + \dunkelrot{\theta_2}) 
        & -a_2 \sin(\gruen{\theta_1} + \dunkelrot{\theta_2}) \\
        l_1 \cos(\gruen{\theta_1}) + a_2 \cos(\gruen{\theta_1} + \dunkelrot{\theta_2}) 
        & a_2 \cos(\gruen{\theta_1} + \dunkelrot{\theta_2}) \\
        0 & 0
    \end{pmatrix}
    \\
    &=
    \begin{pmatrix}
        (l_1^2 + a_2^2 + 2 l_1 a_2  \cos(\dunkelrot{\theta_{2}})) 
        & (a_2^2 + l_1 a_2  \cos(\dunkelrot{\theta_{2}})) \\
        (a_2^2 + l_1 a_2  \cos(\dunkelrot{\theta_{2}})) 
        & a_2^2
    \end{pmatrix}
\end{align*}

Für die Winkelgeschwindigkeiten setzen wir   
$\bm{\omega}_1 = \begin{pmatrix}0 & 0 & \gruen{\theta_1'}\end{pmatrix}^T$ und 
$\bm{\omega}_2 = \begin{pmatrix} 0 & 0 &\gruen{\theta_1'} + \dunkelrot{\theta_2'}\end{pmatrix}^T$.
% Berechnung der Jacobi-Matrizen J_\omega
\begin{align*}
    J_{\omega_1} &=
    \frac{\partial \bm{\omega}_1}{\partial \dot{\textbf{q}}}
    =
    \frac{\partial (\omega_1)}{\partial (\gruen{\theta_1}', \dunkelrot{\theta_2}')}
    =
    \begin{pmatrix}
        0 & 0 \\
        0 & 0 \\
        \frac{\partial \omega_1}{\partial \gruen{\theta_1}'} 
        & \frac{\partial \omega_1}{\partial \dunkelrot{\theta_2}'} 
    \end{pmatrix}
    =
    \begin{pmatrix}
        0 & 0 \\
        0 & 0 \\
        \gruen{1} & \dunkelrot{0}
    \end{pmatrix}
    \\
    J_{\omega_2} &=
    \frac{\partial \bm{\omega}_2}{\partial \dot{\textbf{q}}}
    =
    \frac{\partial (\omega_2)}{\partial (\gruen{\theta_1}', \dunkelrot{\theta_2}')}
    =
    \begin{pmatrix}
        0 & 0 \\
        0 & 0 \\
        \frac{\partial \omega_2}{\partial \gruen{\theta_1}'} 
        & \frac{\partial \omega_2}{\partial \dunkelrot{\theta_2}'} 
    \end{pmatrix}
    =
    \begin{pmatrix}
        0 & 0 \\
        0 & 0 \\
        \gruen{1} & \dunkelrot{1}
    \end{pmatrix}
\end{align*}

Für den Trägheitstensor $I_i$ gilt:
% Allgemeiner Trägheitstensor
\begin{align*}
    I = 
    \begin{pmatrix}
        I_{xx} & I_{xy} & I_{xz} \\
        I_{yx} & I_{yy} & I_{yz} \\
        I_{zx} & I_{zy} & I_{zz}
    \end{pmatrix}   
    =
    \begin{pmatrix}
        \int (y^2 + z^2) \mathrm{d}m & -\int xy \mathrm{d}m & -\int xz \mathrm{d}m \\
        -\int yx \mathrm{d}m & \int (x^2 + z^2) \mathrm{d}m & -\int yz \mathrm{d}m \\
        -\int zx \mathrm{d}m & -\int zy \mathrm{d}m & \int (x^2 + y^2) \mathrm{d}m
    \end{pmatrix}
\end{align*}

Für symmetrische Körper (z.B. Zylinder, Kugel, Quader) 
gilt: $I_{xy} = I_{xz} = I_{yz} = 0$ also erhalten wir eine Diagonalmatrix. 
% Symmetrischer Trägheitstensor
\begin{align*}
    I_i 
    =
    \begin{pmatrix}
        \int_ y^2 \mathrm{d}m & 0 & 0 \\
        0 & \int x^2 \mathrm{d}m & 0 \\
        0 & 0 & \int (x^2 + y^2) \mathrm{d}m
    \end{pmatrix}
    =
    \begin{pmatrix}
        \int_V \rho y^2 \mathrm{d}V & 0 & 0 \\
        0 & \int_V \rho x^2 \mathrm{d}V & 0 \\
        0 & 0 & \int_V \rho (x^2 + y^2) \mathrm{d}V
    \end{pmatrix}
\end{align*}


Für die Massenmatrix $M(\textbf{q})$ gilt somit:
% Berechnung der Massenmatrix M(q)
\begin{align*}
    M(\textbf{q}) 
    &= m_1 J_{v_{\gruen{1}}}^T J_{v_{\gruen{1}}} 
    + J_{\omega_{\gruen{1}}}^T I_1 J_{\omega_{\gruen{1}}}  
    + m_2 J_{v_{\dunkelrot{2}}}^T J_{v_{\dunkelrot{2}}} 
    + J_{\omega_{\dunkelrot{2}}}^T I_2 J_{\omega_{\dunkelrot{2}}}    
    \\
    &=
    m_1 
    \begin{pmatrix}
        a_1^2 & 0 \\
        0 & 0
    \end{pmatrix}
    +
    \underbrace{
        \begin{pmatrix}
            0 & 0 & \gruen{1} \\
            0 & 0 & \dunkelrot{0}
        \end{pmatrix}
        \cdot
        \begin{pmatrix}
            I_{1,xx} & 0 & 0 \\
            0 & I_{1,yy} & 0 \\
            0 & 0 & I_{1,zz}
        \end{pmatrix}}_{
        = 
        \begin{pmatrix}
            0 & 0 & I_{1,zz} \\
            0 & 0 & 0
        \end{pmatrix}
    }
    \cdot   
    \begin{pmatrix}
        0 & 0 \\
        0 & 0 \\
        \gruen{1} & \dunkelrot{0}
    \end{pmatrix}
    \\
    &\quad +
    m_2
    \begin{pmatrix}
        (l_1^2 + a_2^2 + 2 l_1 a_2  \cos(\dunkelrot{\theta_{2}})) 
        & (a_2^2 + l_1 a_2  \cos(\dunkelrot{\theta_{2}})) \\
        (a_2^2 + l_1 a_2  \cos(\dunkelrot{\theta_{2}})) 
        & a_2^2
    \end{pmatrix}
    \\
    &\quad +
    \underbrace{
        \begin{pmatrix}
            0 & 0 & \gruen{1} \\
            0 & 0 & \dunkelrot{1}
        \end{pmatrix}
        \cdot
        \begin{pmatrix}
            I_{2,xx} & 0 & 0 \\
            0 & I_{2,yy} & 0 \\
            0 & 0 & I_{2,zz}
        \end{pmatrix}}_{
        = 
        \begin{pmatrix}
            0 & 0 & I_{2,zz} \\
            0 & 0 & I_{2,zz}
        \end{pmatrix}
    }   
    \cdot
    \begin{pmatrix}
        0 & 0 \\
        0 & 0 \\
        \gruen{1} & \dunkelrot{1}
    \end{pmatrix}
    \\
    &=
    \begin{pmatrix}
        m_1 a_1^2 + I_{1,zz} + m_2 \cdot (l_1^2 + a_2^2 + 2 l_1 a_2  \cos(\dunkelrot{\theta_{2}})) + I_{2,zz}
        & m_2 (a_2^2 + l_1 a_2  \cos(\dunkelrot{\theta_{2}})) +I_{2,zz}  \\
        m_2 (a_2^2 + l_1 a_2  \cos(\dunkelrot{\theta_{2}})) + I_{2,zz}
        & m_2 a_2^2 +I_{2,zz} 
    \end{pmatrix}
\end{align*}

Die Trägheitselemente $I_{i,xx}$ und $I_{i,yy}$ sind nicht relevant.
Für unser Model verwenden wir den Standardansatz eines 
dünnen Stabes der Länge $l_i$ und Masse $m_i$. 
Somit gilt $z= y \approx 0$ und wir erhalten: $I_{i,xx} = I_{i,yy} =0$
% Tensor langer dünner Stab
\begin{align*}
    I_{i,zz} \approx \int_{-\frac{l_{i}}{2}}^{\frac{l_{i}}{2}} 
    \underbrace{\rho}_{\frac{m_{i}}{l_{i}}} x^2 \mathrm{d}x
    = \frac{1}{3} \cdot \frac{m_{i}}{l_{i}} 
    \cdot 
    \left[ 
        \left( \frac{l_{i}}{2} \right)^3 -  \left( -\frac{l_{i}}{2} \right)^3 
    \right]
    = \frac{1}{12} m_{i} l_{i}^{2}
    \Rightarrow
    I_i     
    = 
    \begin{pmatrix}
        0 & 0 & 0 \\
        0 & 0 & 0 \\
        0 & 0 & \frac{1}{12} m_{i} l_{i}^{2}
    \end{pmatrix}
\end{align*}


\subsubsection{Aufstellen der Coriolis-Matrix}

Für die partiellen Ableitungen der kinetischen Energie 
$T(\textbf{q}, \dot{\textbf{q}}) = \frac{1}{2} \dot{\textbf{q}}^T M(\textbf{q}) \dot{\textbf{q}}$ gilt:
% Ableitung dT/d theta
\begin{align*}
    \frac{\partial T}{\partial \gruen{\theta_{1}}}
    &= 
    \frac{1}{2}  \frac{\partial M_{11}}{\partial \gruen{\theta_{1}}} 
    \gruen{\theta_{1}'}^2 
    + \frac{\partial M_{12}}{\partial \gruen{\theta_{1}}} \gruen{\theta_{1}'} 
    \dunkelrot{\theta_{2}'} 
    + \frac{1}{2} \frac{\partial M_{22}}{\partial \gruen{\theta_{1}}} 
    \dunkelrot{\theta_{2}'}(t)^2
    \\
    \frac{\partial T}{\partial \dunkelrot{\theta_{2}}}
    &= 
    \frac{1}{2} \frac{\partial M_{11}}{\partial \dunkelrot{\theta_{2}}} \gruen{\theta_{1}'}^2
    + \frac{\partial M_{12}}{\partial \dunkelrot{\theta_{2}}} \gruen{\theta_{1}'} \dunkelrot{\theta_{2}'} 
    + \frac{1}{2} \frac{\partial M_{22}}{\partial \dunkelrot{\theta_{2}}} \dunkelrot{\theta_{2}'}(t)^2
    \\
    \frac{\partial T}{\partial \gruen{\theta_{1}'}} 
    &= M_{\gruen{1 1}} \gruen{\theta_{1}'} + M_{\gruen{1}\dunkelrot{2}}  \dunkelrot{\theta_{2}'}
    \\
    \frac{\partial T}{\partial \dunkelrot{\theta_{2}'}} 
    &=
    M_{\gruen{1}\dunkelrot{2}} \gruen{\theta_{1}'} + M_{\dunkelrot{22}} \dunkelrot{\theta_{2}'}
\end{align*}

Für die zeitlichen Ableitungen der Größen gilt 
% Ableitung d/dt dT/d theta_1'
\begin{align*}
\frac{\mathrm{d}}{\dt} \frac{\partial T}{\partial \gruen{\theta_{1}'}} 
&=
\underbrace{\frac{\mathrm{d}}{\dt} M_{\gruen{1 1}}}_{\frac{\partial M_{11}}{\partial \theta_{1}} \theta_{1}' + \frac{\partial M_{11}}{\partial \theta_{2}} \theta_{2}'} \cdot \gruen{\theta_{1}'} + M_{\gruen{1 1}} \gruen{\theta_{1}''} + \underbrace{\frac{\mathrm{d}}{\dt} M_{\gruen{1}\dunkelrot{2}}}_{\frac{\partial M_{12}}{\partial \theta_{1}} \theta_{1}' + \frac{\partial M_{12}}{\partial \theta_{2}} \theta_{2}'} \cdot \dunkelrot{\theta_{2}'} + M_{\gruen{1}\dunkelrot{2}} \dunkelrot{\theta_{2}''} 
\\
&= \frac{\partial M_{11}}{\partial \gruen{\theta_{1}}} \gruen{\theta_{1}'}^2
 + \frac{\partial M_{11}}{\partial \dunkelrot{\theta_{2}}} \gruen{\theta_{1}'} \dunkelrot{ \theta_{2}'} 
 +  M_{\gruen{1 1}} \gruen{\theta_{1}''} 
 + \frac{\partial M_{12}}{\partial \gruen{\theta_{1}}} \gruen{\theta_{1}'} \dunkelrot{\theta_{2}'} 
 + \frac{\partial M_{12}}{\partial \dunkelrot{\theta_{2}}} \dunkelrot{\theta_{2}'}^2 
 + M_{\gruen{1}\dunkelrot{2}} \dunkelrot{\theta_{2}''} \\
&=
\frac{\partial M_{11}}{\partial \gruen{\theta_{1}}} \gruen{\theta_{1}'}^2
 + \left[ \frac{\partial M_{11}}{\partial \dunkelrot{\theta_{2}}} + \frac{\partial M_{12}}{\partial \gruen{\theta_{1}}} \right] \gruen{\theta_{1}'} \dunkelrot{ \theta_{2}'} 
 + \frac{\partial M_{12}}{\partial \dunkelrot{\theta_{2}}} \dunkelrot{\theta_{2}'}^2 
 +  M_{\gruen{1 1}} \gruen{\theta_{1}''} 
    + M_{\gruen{1}\dunkelrot{2}} \dunkelrot{\theta_{2}''}
\end{align*}

% Ableitung d/dt dT/d theta_2'
\begin{align*}
\frac{\mathrm{d}}{\dt} \frac{\partial T}{\partial \dunkelrot{\theta_{2}'}} 
&=
\underbrace{\frac{\mathrm{d}}{\dt} M_{\gruen{1}\dunkelrot{2}}}_{\frac{\partial M_{12}}{\partial \theta_{1}} \theta_{1}' + \frac{\partial M_{12}}{\partial \theta_{2}} \theta_{2}'} \cdot \gruen{\theta_{1}'} + M_{\gruen{1}\dunkelrot{2}} \gruen{\theta_{1}''} + \underbrace{\frac{\mathrm{d}}{\dt} M_{\dunkelrot{22}}}_{\frac{\partial M_{22}}{\partial \theta_{1}} \theta_{1}' + \frac{\partial M_{22}}{\partial \theta_{2}} \theta_{2}'} \cdot \dunkelrot{\theta_{2}'} + M_{\dunkelrot{22}} \dunkelrot{\theta_{2}''} \\
&= 
\frac{\partial M_{12}}{\partial \theta_{1}}  \gruen{\theta_{1}'}^2 
 + \frac{\partial M_{12}}{\partial \theta_{2}} \gruen{\theta_{1}'} \dunkelrot{\theta_{2}'} 
 + M_{\gruen{1}\dunkelrot{2}} \gruen{\theta_{1}''}
 + \frac{\partial M_{22}}{\partial \theta_{1}} \gruen{\theta_{1}'} \dunkelrot{\theta_{2}'} 
 + \frac{\partial M_{22}}{\partial \theta_{2}}  \dunkelrot{\theta_{2}'}^2 
 + M_{\dunkelrot{22}} \dunkelrot{\theta_{2}''}
\end{align*}

Es gilt 
% Ableitung d/dt dT/d theta_1' - dT/d theta_1
\begin{align*}
    \frac{\mathrm{d}}{\dt} \frac{\partial T}{\partial \gruen{\theta_{1}'}}  - \frac{\partial T}{\partial \gruen{\theta_{1}}}
    &=
    \left[ \frac{\partial M_{11}}{\partial \gruen{\theta_{1}}} - \frac{1}{2}  \frac{\partial M_{11}}{\partial \gruen{\theta_{1}}} \right] \gruen{\theta_{1}'}^2
    + \left[ \frac{\partial M_{11}}{\partial \dunkelrot{\theta_{2}}} + \frac{\partial M_{12}}{\partial \gruen{\theta_{1}}} -  \frac{\partial M_{12}}{\partial \gruen{\theta_{1}}} \right] \gruen{\theta_{1}'} \dunkelrot{ \theta_{2}'} 
    + \left[\frac{\partial M_{12}}{\partial \dunkelrot{\theta_{2}}} - \frac{1}{2} \frac{\partial M_{22}}{\partial \gruen{\theta_{1}}} \right] \dunkelrot{\theta_{2}'}^2 \\
    & \quad
    +  M_{\gruen{1 1}} \gruen{\theta_{1}''} 
    + M_{\gruen{1}\dunkelrot{2}} \dunkelrot{\theta_{2}''}
\end{align*}

% Ableitung d/dt dT/d theta_2' - dT/d theta_2
\begin{align*}
\frac{\mathrm{d}}{\dt} \frac{\partial T}{\partial \dunkelrot{\theta_{2}'}} - \frac{\partial T}{\partial \dunkelrot{\theta_{2}}}
&= 
\left[ \frac{\partial M_{12}}{\partial \gruen{\theta_{1}}} - \frac{1}{2} \frac{\partial M_{11}}{\partial \dunkelrot{\theta_{2}}} \right] \gruen{\theta_{1}'}^2 
+ \left[ \frac{\partial M_{12}}{\partial \dunkelrot{\theta_{2}}} + \frac{\partial M_{22}}{\partial \gruen{\theta_{1}}} - \frac{\partial M_{12}}{\partial \dunkelrot{\theta_{2}}} \right] \gruen{\theta_{1}'} \dunkelrot{\theta_{2}'} 
+ \left[ \frac{\partial M_{22}}{\partial \dunkelrot{\theta_{2}}} - \frac{1}{2} \frac{\partial M_{22}}{\partial \dunkelrot{\theta_{2}}} \right]  \dunkelrot{\theta_{2}'}^2 \\
& \quad + M_{\gruen{1}\dunkelrot{2}} \gruen{\theta_{1}''} + D_{\dunkelrot{22}} \dunkelrot{\theta_{2}''} \\
&=
\underbrace{\left[ \frac{\partial M_{12}}{\partial \gruen{\theta_{1}}} \gruen{\theta_{1}'} - \frac{1}{2} \frac{\partial M_{11}}{\partial \dunkelrot{\theta_{2}}} \gruen{\theta_{1}'} \right]}_{C_{\dunkelrot{2}\gruen{1}}} \gruen{\theta_{1}'}
+ \underbrace{\left[ \frac{1}{2} \frac{\partial M_{22}}{\partial \dunkelrot{\theta_{2}}}  \dunkelrot{\theta_{2}'} + \frac{\partial M_{22}}{\partial \gruen{\theta_{1}}} \gruen{\theta_{1}'} \right]}_{C_{\dunkelrot{22}}}  \dunkelrot{\theta_{2}'}
+ M_{\gruen{1}\dunkelrot{2}} \gruen{\theta_{1}''} + M_{\dunkelrot{22}} \dunkelrot{\theta_{2}''}
\end{align*}

Für die Einträge der Matrix $C$ gilt
% Berechnung der Einträge der Coriolis-Matrix C
\begin{align*}
C_{11} 
&=
\frac{1}{2}  \underbrace{\frac{\partial M_{11}}{\partial \gruen{\theta_{1}}} }_{= 0} \gruen{\theta_{1}'} + \underbrace{\frac{\partial M_{11}}{\partial \dunkelrot{\theta_{2}}}}_{ =  -2m_2 l_1 a_2 \sin(\dunkelrot{\theta_2})} \dunkelrot{ \theta_{2}'}
=
-2m_2 l_1 a_2 \sin(\dunkelrot{\theta_2}) \dunkelrot{ \theta_{2}'} 
\\
C_{12}
&=
\underbrace{\frac{\partial M_{12}}{\partial \dunkelrot{\theta_{2}}}}_{= - m_2 l_1 a_2 \sin(\dunkelrot{\theta_2})} \dunkelrot{\theta_{2}'} - \frac{1}{2} \underbrace{\frac{\partial M_{22}}{\partial \gruen{\theta_{1}}}}_{ = 0} \dunkelrot{\theta_{2}'}
=
- m_2 l_1 a_2 \sin(\dunkelrot{\theta_2}) \dunkelrot{\theta_{2}'}
\\
C_{21} 
&= \underbrace{\frac{\partial M_{12}}{\partial \gruen{\theta_{1}}}}_{= 0} \gruen{\theta_{1}'} - \frac{1}{2} \underbrace{\frac{\partial M_{11}}{\partial \dunkelrot{\theta_{2}}}}_{= -2m_2 l_1 a_2 \sin(\dunkelrot{\theta_2}) } \gruen{\theta_{1}'}
=
m_2 l_1 a_2 \sin(\dunkelrot{\theta_2})\gruen{\theta_{1}'}
\\
C_{22} 
&=
\frac{1}{2} \underbrace{\frac{\partial M_{22}}{\partial \dunkelrot{\theta_{2}}}}_{ = 0} \dunkelrot{\theta_{2}'} + \underbrace{ \frac{\partial M_{22}}{\partial \gruen{\theta_{1}}}}_{ = 0} \gruen{\theta_{1}'}
= 0
\end{align*}

Es gilt somit 
% Resultat Coriolis-Matrix C(q, qdot) * qdot
\begin{align*}
C(\textbf{q}, \dot{\textbf{q}}) \cdot \dot{\textbf{q}}
=
\begin{pmatrix}
    -2m_2 l_1 a_2 \sin(\dunkelrot{\theta_2}) \dunkelrot{ \theta_{2}'} 
    &
    - m_2 l_1 a_2 \sin(\dunkelrot{\theta_2}) \dunkelrot{\theta_{2}'} 
    \\
    m_2 l_1 a_2 \sin(\dunkelrot{\theta_2})\gruen{\theta_{1}'}
    &
    0
\end{pmatrix}
\begin{pmatrix}
    \gruen{\theta_{1}'} \\
    \dunkelrot{\theta_{2}'} 
\end{pmatrix}
\end{align*}

\subsubsection{Gravitationsterme}

Für die potenzielle Energie $V$ gilt:
% V = m*g*h als Potenzielle Energie
\begin{align*}
    V 
    &= m_1 \cdot g \cdot y_1 + m_2 \cdot g \cdot y_2 
    = m_1  g  a_1 \sin(\gruen{\theta_1}) 
    + m_2  g \left( l_1 \sin(\gruen{\theta_1}) 
    + a_2 \sin(\gruen{\theta_1} + \dunkelrot{\theta_2}) \right)
\end{align*}

Für die partiellen Ableitungen erhalten wir
% Berechnung der Gravitationsterme G(q) = dV/dq
\begin{align*}
    G(\textbf{q})
    =
    \frac{\partial V}{\partial \textbf{q}}
    =
    \begin{pmatrix}
        \frac{\partial V}{\partial \gruen{\theta_1}} \\
        \frac{\partial V}{\partial \dunkelrot{\theta_2}}
    \end{pmatrix}
    =
    \begin{pmatrix}
        m_1  g  a_1 \cos(\gruen{\theta_1}) + m_2  g \left( l_1 \cos(\gruen{\theta_1}) 
        + a_2 \cos(\gruen{\theta_1} + \dunkelrot{\theta_2}) \right) \\
        m_2  g \left( l_1 \cos(\gruen{\theta_1}) + a_2 \cos(\gruen{\theta_1} 
        + \dunkelrot{\theta_2}) \right)
    \end{pmatrix}
\end{align*}

% Hier Berechnung für zusätzliches Gewicht an link2
\newpage
\subsection{Einfaches Greifobjekt}
Wir modellieren ein Variables Greifobjekt, welches am Ende von Link 2 gehalten wird. 
Wir betrachten nachfolgend das Greifobjekt als Punktmasse mit Masse $m_G$. 

% Greifobjekt
%--------------------------------------------------------------------------------------------------------
\begin{center}
    \begin{tikzpicture}[scale = 1.75, every node/.style={transform shape}]
        % Koordinatensystem
        %\draw[very thin, gray] (-5,-5) grid (5,5); % Raster
        %\draw[->] (-5,0) -- (5,0); % x-Achse
        %\draw[->] (0,-5) -- (0,5); % y-Achse
        
        % Erste Zeichnung (Original)
        \fill[hellblau] (-4,-1) circle (0.25);  % Füllt das Objekt mit hellblau
        %\draw[white, line width = 0.2mm] (-4,-1) circle (0.25);  % Weißer Rand um das Objekt
        
        %\draw[white, line width = 0.2mm] (-4.25,-1.4) rectangle (-3.75,-1);  % Weißer Rand um das Rechteck
        \fill[hellblau] (-4.25,-1.4) rectangle (-3.75,-1);  % Füllt das Rechteck mit hellblau
        %\draw[thick, white] (-4.5,-1.4) -- (-3.5,-1.4);
        
        % Grüne Linie angepasst (Endpunkt auf (-2.0, 0.0))
        \draw[thick, gruen] (-4,-1) -- (-3.0,1.0);
        
        % Rote Linie angepasst (von (-2.0, 0.0) nach (-1.0, 2.0))
        \draw[thick, dunkelrot] (-3.0,1.0) -- (-1.0,1.0);
        
        % Glieder
        \fill[blau] (-3.0,1.0) circle (0.1);
        \draw[white, thin] (-3.0,1.0) circle (0.1);  % Weißer Rand um das Objekt
        \fill[blau] (-4.0,-1.0) circle (0.1);
        \draw[white, thin] (-4.0,-1.0) circle (0.1);  % Weißer Rand um das Objekt
        
        % Beschriftungen
        \node at (-3, -0.75) {\tiny $\gruen{l_1}$};  % 
        \node at (-1.5, 1.5) {\tiny $\dunkelrot{l_2}$};  % 
        
        % Schwerpunkte 
        %\fill[white] (-3.0,-0.5) circle (0.05);
        %\fill[white] (-1.5,1) circle (0.05);
        
        %\draw[white, <->] (-1.85,0.15) -- (-1.5,0.85);
        %\draw[white, <->] (-3.85,-0.85) -- (-3.1,-0.45);
        
        % Beschriftungen
        \node at (-3.5, -0.4) {\tiny $a_1$};  % 
        \node at (-1.5, 0.5) {\tiny $a_2$};  % 

        \draw[dunkelrot, thick] (-1.0,0.9) -- (-1.0, 1.1);
        \draw[dunkelrot, thick] (-1.0,0.9) -- ++(0.1,0.0);
        \draw[dunkelrot, thick] (-1.0,1.1) -- ++(0.1,0.0);
        \node at (-0.6,0.7) {\tiny $m_G$}; % Beschriftung

        % Erste Zeichnung (Original) verschoben um +4 in x
        \fill[hellblau] (0,-1) circle (0.25);  % Füllt das Objekt mit hellblau
        \draw[white, line width = 0.2mm] (0,-1) circle (0.25);  % Weißer Rand um das Objekt
        
        \draw[white, line width = 0.2mm] (-0.25,-1.4) rectangle (0.25,-1);  % Weißer Rand um das Rechteck
        \fill[hellblau] (-0.25,-1.4) rectangle (0.25,-1);  % Füllt das Rechteck mit hellblau
        \draw[thick, white] (-0.5,-1.4) -- (0.5,-1.4);
        
        % Grüne Linie angepasst (Endpunkt auf (1.0, 1.0))
        \draw[thick, gruen] (0,-1) -- (1.0,1.0);
        
        % Rote Linie angepasst (von (1.0, 1.0) nach (3.0, 1.0))
        \draw[thick, dunkelrot] (1.0,1.0) -- (3.0,1.0);

        % Orange Linie zum Greifobjekt
        \draw[thick, orange] (3.0,1.0) -- (3.3,1.4);
        \draw[thick, orange] (3.364,1.352) -- (3.236,1.448);
        \draw[thick, orange] (3.364,1.352) -- ++(0.072,0.096);
        \draw[thick, orange] (3.236,1.448) -- ++(0.072,0.096);

        % Winkel
        \draw [orange, thick] (3.35,1.0)  arc[start angle=0,end angle=45,radius=0.35cm];
	    \node at (3.25, 0.9) {\tiny $\orange{\theta_3}$};  % Winkelbeschriftung
        \draw[thick, dunkelrot, dotted] (3,1) -- (4.0,1.0);
        \node at (3.0, 1.3) {\tiny $\orange{l_3}$};

        \fill[blau] (3.0,1.0) circle (0.1);
        \draw[white, thin] (3.0,1.0) circle (0.1);  % Weißer Rand um das Objekt

        % Glieder
        \fill[blau] (1.0,1.0) circle (0.1);
        \draw[white, thin] (1.0,1.0) circle (0.1);  % Weißer Rand um das Objekt
        \fill[blau] (0.0,-1.0) circle (0.1);
        \draw[white, thin] (0.0,-1.0) circle (0.1);  % Weißer Rand um das Objekt
    \end{tikzpicture}
\end{center}
%--------------------------------------------------------------------------------------------------------

Für die Masse von Link 2 inklusive Greifobjekt gilt:
% m_i' = m_i + m_G
\begin{align*}
    m_{i}' = m_{i} + m_{G}
\end{align*}

Für den Schwerpunkt von Link 2 inklusive Greifobjekt gilt:
% a_i' = (m_i * a_i + m_G * d) / (m_i + m_G)
\begin{align*}
    \textbf{a}_{2}'
    &= \frac{m_{i} \cdot \textbf{a}_{2} + m_{G} \cdot \textbf{d}}{m_{i} + m_{G}}
    \quad
    \textbf{d}(\orange{\theta_3}) =
    \begin{pmatrix}
        l_2 + r \cdot \cos(\orange{\theta_3}) \\
        r \cdot \sin(\orange{\theta_3}) \\
        0
    \end{pmatrix}
\end{align*}

Für den Trägheitstensor von Link 2 inklusive Greifobjekt gilt:
% I_i' = I_i + m_i * S(a_i - a_i') + m_G * S(d - a_i')
\begin{align*}
    I_{i}' 
    &= 
    I_{i} + m_{i} \cdot S(\textbf{a}_{i} - \textbf{a}_{i}') 
    + m_{G} \cdot S(\textbf{d} - \textbf{a}_{i}')
    \quad
    S(\textbf{v}) =
    \begin{pmatrix}
        v_{y}^2 + v_{z}^2 & -v_{x} v_{y} & -v_{x} v_{z} \\
        -v_{x} v_{y} & v_{x}^2 + v_{z}^2 & -v_{y} v_{z} \\
        -v_{x} v_{z} & -v_{y} v_{z} & v_{x}^2 + v_{y}^2
    \end{pmatrix}
\end{align*}

Wobei $S(\textbf{v})$ die Steiner-Matrix ist. 
Wir erhalten die Matrix direkt aus der Definition des Trägheitstensors (\ref{eq:traegheitstensor}) einer Punktmasse
mit $\rho(\textbf{r}) = m \cdot \delta(\text{r}-\textbf{d})$.
Für die Verschiebungsvektoren gilt:
% Berechnung a_2'
\begin{align*}
    \textbf{d}
    =
    \begin{pmatrix}
        l_2 \\
        0 \\
        0 
    \end{pmatrix}
    \textbf{a}_{2} =
    \begin{pmatrix}
        \frac{l_2}{2} \\
        0 \\
        0 
    \end{pmatrix}
    \Rightarrow
    \textbf{a}_{2}'
    =
    \frac{l_2}{m_{2} + m_{G}}
    \begin{pmatrix}
        \frac{m_{2}}{2} + m_{G} \\
        0 \\
        0
    \end{pmatrix}
\end{align*}

Für die Steiner-Matrizen gilt entsprechend
% Berechnung der Steiner-Matrizen
\begin{align*}
    S(\textbf{a}_{2} - \textbf{a}_{2}')
    =
    \begin{pmatrix}
        0 & 0 & 0 \\
        0 & \left[\frac{l_2}{2} - a_{2}'\right]^2 & 0 \\
        0 & 0 & \left[\frac{l_2}{2} - a_{2}'\right]^2
    \end{pmatrix}
    \quad
    S(\textbf{d} - \textbf{a}_{2}')
    =
    \begin{pmatrix}
        0 & 0 & 0 \\
        0 & \left[l_2^2 - a_{2}'\right]^2 & 0 \\
        0 & 0 & \left[l_2^2 - a_{2}'\right]^2
    \end{pmatrix}
\end{align*}

Somit erhalten wir für den Trägheitstensor von Link 2 inklusive Greifobjekt:
% Berechnung I_2'
\begin{align*}
    I_{2}'
    &=
    \begin{pmatrix}
        I_{2,xx} & 0 & 0 \\
        0 & I_{2,yy} 
        + m_2 \cdot \left[\frac{l_2^2}{2} - a_{2}'\right]^2 
        + m_G \cdot \left[\frac{l_2^2}{2} - a_{2}'\right]^2 
        & 0 \\
        0 & 0 & I_{2,zz} 
        + m_2 \cdot \left[\frac{l_2^2}{2} - a_{2}'\right]^2 
        + m_G \cdot \left[\frac{l_2^2}{2} - a_{2}'\right]^2
    \end{pmatrix}
\end{align*}

Für die Koordinaten des Schwerpunktes von Link 3 gilt:
% Berechnung der Koordinaten des Schwerpunktes von Link 3
\begin{align*}
    \textbf{r}_3
    &=
    \begin{pmatrix}
        l_1 \cos(\gruen{\theta_1}) 
        + l_2 \cos(\gruen{\theta_1} + \dunkelrot{\theta_2}) 
        + l_3 \cos(\gruen{\theta_1} + \dunkelrot{\theta_2} + \orange{\theta_3}) \\
        l_1 \sin(\gruen{\theta_1}) 
        + l_2 \sin(\gruen{\theta_1} + \dunkelrot{\theta_2}) 
        + l_3 \sin(\gruen{\theta_1} + \dunkelrot{\theta_2} + \orange{\theta_3}) \\
        0
    \end{pmatrix}
\end{align*}
Berechnung der Jacobi-Matrizen für Link 3 bezüglich der Geschwindigkeiten:
% Jacobi-Matrizen für Link 3
\begin{align*}
    J_{v_3} &=
    \frac{\partial \textbf{r}_3}{\partial \textbf{q}}
\end{align*}

Für die Winkelgeschwindigkeit des Links 3 setzen wir 
$\bm{\omega_3}= 
\begin{pmatrix} 
    0 & 0 & \gruen{\theta_1'} + \dunkelrot{\theta_2'} + \orange{\theta_3'} 
\end{pmatrix}$

und erhalten für die Jacobi-Matrix bezüglich der Winkelgeschwindigkeit:
% Jacobi-Matrizen für Link 3
\begin{align*}
    J_{\omega_3} &=
    \frac{\partial \bm{\omega_3}}{\partial \dot{\textbf{q}}} 
    =
    \begin{pmatrix}
        0 & 0 & 0 \\
        0 & 0 & 0 \\
        1 & 1 & 1
    \end{pmatrix}
\end{align*}


\newpage


%------------------------------------------------------------------------------------------------------------
\section{Theorie optimaler Steuerungungsprobleme}

Wir halten uns an die Definitionen in \cite{gerdts2012}

\begin{definition}[Optimales Steuerungsproblem]
    Seien $t_0<t_f$ feste Zeiten und 
    %
    \begin{align*}
        \Phi &: \mathbb{R}^{n_{\bm{x}}} \to \mathbb{R}, \\
        L &: [t_0,t_f] \times \mathbb{R}^{n_{\bm{x}}}  \to \mathbb{R},  \\
        f &: [t_0,t_f] \times \mathbb{R}^{n_{\bm{x}}} \times \mathbb{R}^{n_{\bm{u}}} 
        \to \mathbb{R}^{n_{\bm{x}}}
    \end{align*}
    %
    hinreichend glatte Funktionen und $\mathcal{U} \subset \mathbb{R}^{n_{\bm{u}}}$ 
    eine abgeschlossene konvexe nichtleere Menge. 
    %
    \begin{align*}
        \min_{\bm{u}} \int_{t_0}^{t_f} L(t,\bm{x}(t),\bm{u}(t)) \dt + \Phi(\bm{x}(t_f))
    \end{align*}
    %
    mit $\bm{x} \in W_{1,\infty}^{n_{\bm{x}}}([t_0,t_f]), 
    \bm{u} \in L_{\infty}^{n_{\bm{u}}}([t_0,t_f])$ und 
    %
    \begin{align*}
        \dot{\bm{x}}(t) = f(t,\bm{x}(t),\bm{u}(t)), \quad \bm{x}(t_0) = \bm{x}_0, \quad \bm{u}(t) \in \mathcal{U} \text{ f.ü. } t \in [t_0,t_f]
    \end{align*}
\end{definition}

Wir betrachten ein Steuerungsproblem und definieren die Value-Function $V$ mittels
%
\begin{align*}
V(x(t),\dunkelgelb{t}) := \min_{\dunkelrot{u}} \left\{ \int_{\dunkelgelb{t}}^{t_f} L(s,x(s),\dunkelrot{u}(s)) \ds + \Phi(x(t_f)) \right\}
\end{align*}

\begin{satz}[Hamilton-Jacobi-Bellman Gleichung]
Die Value-Function erfüllt
%
\begin{align*}
0 = \frac{\partial V(x, t)}{\partial t} + \min_{u} \left[ L(t,x, u) + \frac{\partial V(x,t)}{\partial x} \cdot f(t,x,u) \right], \quad x \in \mathbb{R}^n, \, t_0 \leq t \leq t_f
\end{align*}
%
mit den Randbedingungen
%
\begin{align*}
V(x, \lila{t_f}) = \Phi(x(\lila{t_f})) \quad V(x, \dunkelgelb{t_0} ) = \min_u \int_{\dunkelgelb{t_0}}^{t_f} L(t,x(t),u(t)) \dt + \Phi(x(t_f)) 
\end{align*}
\end{satz}

%------------------------------------------------------------------------------------------------------------
\proof{ 

Die Value-Function $V(x,t)$ beschreibt die minimalen Kosten von Zustand $x$ zum Endzeitpunkt $t_f$.  Für $s \in [t,t_f]$ gilt
%
\begin{itemize}
\item Systemdynamik
%
\begin{align*}
x'(s) = f(x(s), u(s)), \quad x(t) = x
\end{align*}

\item Kostenfunktional
%
\begin{align*}
J(u) = \int_t^{t_f} L(s,x(s),u(s)) \ds + \Phi(x(t_f))
\end{align*}
\end{itemize}

Angenommen, wir wechseln zum Zeitpunkt $t+h$ zu einer optimalen Steuerung, dann gilt für unser Zielfunktional 
%
\begin{align*}
J(u) 
&= \int_{t}^{t+h} L(s,x(s),u(s)) \ds + \dunkelrot{\int_{t+h}^{t_f} L(s,x(s),u(s)) \ds + \Phi(x(t_f)) } \\
&= \int_{t}^{t+h} L(s,x(s),u(s)) \ds + \dunkelrot{V(x(t+h),t+h)} 
\end{align*}

Da $V(x(t), t)$ die minimalen Kosten vom Zustand $x$ zur Zeit $t$ angibt, gilt:
%
\begin{align*}
V(x(t),t) 
&\leq \int_{t}^{t+h} L(s,x(s),u(s)) \ds + \dunkelrot{V(x(t+h),t+h)} \\
\Rightarrow 0 &\leq \int_{t}^{t+h} L(s,x(s),u(s)) \ds + \dunkelrot{V(x(t+h),t+h)} - V(x(t),t) 
\end{align*}

Im Grenzübergang $h \to 0$ erhalten wir
%
\begin{align*}
0 &\leq \gruen{\lim_{h \to 0}  \frac{1}{h}  \int_{t}^{t+h} L(s,x(s),u(s)) \ds} + \dunkelrot{\lim_{h \to 0} \frac{ V(x(t+h),t+h) - V(x,t) }{h}} \\
&= \gruen{ L(t,x(t),u(t))} + \dunkelrot{\frac{\mathrm{d}}{\dt} V(x(t),t)} \\
&= \gruen{ L(t,x(t),u(t))} + \dunkelrot{\frac{\partial V}{\partial t}  + \frac{\partial V}{\partial x} \cdot x'(t)} \\
&= \gruen{ L(t,x(t),u(t))} + \dunkelrot{\frac{\partial V}{\partial t}  + \frac{\partial V}{\partial x} \cdot f(t,x(t), u(t))}
\end{align*}

Durch Minimierung bezüglich $u$ erhalten wir die Gleichheit und somit
%
\begin{align*}
0 &= \frac{\partial V}{\partial t} + \orange{\min_{u} \left[L(t,x(t),u(t)) +  \frac{\partial V}{\partial x} \cdot f(t,x(t), u(t))   \right]} \\
&= \frac{\partial V}{\partial t} + \orange{\textsf{H}\left(t,x(t) , \frac{\partial V}{\partial x}(t,x(t)) \right)}
\end{align*}
}
%------------------------------------------------------------------------------------------------------------


\subsection{Herleitung des Minimumprinzips}

Berechnung der totalen Ableitung der Value-Function $V$ entlang der Charakteristik $x(t)$:
%
\begin{align*}
\frac{dV}{dt} &= \blau{\frac{\partial V}{\partial x}} \frac{dx}{dt} + \lila{\frac{\partial V}{\partial t}} = \blau{p(x,t)} \cdot x'(t) \lila{- H\left( t,x , \frac{\partial V}{\partial x} \right)} \\
\end{align*}

Berechnung der totalen Ableitung von $p = \frac{\partial V}{\partial x}$:
%
\begin{align*}
\frac{dp}{dt} &= \frac{\mathrm{d}}{\dt} \left( \frac{\partial V}{\partial x} \right) =  \frac{\partial^2 V}{\partial x^2} \cdot x'(t) + \rot{\frac{\partial V}{\partial x \partial t}}
\end{align*}

Differenzieren der HJB bezüglich $x$ liefert
%
\begin{align*}
0 = \frac{\mathrm{d}}{\dx} \left[ \frac{\partial V}{\partial t} + H\left(t,x, \orange{ \frac{\partial V}{\partial x}} \right) \right] = \rot{\frac{\partial V}{\partial t \partial x}} + \frac{\partial H}{\partial x} + \frac{\partial H}{\partial \orange{p}} \cdot \frac{\partial^2 V}{\partial x^2}
\end{align*}

Einsetzen der Gleichung liefert:
%
\begin{align*}
\frac{dp}{dt} =  \frac{\partial^2 V}{\partial x^2} \cdot x'(t) + \rot{\left(- \frac{\partial H}{\partial x} - \frac{\partial H}{\partial p} \cdot \frac{\partial^2 V}{\partial x^2} \right)}
= \frac{\partial^2 V}{\partial x^2} \cdot \gruen{\left(x'(t) - \frac{\partial H}{\partial p} \right)} - \frac{\partial H}{\partial x}
\end{align*}

Wir erhalten somit die bekannten, notwendigen Optimalitätsbedingungen:
%
\begin{align*}
p'(t) = -H_x , \quad \gruen{x'(t) = f(t,x,u^*)}, \quad u^* = \textsf{argmin} \left[ L(t,x,u)+ p(t,x)f(t,x,u) \right]
\end{align*}

%-----------------------------------------------------------------------------------------------------------
\subsection{Anwendung auf Linear-Quadratische Probleme}

\subsubsection{Finite time problem}

Gegeben sei das Steuerungsproblem 
%
\begin{align*}
\min_u &  \int_{\dunkelgelb{t_0}}^{t_f} x^T Q x + u^T R u \dt + \underbrace{(x(t_f)- x_f)^T S (x(t_f)- x_f)}_{x(t_f)^T S x(t_f) -2x_{f}^T S x(t_f) +x_{f}^{T} S x_f} 
= V(\dunkelgelb{t_0},x(\dunkelgelb{t_0})) + x_{f}^{T} S x_f
\end{align*}

unter der Nebenbedingung $x'(t) = Ax+Bu$ und $x(t_0) = x_0$. Der Term $x_{f}^{T} S x_f$ ist unabhängig von $u$ und kann bei der Minimierung weggelassen werden.

% Aufbau der Value Funktion und HJB
%----------------------------------------------------------------------------------------------
%\begin{itemize}
%\item Als Ansatz für die Value-Funktion wählen wir 
%
\begin{align*}
V(x,t) 	&= x^T K(t) x + 2 s(t)^T x +r(t) \quad V(x,t_f) = x(t_f)^T K(t_f) x(t_f) + 2 s(t_f)^T x(t_f) + r(t_f)
\end{align*}

Wir erhalten für die Endwerte
%
\begin{align*}
K(t_f) = S, \quad s(t_f) = - S x_f \quad r(t_f) = 0
\end{align*}

%\item Für die partiellen Ableitungen gilt:
%
\begin{align*}
V_t = x^T K'(t) x + 2 s'(t)^T x +r'(t) \quad \rot{V_x = 2K(t)x +2s(t)} 
\end{align*}


%\item Minimierung bezüglich $u$ liefert 
%
\begin{align*}
\frac{\partial}{\partial u} \left[ x^T Q x + u^T R u + V_x^T \cdot (Ax+Bu) \right] &= 2Ru + B^T \cdot V_x \Rightarrow u^{*}(t) = - \frac{1}{2} R^{-1} B^T \rot{V_x}
\end{align*}

%\item Da $R^T = R$ gilt, folgt $(R^{-1})^T = R^{-1}$ und $K^T = K$, da Hesse-Matrix von $V$. %   weiter mit $u^{*}$
%

% Hier weitermachen und ausmultiplizieren
%%%%%%%%%%%%%%%%%%%%%%%%%%%%%%%%%%%%%%%%%%%%%%%%%%%%%%%%%%
\begin{align*}
(u^{*})^T R u^{*} 
&=  \left(- \frac{1}{2} R^{-1} B^T V_x \right)^T R \left(- \frac{1}{2} R^{-1} B^T V_x \right) \\
&= \frac{1}{4} \cdot \left( R^{-1} B^T \rot{V_x} \right)^T R \left( R^{-1} B^T \rot{V_x} \right) \\
&= \frac{1}{4} \cdot \left(\rot{2} R^{-1} B^T \rot{(K(t)x + s(t))} \right)^T R \left( \rot{2} R^{-1} B^T \rot{(K(t)x + s(t))} \right) \\
&=  (K(t)x + s(t))^T \orange{B R^{-1}} \gruen{ R R^{-1}} \orange{B^T} (K(t)x + s(t)) \\
&=    < K(t)x + s(t) , \orange{P} (2K(t)x + s(t)) >  \\
&=  < K(t)x, P K(t)x > +  \underbrace{< K(t)x, Ps(t) >}_{< s(t),  \orange{P^{T}} K(t)x  >} + < s(t), P K(t)x  > + <s(t), P s(t) >  \\
&= x^T K(t)^T P K(t)x + 2 \cdot s(t)^T P K(t) x +  s(t)^T P s(t)
\end{align*}

mit $P = B R^{-1} B^T$ und $P^T = P$.  
%

\begin{align*}
V_x^{T}  \left( - \frac{1}{2} B R^{-1} B^T V_x \right)
&= (2K(t)x +2s(t))^T \left( - \frac{1}{2} B R^{-1} B^T (2K(t)x +2s(t)) \right) \\
&= -2 \cdot (K(t)x + s(t))^T P (K(t)x + s(t)) \\
&= -2 \cdot x^T K(t)^T P K(t)x -4 \cdot s(t)^T P K(t)x -2 \cdot s(t)^T P s(t) 
\end{align*}

\begin{align*}
V_x^T Ax &= 2 (K(t)x + s(t))^T A x = 2 x^T K(t)^T A x + 2 s(t)^T Ax = x^T (K(t)^T A + A^T K(t) ) x + 2 s(t)^T Ax
\end{align*}


\begin{align*}
- V_t 	&= \min_u \left[ x^T Q x + u^T R u + V_x^T \cdot (Ax+Bu) \right] 
\end{align*}



\begin{align*}
- V_t 								&=  x^T Q x + (u^{*})^T R u^{*}  + V_x^T \cdot \left( Ax- \frac{1}{2} B R^{-1} B^T V_x \right) 
\\
\dunkelrot{-} x^T \dunkelrot{K'(t)} x - 2 \blau{s'(t)}^T x  -\orange{r'(t)} 	&= x^T (\dunkelrot{Q-KPK+ KA + A^T K}) x - 2 \cdot (\blau{(P K -A)^T s(t)})^T x - \orange{ s(t)^T P s(t)}
\end{align*}

\begin{align*}
K'(t) &= -Q+K(t) B R^{-1} B^T K(t) - K(t) A - A^T K(t) \quad K(t_f) = S \\
s'(t) &= (K(t) B R^{-1} B^T  - A^T) s(t) \quad s(t_f) = -S x_f  \\
r'(t) &= s(t)^T B R^{-1} B^T s(t) \quad r(t_f) = 0
\end{align*}

\subsubsection{infinite time problem}

Im Grenzfall $t \to \infty$ betrachten wir das Steuerungsproblem
%
\begin{align*}
\min_u &  \int_{\dunkelgelb{t_0}}^{\infty} (x-x_f)^T Q (x-x_f)  + u^T R u \dt 
= V(x(\dunkelgelb{t_0}))
\end{align*}

unter der Nebenbedingung $x'(t) = Ax+Bu$ und $x(t_0) = x_0$. Als Ansatz für die Value Funktion wählen wir erneut 
%
\begin{align*}
V(x) = x^T K x + 2s^T x + r \Rightarrow V_x = 2Kx + 2s
\end{align*}

Es gilt erneut 
%
\begin{align*}
\frac{\partial}{\partial u} \left((x-x_f)^T Q (x-x_f)  + u^T R u  + V_x^T (Ax+Bu) \right) = 2Ru + B^T V_x  = 0 \Rightarrow u^{*} = -\frac{1}{2} R^{-1} B^T V_x 
\end{align*}


\begin{align*}
(x-x_f)^T Q (x-x_f) 
&= x^T Q x - 2 x_f^T Q x + x_f^T Q x_f
\end{align*}

\begin{align*}
0 &= x^T ( Q - KPK + KA + A^T K ) x -2 ((PK-A)^T s + Qx_f)^T x - s^T (P) s + x_f^T Q x_f
\end{align*}

\newpage
%----------------------------------------------------------------------------------------------


\section{Linearisierung der Dynamik}

Wir gehen analog zu \cite{ICIIECS2017} vor.
% Arbeitsbereich
Wir definieren den Zustand 
$\textbf{x} = 
\begin{bmatrix}
    \textbf{q} \\ \dot{\textbf{q}}
\end{bmatrix}$
und schreiben unsere Dynamikgleichung wir folgt um 
% Linearisierung der allgemeinen Dynamik x = [q; q']
\begin{align*}
    M(\textbf{q}) \ddot{\textbf{q}} 
    + C(\textbf{q}, \dot{\textbf{q}}) \dot{\textbf{q}} 
    + G(\textbf{q}) = \bm{u} 
    \Rightarrow
    \underbrace{
            \begin{bmatrix}
                \dot{\textbf{q}} \\ \ddot{\textbf{q}}
            \end{bmatrix}
    }_{
        \dot{\textbf{x}}
    }
    =
    \underbrace{
        \begin{bmatrix}
            \dot{\textbf{q}} \\
            M^{-1}(\textbf{q}) 
            \left[ 
                \bm{u} - C(\textbf{q}, \dot{\textbf{q}}) \dot{\textbf{q}} - G(\textbf{q}) 
                \right]
        \end{bmatrix}
    }_{
        f(\textbf{x}, \bm{u})
    }
\end{align*}

Wir linearisieren um den Arbeitspunkt $(\bm{x}^*, \bm{u}^*)$
% Linearisierung x' = f(x,u) approx f(x*,u*) + df/dx (x-x*) + df/du (u-u*)
\begin{align*}
    \dot{\bm{x}} 
    &= f(\bm{x},\bm{u}) \\
    &\approx \underbrace{f(\bm{x}^*,\bm{u}^*)}_{= 0} + 
    \underbrace{\frac{\partial f}{\partial x}(\bm{x}^*,\bm{u}^*)}_{A(\bm{x}^*,\bm{u}^*)}(\bm{x}-\bm{x}^*) 
    + \underbrace{\frac{\partial f}{\partial u}(\bm{x}^*,\bm{u}^*)}_{B(\bm{x}^*,\bm{u}^*)}(\bm{u}-\bm{u}^*) 
    \\
    &= A(\bm{x}^*,\bm{u}^*) \bar{\bm{x}} + B(\bm{x}^*,\bm{u}^*) \bar{\bm{u}}
\end{align*}

Wobei $\bar{\bm{x}}(t) = \bm{x}(t) -\bm{x}^*$ und $\bar{\bm{u}}(t) = \bm{u}(t)-\bm{u}^*$ gilt. 
Für die Ableitung der Größen gilt
% Ableitung der Abweichung
\begin{align*}
    \frac{\mathrm{d} \bar{\bm{x}}}{\dt}
    = \dot{\bm{x}} - \underbrace{\frac{\mathrm{d} \bm{x}^*}{\dt}}_{= 0} 
    = \dot{\bm{x}}(t) 
    \Rightarrow 
    \bar{\bm{x}}(t) = A \bar{\bm{x}}(t) + B \bar{\bm{u}}(t)
\end{align*}

Somit ist die Rückführung von $\bar{\bm{x}}$ auf $\bm{0}$ 
äquivalten zu der Rückführung von $\bm{x}$ auf den Arbeitspunkt $\bm{x}^*$.
Wir definieren 
$h(\bm{q} , \dot{\bm{q}}, \bm{u}):= \bm{u} - C(\bm{q},\dot{\bm{q}}) \dot{\bm{q}} - G(\bm{q})$.
und wenden die Produktregel auf den Term $M(\bm{q}) \ddot{\bm{q}}$ an, um die Ableitung von $f$ nach $\bm{x}$ zu bestimmen.
% D(M q'') = DM q'' + M D(q'') = Dh
\begin{align*}
    \frac{\partial}{\partial \bm{x}} 
    \left[ 
        \vphantom{\int}
        M(\bm{q}) \ddot{\bm{q}}
    \right] 
    &= 
    \frac{\partial}{\partial \bm{x}} 
    \left[
        \vphantom{\int}
        h(\bm{q}, \dot{\bm{q}}, \bm{u})
    \right]
    \Rightarrow
    \frac{\partial M}{\partial \bm{x}} \ddot{\bm{q}} 
    + M(\bm{q}) \frac{\partial \ddot{\bm{q}}}{\partial \bm{x}} 
    = 
    \frac{\partial h}{\partial \bm{x}}
\end{align*}

Werten wir $h$ im Arbeitspunkt $(\bm{x}^*, \bm{u}^*)$ aus, 
so gilt $\ddot{\bm{q}} = \dot{\bm{q}} = \bm{0}$, 
somit fällt der erste Term weg und wir erhalten
% Am Gleichgewichtspunkt q'' = 0
\begin{align*}
    \frac{\partial h}{\partial \bm{x}}(\bm{x}^*, \bm{u}^*)
    &=
    \underbrace{
        \frac{\partial M}{\partial \bm{x}} \ddot{\bm{q}} 
        }_{=0}
    + M(\bm{x}^*) \frac{\partial \ddot{\bm{q}}}{\partial \bm{x}}  
    \Rightarrow
    \frac{\partial \ddot{\bm{q}}}{\partial \bm{x}} 
    = 
    M^{-1}(\bm{q}^*) \frac{\partial h}{\partial \bm{x}}(\bm{x}^*, \bm{u}^*)
\end{align*}

Für die Ableitung der Funktion $h = \bm{u} - C(\bm{q}, \dot{\bm{q}}) \dot{\bm{q}} - G(\bm{q})$ in ($\bm{x}^*, \bm{u}^*$) gilt
% Ableitung von h nach x    
\begin{align*}
    \frac{\partial h}{\partial \bm{x}}(\bm{x}^*, \bm{u}^*)
    =
    \begin{bmatrix}
        \frac{\partial h}{\partial \bm{q}} & \frac{\partial h}{\partial \dot{\bm{q}}}
    \end{bmatrix}
    &=
    \begin{bmatrix}
        -\frac{\partial C}{\partial \bm{q}} \dot{\bm{q}} - \frac{\partial G}{\partial \bm{q}} & -C(\bm{q}, \dot{\bm{q}}) - \frac{\partial C}{\partial \dot{\bm{q}}} \dot{\bm{q}} \\
    \end{bmatrix}
    (\bm{q}^*, \dot{\bm{q}}^* = \bm{0}) \\
    &=
    \begin{bmatrix}
        -\frac{\partial G}{\partial \bm{q}} & -C(\bm{q}^*, \bm{0})
    \end{bmatrix}
\end{align*}

Fassen wir die Ergebnisse zusammen, so gilt für die Ableitung von $f$ nach $\bm{x}$ im Arbeitspunkt $(\bm{x}^*, \bm{u}^*)$
% Ableitung von f nach x    
\begin{align*}
    A(\bm{x}^*, \bm{u}^*)
    &=
    \begin{bmatrix}
        \frac{\partial \dot{\bm{q}}}{\partial \bm{q}} 
        & \frac{\partial \dot{\bm{q}}}{\partial \dot{\bm{q}}} \\[6pt]
        \frac{\partial \ddot{\bm{q}}}{\partial \bm{q}} 
        & \frac{\partial \ddot{\bm{q}}}{\partial \dot{\bm{q}}}
    \end{bmatrix}
    (\bm{x}^*, \bm{u}^*) 
    =
    \begin{bmatrix}
        0 & I \\
        M^{-1}(\bm{q}^*) 
        \left[
            -\frac{\partial G}{\partial \bm{q}}(\bm{q}^*) 
        \right]
        & 
        M^{-1}(\bm{q}^*) 
        \left[
            -C(\bm{q}^*,\bm{0})
        \right]
    \end{bmatrix}   
\end{align*}

Für die Ableitung von $f$ nach $\bm{u}$ gilt
% Ableitung von f nach u
\begin{align*}
    B(\bm{x}^*, \bm{u}^*)
    &=
    \begin{bmatrix}
        \frac{\partial \dot{\bm{q}}}{\partial \bm{u}} \\[6pt]
        \frac{\partial \ddot{\bm{q}}}{\partial \bm{u}}
    \end{bmatrix}
    (\bm{x}^*, \bm{u}^*) 
    =
    \begin{bmatrix}
        0 \\[6pt]
        M^{-1}(\bm{q}^*)
    \end{bmatrix}
\end{align*}

% Satz mit den Blockmatrizen
%----------------------------------------------------------------------------------------------
\begin{satz}
    Gegeben sei die Dynamikgleichung
    %
    \begin{align*}
        M(\bm{q}) \ddot{\bm{q}} 
        + C(\bm{q}, \dot{\bm{q}}) \dot{\bm{q}} 
        + G(\bm{q}) = \bm{u}
        \Rightarrow
        \dot{\bm{x}} = f(\bm{x}, \bm{u})
    \end{align*}
    und ein Arbeitspunkt $(\bm{x}^*, \bm{u}^*)$ mit $f(\bm{x}^*, \bm{u}^*) = \bm{0}$. 
    Dann gilt für die Linearisierung:
    %
    \begin{align*}
        \dot{\bm{x}}
        &=
        A(\bm{x}^*, \bm{u}^*) \bm{x} + B(\bm{x}^*, \bm{u}^*) \bm{u} \\
        &=
        \begin{bmatrix}
            0 & I \\
            M^{-1}(\bm{q}^*) 
            \left[
                -\frac{\partial G}{\partial \bm{q}}(\bm{q}^*) 
            \right]
            & 
            M^{-1}(\bm{q}^*) 
            \left[
                -C(\bm{q}^*,\bm{0})
            \right]
        \end{bmatrix} 
        \bm{x} +
        \begin{bmatrix}
            0 \\[6pt]
            M^{-1}(\bm{q}^*)
        \end{bmatrix}
        \bm{u}
    \end{align*}
\end{satz}
% (M^-1 h)_1 = (M^-1)_{11} h_1 + (M^{-1})_{12} h_2
% (M^-1 h)_2 = (M^{-1})_{21} h_1 + (M^{-1})_{22} h_2

\subsection{Linearisierung des two link revolute manipulators}

Wir definieren  $x_1 = q_1$, $x_2 = q_2$, $x_3 = \dot{q_1}$, $x_4 = \dot{q_2}$  
und erhalten $\dot{\bm{x}} = f(\bm{x},\bm{u})$ mit
% Hier f genau definieren
\begin{align*}
    \begin{bmatrix}
        f_1(\bm{x},\bm{u})  \\
        f_2(\bm{x},\bm{u}) 
    \end{bmatrix} 
    =
    \begin{bmatrix}
        \dot{q}_1 \\
        \dot{q}_2
    \end{bmatrix}
    \quad
    \begin{bmatrix}
        f_3(\bm{x},\bm{u}) \\
        f_4(\bm{x},\bm{u}) 
    \end{bmatrix}
    =
    M(\bm{q})^{-1}
    \left[ 
        \bm{u} - C(\bm{x}) 
        \dot{\bm{q}}
        - G(\bm{q}) 
    \right]
\end{align*}

Für die Corioliskraft im Punkt $(\bm{x}^*,\bm{u}^*)$, 
bzw. für die Matrix $C(\bm{q}, \dot{\bm{q}}= \bm{0})$ gilt:
%
\begin{align*}
    C(\bm{q}^*, \bm{0}) 
    =
    \begin{bmatrix}
        0 & -m_2 l_1 a_2 \sin(q_2^*) \cdot 0 \\
        m_2 l_1 a_2 \sin(q_2^*) \cdot 0 & 0
    \end{bmatrix}
    =
    \begin{bmatrix}
        0 & 0 \\
        0 & 0
    \end{bmatrix}
\end{align*}

Wir berechnen zunächst die Ableitung von $h$ bzgl. $\bm{x}$ im Punkt $(\bm{x}^*, \bm{u}^*)$ 
und erhalten
% Ableitung von h nach x
% h_x = [h_q \\ h_q']
\begin{align*}
    \frac{\partial h}{\partial \bm{x}}(\bm{x}^*, \bm{u}^*)
    &=
    \begin{bmatrix}
        - \frac{\partial G}{\partial \bm{q}} & - \underbrace{C(\bm{q}^*, \bm{0})}_{= \bm{0}}
    \end{bmatrix}
    =
    \begin{bmatrix}
        - \frac{\partial G_1}{\partial q_1}(\bm{q}^*) 
        & - \frac{\partial G_1}{\partial q_2}(\bm{q}^*) 
        & 0 & 0 \\
        - \frac{\partial G_2}{\partial q_1}(\bm{q}^*) 
        & - \frac{\partial G_2}{\partial q_2}(\bm{q}^*) 
        & 0 & 0
    \end{bmatrix}
\end{align*}

Für die konkrete Matrix $A = f_{\bm{x}}(\bm{x}^*, \bm{u}^*)$ gilt
% Matrix A
\begin{align*}
    A(\bm{x}^*, \bm{u}^*)
    &=
    \begin{bmatrix}
        \frac{\partial f_1}{\partial x_1}  
        & \frac{\partial f_1}{\partial x_2}  
        & \frac{\partial f_1}{\partial x_3}  
        & \frac{\partial f_1}{\partial x_4} 
        \\ \frac{\partial f_2}{\partial x_1} 
        & \frac{\partial f_2}{\partial x_2} 
        & \frac{\partial f_2}{\partial x_3} 
        & \frac{\partial f_2}{\partial x_4} 
        \\ \frac{\partial f_3}{\partial x_1} 
        & \frac{\partial f_3}{\partial x_2} 
        & \frac{\partial f_3}{\partial x_3} 
        & \frac{\partial f_3}{\partial x_4} 
        \\ \frac{\partial f_4}{\partial x_1} 
        & \frac{\partial f_4}{\partial x_2} 
        & \frac{\partial f_4}{\partial x_3} 
        & \frac{\partial f_4}{\partial x_4}
    \end{bmatrix}(\bm{x}^*, \bm{u}^*) 
    =  
    \begin{bmatrix}
        \begin{matrix}
            0 & 0 \\
            0 & 0
        \end{matrix}
        &
        \begin{matrix}
            1 & 0 \\
            0 & 1
        \end{matrix}
        \\[12pt]
        - M^{-1}(\bm{q}^*) \frac{\partial G}{\partial \bm{q}}(\bm{q}^*)
        &
        \begin{matrix}
            0 & 0 \\
            0 & 0
        \end{matrix}
    \end{bmatrix}
\end{align*}

Für die konkrete Matrix $B = f_{\bm{u}}(\bm{x}^*, \bm{u}^*)$ gilt
% Matrix B
\begin{align*}
    B(\bm{x}^*, \bm{u}^*)
    &=
    \begin{bmatrix}
        \frac{\partial f_1}{\partial u_1}  
        & \frac{\partial f_1}{\partial u_2}  
        \\ \frac{\partial f_2}{\partial u_1} 
        & \frac{\partial f_2}{\partial u_2} 
        \\ \frac{\partial f_3}{\partial u_1} 
        & \frac{\partial f_3}{\partial u_2} 
        \\ \frac{\partial f_4}{\partial u_1} 
        & \frac{\partial f_4}{\partial u_2} 
    \end{bmatrix}(\bm{x}^*, \bm{u}^*) 
    =  
    \begin{bmatrix}
        \begin{matrix}
            0 & 0 \\
            0 & 0
        \end{matrix}
        \\[12pt]
        M^{-1}(\bm{q}^*)
    \end{bmatrix}   
\end{align*}

Für die konkret Berechnung der partiellen Ableitungen der Gravitationskraft gilt
% Ableitung der Gravitationskraft
\begin{align*}
    G(\bm{q}) &=
    \begin{bmatrix}
        m_1  g  a_1 \cos(\gruen{q_1}) + m_2  g \left( l_1 \cos(\gruen{q_1}) 
        + a_2 \cos(\gruen{q_1} + \dunkelrot{q_2}) \right) \\
        m_2  g \left( l_1 \cos(\gruen{q_1}) + a_2 \cos(\gruen{q_1} + \dunkelrot{q_2}) \right)
    \end{bmatrix}
    \\
    \frac{\partial G}{\partial \bm{q}}
    &=
    \begin{bmatrix}
        \frac{\partial G_1}{\partial q_1}
        & \frac{\partial G_1}{\partial q_2} \\
        \frac{\partial G_2}{\partial q_1}
        & \frac{\partial G_2}{\partial q_2}
    \end{bmatrix} \\
    &=
    \begin{bmatrix}
        - m_1 g a_1 \sin(\gruen{q_1}) 
        - m_2 g \left( l_1 \sin(\gruen{q_1}) + a_2 \sin(\gruen{q_1} + \dunkelrot{q_2}) \right)
        & - m_2 g a_2 \sin(\gruen{q_1} + \dunkelrot{q_2}) \\
        - m_2 g \left( l_1 \sin(\gruen{q_1}) + a_2 \sin(\gruen{q_1} + \dunkelrot{q_2}) \right)
        & - m_2 g a_2 \sin(\gruen{q_1} + \dunkelrot{q_2})
    \end{bmatrix}
\end{align*}

Die Massenmatrix $M(\bm{q})$ ist gegeben durch den Ausdruck:
% Massenmatrix M^-1
\begin{align*}
    M(\bm{q})
    =
    \begin{bmatrix}
        m_1 a_1^2 + I_{1,zz} + m_2 \cdot (l_1^2 +a_2^2 + 2 l_1 a_2  \cos(\dunkelrot{q_{2}}) ) + I_{2,zz}
        &  m_2 \cdot (a_2^2 +  l_1 a_2  \cos(\dunkelrot{q_{2}})) + I_{2,zz} \\
        m_2 \cdot (a_2^2 +  l_1 a_2  \cos(\dunkelrot{q_{2}})) + I_{2,zz}
        & m_2 a_2^2 + I_{2,zz}
    \end{bmatrix}
\end{align*}


Für die Inverse der Massenmatrix $M^{-1}$ gilt analytisch 
mittels $F_{\dunkelrot{q_2}} := m_2 l_1 a_2 \cos(\dunkelrot{q_{2}})$

\begin{align*}
    \det(M) &= M_{11} M_{22} - M_{12}^2 \\
            &= \left[ m_1 a_1^2 + I_{1,zz} + m_2 \cdot (l_1^2 +a_2^2 + 2 l_1 a_2  
            \cos(\dunkelrot{q_{2}}) ) + I_{2,zz} \right]
             \cdot (m_2 a_2^2 + I_{2,zz}) \\
            &\quad - (m_2 \cdot (a_2^2 +  l_1 a_2  \cos(\dunkelrot{q_{2}})) + I_{2,zz})^2 \\
            &= 
            \left[
            m_1 a_1^2 + I_{1,zz} + I_{2,zz} + m_2 l_1^2 +m_2 a_2^2 + 
            2 \underbrace{m_2 l_1 a_2  
            \cos(\dunkelrot{q_{2}})}_{F_{\dunkelrot{q_{2}}}} 
            \right] 
            (m_2 a_2^2 + I_{2,zz}) \\ 
            &\quad
            - (m_2 a_2^2 + I_{2,zz})^2
            - 2 \underbrace{m_2 l_1 a_2 \cos(\dunkelrot{q_{2}})}_{F_{\dunkelrot{q_{2}}}}
            \cdot (m_2 a_2^2 + I_{2,zz})
            - F_{\dunkelrot{q_{2}}}^2 \\
            &= 
            (m_2 a_2^2 + I_{2,zz}) \cdot 
            \left[
                m_1 a_1^2 + I_{1,zz} + m_2 l_1^2 +
                \underbrace{I_{2,zz}  + m_2 a_2^2 - m_2 a_2^2 - I_{2,zz}}_{= 0}
            \right]
            - \underbrace{
                (m_2 l_1 a_2 )^2 (1- \sin(\dunkelrot{q_{2}})^2)
                }_{F_{\dunkelrot{q_{2}}}^2}
            \\
            &= 
            (m_2 a_2^2 + I_{2,zz}) \cdot (m_1 a_1^2 + I_1 + m_2 l_1^2)
            - (m_2 l_1 a_2 )^2 + (m_2 l_1 a_2 )^2 \sin(\dunkelrot{q_{2}})^2
            \\
            &= (m_2 a_2^2 + I_{2,zz}) \cdot (m_1 a_1^2 + I_1) 
            + \underbrace{
                (m_2 a_2^2 + I_{2,zz}) \cdot m_2 l_1^2
            }_{m_2^2 l_1^2 a_2^2+ m_2 l_1^2 I_{2,zz}}
            - (m_2 l_1 a_2 )^2 + (m_2 l_1 a_2 )^2 \sin(\dunkelrot{q_{2}})^2
            \\
            &= (m_1 a_1^2 + I_1 + m_2 l_1^2)
            \cdot (m_2 a_2^2 + I_2) 
            + m_2 l_1^2 I_{2,zz}
            + m_2^2 l_1^2 a_2^2  \sin(\dunkelrot{q_{2}})^2
\end{align*}

Für die Inverse der Massenmatrix gilt somit
\begin{align*}
    M^{-1} &= \frac{1}{\det(M)} 
    \begin{pmatrix}
        M_{22} & -M_{12} \\
        -M_{21} & M_{11}
    \end{pmatrix} \\
    &= \frac{1}{\det(M)} 
    \begin{pmatrix}
        m_2 a_2^2 + I_2 & - (m_2 \cdot (a_2^2 +  l_1 a_2  \cos(\dunkelrot{x_{2}})) + I_2) \\
        - (m_2 \cdot (a_2^2 +  l_1 a_2  \cos(\dunkelrot{x_{2}})) + I_2) & m_1 a_1^2 + I_1 + m_2 \cdot (l_1^2 +a_2^2 + 2 l_1 a_2  \cos(\dunkelrot{x_{2}}) ) + I_2 
    \end{pmatrix}
\end{align*}

\begin{align*}
    M_{11} &= m_1 a_1^2 + I_1 + m_2 \cdot (l_1^2 +a_2^2 + 2 l_1 a_2  \cos(\dunkelrot{x_{2}}) ) + I_2 \\
    M_{12} &= m_2 \cdot (a_2^2 +  l_1 a_2  \cos(\dunkelrot{x_{2}})) + I_2 \\
    M_{21} &= M_{12} \\
    M_{22} &= m_2 a_2^2 + I_2
\end{align*}
%----------------------------------------------------------------------------------------------



\usetikzlibrary{positioning}
\begin{tikzpicture}[>=latex]
    \node[draw= boxframe,anchor = north west ,align=left] (a) at (-2,0) {
        Ursprungsproblem:\\
        $\begin{aligned}
            \min_u & \int_0^\infty 
            (\bm{q} - \bm{q}_f)^T Q (\bm{q} - \bm{q}_f) 
            + (\bm{u} - \bm{u}_f)^T R (\bm{u} - \bm{u}_f) \dt \\
            	\textsf{s.t. } & M(\bm{q}) \ddot{\bm{q}} 
                + C(\bm{q},\dot{\bm{q}}) \dot{\bm{q}} 
                + G(\bm{q}) = \bm{u};
                \quad
                \bm{u}_f - G(\bm{q}_f) = \bm{0} 
        \end{aligned}$
    };
    \node[draw = rot, anchor = north west, align=left] (b) at ([xshift=1cm]a.north east) {
        Berechnung von $M^{-1}(\bm{q}_f)$ und $\frac{\partial G}{\partial \bm{q}}(\bm{q}_f)$: \\
        %
        $\begin{aligned}
            A &= 
            \begin{bmatrix}
                \bm{0}_2 & I_2 \\
                M^{-1} \frac{\partial G}{\partial \bm{q}}
                & 
                \bm{0}_2
            \end{bmatrix}, \quad
            B &=
            \begin{bmatrix}
                \bm{0} \\
                M^{-1}
            \end{bmatrix}
        \end{aligned}$
    };
    \node[draw = rot, below=of b, align=center] (c) {
        LQR mit $\bar{\bm{x}} = \bm{x} - \bm{x}_f, \bar{\bm{u}} = \bm{u} - \bm{u}_f$ \\

        $\begin{aligned}
            \min_{\bar{\bm{u}}} & \int_0^\infty 
            \bar{\bm{x}}^T Q \bar{\bm{x}} + \bar{\bm{u}}^T R \bar{\bm{u}} \dt 
            \quad\dot{\bar{\bm{x}}} = A \bar{\bm{x}} + B \bar{\bm{u}};
        \end{aligned}$ \\
        Lösung der Riccati-Gleichung für $K \in \mathbb{R}^{2 \times 2}$ \\
        $\begin{aligned}
            Q - K B R^{-1} B^T K + KA + A^T K &= 0 
        \end{aligned}$
        };
    \node[draw = rot, below=of c, align=center] (d) {Closed-Loop Regelung:\\
        $\begin{aligned}
            \bm{u} 
            = \bm{u}_f + \bar{\bm{u}} 
            = G(\bm{q}_f) -R^{-1} B^T K (\bm{x} - \bm{x}_f)
        \end{aligned}$
        };
    \node[draw = boxframe, anchor=south, align=left] (e) at ([yshift=-6cm]a.south) {
        Lösung Originalsystem \\
        $\begin{aligned}
            \bm{x}_0 &= \bm{x}(0) \\
            \bm{u}_i &= G(\bm{q}_f) - R^{-1} B^T K (\bm{x}_i - \bm{x}_f) \\
            \bm{x}_{i+1} &= \bm{x}_i + h \cdot f(\bm{x}_i, \bm{u}_i)  
        \end{aligned}$
        };

    \draw[->, color = boxframe] (a) -- (b);
    \draw[->, color = boxframe] (b) -- (c);
    \draw[->, color = boxframe] (c) -- (d);
    \draw[->, color = boxframe] (d.west) -- ++(-2.45cm,0cm);
    \draw[->, color = gruen] (a) -- (e);
\end{tikzpicture}

\subsection{Riccati Regler}

Es gilt 
%
\begin{align*}
F(K) := Q - KPK + KA + A^T K \Rightarrow F(K) = 0
\end{align*}

% Wichtige Grundbegriffe aus LA
%----------------------------------------------------------------------------
\begin{definition}
Sei $X \in \mathbb{R}^{n \times n}$ dann definieren wir 
%
\begin{align*}
\textsf{vec}(X) = 
\begin{pmatrix}
X_{11} & X_{21} & \cdots & X_{n1} & X_{12} & X_{22} & \cdots X_{nn}  
\end{pmatrix}^T 
\in 
\mathbb{R}^{n^2}
\end{align*}
%
Seien $A \in \mathbb{R}^{n \times m}, B \in \mathbb{R}^{p \times r}$, dann definieren wir 
%
\begin{align*}
A \otimes B =
\begin{pmatrix}
A_{11} B 	& \cdots 	& A_{1m} B 	\\
A_{12} B 	& \hdots 	& A_{2m} B	\\
\vdots		& \ddots	& \vdots	\\
A_{n1} B	& \cdots	& A_{nm} B
\end{pmatrix}
\in 
\mathbb{R}^{mp \times nr}
\end{align*}
\end{definition}
%----------------------------------------------------------------------------

Es gilt die Formel
%
\begin{align*}
\textsf{vec}(MXN) = (N^T \otimes M) \textsf{vec}(X)  
\end{align*}

Betrachten wir den Fall $M= 1$ und $N=A$ bzw. $N = 1$ und $M=A^T$ erhalten wir mit $X=K$ 
%
\begin{align*}
\textsf{vec}(KA)= (A^T \otimes 1) \textsf{vec}(K) \quad \textsf{vec}(A^T K) = (1 \otimes A^T) \textsf{vec}(K)
\end{align*}

Anwendung auf unser Problem liefert
%
\begin{align*}
\underbrace{\textsf{vec}(F(K))}_{f(x = \textsf{vec}(K))} = \underbrace{\textsf{vec}(Q)}_{q} - \underbrace{\textsf{vec}(KPK)}_{g(x = \textsf{vec}(K))} + \underbrace{\textsf{vec}(KA)}_{(A^T \otimes 1) \textsf{vec}(K)} + \underbrace{\textsf{vec}(A^T K)}_{(1 \otimes A^T) \textsf{vec}(K)}
\end{align*}

Durch Einführung von $x = \textsf{vec}(K) \in \mathbb{R}^{n^2}$ erhalten wir 
%
\begin{align*}
f(x) = q - g(x)+ (A^T \otimes 1)x + (1 \otimes A^T)x
\end{align*}

Da $(KP)^T = KP$ gilt 
% 
\begin{align*}
g(\textsf{vec}(K)) = \textsf{vec}(KPK) = (1 \otimes KP) \textsf{vec}(K) 
\end{align*}

% Hier arbeit 14.07
%-----------------------------------------------------------------------------------

Wir betrachten die Variation $K(\epsilon) = K + \epsilon \cdot \bar{K}$ und die Funktion $h:\mathbb{R} \to \mathbb{R}^{n^2}$ mit
%
\begin{align*}
h(\epsilon) 
:= g(\textsf{vec}(K(\epsilon))) 
&= \textsf{vec}(K(\epsilon) P K(\epsilon)) \\
&= \textsf{vec}((K + \epsilon \cdot \bar{K}) P (K + \epsilon \cdot \bar{K})) \\
&= \textsf{vec}(KPK + \epsilon KP \bar{K} + \epsilon \bar{K} PK + \epsilon^2 \bar{K} P \bar{K})
\end{align*}

Ableitung nach $\epsilon$ und Auswertung in $\epsilon = 0$ ergibt
%
\begin{align*}
h'(0) = \frac{d}{d\epsilon} g(\textsf{vec}(K + \epsilon \bar{K})) \bigg|_{\epsilon = 0} = \textsf{vec}(KP\bar{K} + \bar{K}PK)
= \textsf{vec}(KP\bar{K}) + \textsf{vec}(\bar{K}PK)
\end{align*}

Umschreiben als Matrix-Vektor-Produkt
%
\begin{align*}
\textsf{vec}(KP \cdot \bar{K} \cdot 1) = (1 \otimes KP) \textsf{vec}(\bar{K}) \quad \textsf{vec}(1 \cdot \bar{K} \cdot PK) = ((PK)^T \otimes 1) \textsf{vec}(\bar{K})
\end{align*}

Wir erhalten als Ableitung 
%
\begin{align*}
g'(x) = (1 \otimes KP) + ((PK)^T \otimes 1) \Rightarrow f'(x) = - (1 \otimes KP) - ((PK)^T \otimes 1) + (A^T \otimes 1) + (1 \otimes A^T)
\end{align*}

\subsubsection{Implementierung}

\begin{minipage}{0.65\textwidth}
\begin{lstlisting}
def vec(X):
    return X.reshape(-1, order='F')

def unvec(v, n):
    return v.reshape((n, n), order='F')
\end{lstlisting}
\end{minipage}
%
\begin{minipage}{0.35\textwidth}
Hilfsfunktionen für die Vektorisierung
\end{minipage}

\begin{minipage}{0.65\textwidth}
\begin{lstlisting}
# Newton Kleinman Verfahren 
def solv_CARE(A,B,R,Q,tol=1e-8,max_iter=50):
	n = A.shape[0]
	I = np.eye(n)
	q = vec(Q)
	L = np.kron(I, A.T) + np.kron(A.T, I)
	P = B @ np.linalg.solve(R, B.T)
	# Startwert K0 hier Einheitsmatrix
	K = I
	x = vec(K)	
	
	for i in range(max_iter):
		X = K @ P #KP
		vec_KPK = np.kron(I,X) @ x
		f = q - vec_KPK + L @ x 
		Dg = np.kron(I,X)+np.kron(X,I) 
		Df = L - Dg
		dx = np.linalg.solve(Df, -f)
		x_new = x + dx
		# Abbruch
		if np.linalg.norm(dx) / np.linalg.norm(x_new) < tol:
			x=x_new
			break
        
        x = x_new
        K = unvec(x, n)
	return K
\end{lstlisting}
\end{minipage}
%
\begin{minipage}{0.35\textwidth}
\rot{Erklärung des Codes}

\blau{Stabilität der Startlösung hier Gershgorin kreise und Stabilität erklären}

Für $Y$ mit $RY=B^T \Rightarrow Y = B^{-1}B^T$ und $B \cdot Y = P$
\end{minipage}
%

\newpage


\section{Gradientenverfahren}

\begin{satz}[Gradientenverfahren mit Linesearch ]
Sei $f:\mathbb{R}^{n} \to \mathbb{R}$ differenzierbar, 
%
%
\begin{align*}
    \bm{x}_{\rot{k+1}} &= \bm{x}_{\rot{k}} - \alpha_{\gruen{j}} \cdot \nabla f(\bm{x}_{\rot{k}}) \quad
    \min \left\{ \gruen{j}:  
    \underbrace{
    f(\bm{x}_{\rot{i}} - \alpha_{\gruen{j}} 
    \nabla f(\bm{x}_{\rot{i}}))}
    _{
        \varphi(\alpha_{\gruen{j}})
        }  \leq \underbrace{f(\bm{x}_{\rot{i}})}_{\varphi(0)} \underbrace{-  \blau{c} \cdot \alpha_{\gruen{j}}  \norm{ \nabla f(\bm{x}_{\rot{i}})}^2}_{\blau{c} \cdot \varphi'(0) \cdot \alpha_{\gruen{j}},  } \right\}  
\end{align*}

konvergiert für $\rot{k \to \infty}$ gegen die Lösung von $\min_x f(x)$.
\end{satz}

Ziel Berechnung des Gradienten $J'(\bm{u})$ für das Steuerungsproblem

\begin{align*}
    J^{aux}(\bm{u}) 	
            &= \int_{t_0}^{t_f} 
            L(t,\bm{x},\bm{u}) 
            + \bm{p}(t) \cdot 
            \left[
                \vphantom{\int}
                f(t,\bm{x},\bm{u}) - \dot{\bm{x}}(t)
            \right] \dt 
            + \Phi(\bm{x}(t_f)) \\
			&= \int_{t_0}^{t_f} 
            L(t,\bm{x},\bm{u}) + \bm{p}(t) \cdot f(t,\bm{x},\bm{u}) \dt 
            -  \int_{t_0}^{t_f} \bm{p}(t) \dot{\bm{x}}(t) \dt 
            + \Phi(\bm{x}(t_f)) \\
            &= \int_{t_0}^{t_f} H[t] \dt 
            - \left[
                \vphantom{\int}
                \bm{p}(t) \bm{x}(t) 
                \right]_{t_0}^{t_f} 
            + \int_{t_0}^{t_f} \dot{\bm{p}}(t) \bm{x}(t) \dt + \Phi(\bm{x}(t_f)) \\
			&= \int_{t_0}^{t_f} H[t] + \dot{\bm{p}}(t) \bm{x}(t) \dt 
            - \left[
                \vphantom{\int}
                \bm{p}(t) \bm{x}(t) 
                \right]_{t_0}^{t_f} + \Phi(\bm{x}(t_f)) \\
			&= \int_{t_0}^{t_f} H[t] + \dot{\bm{p}}(t) \bm{x}(t) \dt 
            - \bm{p}(t_f) \bm{x}(t_f) +  \gelb{\bm{p}(t_0) \bm{x}(t_0)} + \Phi(\bm{x}(t_f))
\end{align*}

Der Term $\gelb{\bm{p}(t_0) \bm{x}(t_0)}$ ist unabhängig von $\bm{u}$ und fällt bei der Optimierung weg. 
Bilden der Gateau Ableitung in $\bm{u}$ entlang $\bm{h}$ liefert  

\begin{align*}
DJ^{aux}(\bm{u},\bm{h}) 
        &= \int_{t_0}^{t_f} H_x [t] \cdot S(t)  
        + H_u [t] \cdot h(t)  
        + \dot{\bm{p}}(t) S(t)  \dt - \bm{p}(t_f) S(t_f)  
        + \Phi_x (\bm{x}(t_f)) S(1) \\
		&= \int_{t_0}^{t_f} ( 
                \underbrace{
                    H_x [t] + \dot{\bm{p}}(t)
                    }_{ = 0}
                    ) 
                \cdot S(t)  + H_u [t] \cdot h(t)  \dt  
                + ( \underbrace{\Phi_x (\bm{x}(t_f)) - \bm{p}(t_f)}_{= 0} ) \cdot S(t_f)  \\
		&= \int_0^T H_u [t] \cdot h(t) \dt
\end{align*}

Hierbei bezeichnet $S(t)$ die Sensitivity function von $\bm{x}$ in $\bm{u}$ in Richtung $\bm{h}$, also
\begin{align*}
S(t) = \frac{\partial \bm{x}(t)}{\partial \bm{u}} \cdot \bm{h}
\end{align*}
Wir erhalten somit $\nabla J(\bm{u}) = H_u [t]$

\newpage

\subsection{Gradientenverfahren für den two link revolute Manipulator}

Wir betrachten das Steuerungsproblem 
%
\begin{align*}
    \min_{\bm{u}} &\int_{t_0}^{t_f} L(\bm{x},\bm{u}) \dt + \Phi(\bm{x}(t_f)) \\
    L(\bm{x},\bm{u}) 
    &= 
    (\bm{u} - \bm{u}_f)^T \cdot R \cdot (\bm{u} - \bm{u}_f) \\ 
    \Phi(\bm{x}(t_f)) &= (\bm{q}(t_f) - \bm{q}_f)^T \cdot Q \cdot (\bm{q}(t_f) - \bm{q}_f) \\
    \dot{\bm{x}} &= f(\bm{x},\bm{u}) =  
    \begin{bmatrix}
        \dot{\bm{q}} \\
        M^{-1}(\bm{q}) \cdot ( - C(\bm{q},\dot{\bm{q}}) \cdot \dot{\bm{q}} - G(\bm{q}) + \bm{u} )
    \end{bmatrix}
    \\
    \bm{x} 
    &=
    \begin{bmatrix}
        \bm{q} \\
        \dot{\bm{q}}
    \end{bmatrix}
    \quad
    \bm{x}(t_0) =
    \begin{bmatrix}
        \bm{q}(t_0) \\
        \dot{\bm{q}}(t_0)
    \end{bmatrix}
\end{align*}

Für die adjungierte Gleichung gilt
%
\begin{align*}
    \dot{\bm{p}} = - H_{\bm{x}} 
    =
    -
    \begin{bmatrix}
        \frac{\partial H}{\partial \bm{q}} \\[6pt]
        \frac{\partial H}{\partial \dot{\bm{q}}}
    \end{bmatrix},
    \quad
    \bm{p}(t_f) = \Phi_{\bm{x}}(\bm{x}(t_f))
    = \begin{bmatrix}
        2 Q (\bm{q}(t_f) - \bm{q}_f) \\
        \bm{0}
    \end{bmatrix}
    \quad
    H[t] = L(\bm{u}) + \bm{p}^T \cdot f(\bm{x},\bm{u})
\end{align*}

Für den Gradienten gilt
%
\begin{align*}
\nabla J(\bm{u}) 
&= 
\frac{\partial H}{\partial \bm{u}}
=
2 R \cdot (\bm{u} - \bm{u}_f) +
\frac{\partial f}{\partial \bm{u}}^T \cdot \bm{p} 
\end{align*} 

Wir nutzen die spezielle Struktur unseres Problems
%
\begin{align*}
    \frac{\partial f}{\partial \bm{u}}
    =
    \begin{bmatrix}
        \bm{0}_2 \\
        M^{-1}(\bm{q})
    \end{bmatrix}
    \Rightarrow
    \frac{\partial f}{\partial \bm{u}}^T \cdot
    \begin{bmatrix}
        \bm{p}_{\bm{q}} \\
        \bm{p}_{\dot{\bm{q}}}
    \end{bmatrix}
    =
    M^{-1}(\bm{q}) \cdot \bm{p}_{\dot{\bm{q}}}
    \Rightarrow
    \nabla J(\bm{u})
    =
    2 R \cdot (\bm{u} - \bm{u}_f) + M^{-1}(\bm{q}) \cdot \bm{p}_{\dot{\bm{q}}}
\end{align*}
 
% Hier konkrerte Schritte des Algorithmus
\begin{itemize}
    \item[1)] Foward Integration der State Equation: $\dot{\bm{x}}=f(\bm{x},\bm{u})$
    %
    \begin{align*}
        \bm{x}_0 &= \bm{x}(t_0) \\
        \bm{x}_{i+1} 
        &=
        \bm{x}_i + h \cdot
        \begin{bmatrix}
            \dot{\bm{q}}_i \\
            M^{-1}(\bm{q}_i) \cdot ( - C(\bm{q}_i,\dot{\bm{q}}_i) \cdot \dot{\bm{q}}_i - G(\bm{q}_i) + \bm{u}_i )
        \end{bmatrix}
        \quad i = 0, \ldots, N-1
    \end{align*}
    %
    Abspeichern Array $\bm{X} = [\bm{x}_0, \bm{x}_1, \ldots, \bm{x}_N]$
    \item[2)] Backward Integration der Adjungierten Gleichung: 
    $\dot{\bm{p}} = - H_{\bm{x}} = - \frac{\partial f}{\partial \bm{x}}$
    %
    \begin{align*}
        \bm{p}_N 
        &= \bm{p}(t_f) 
        = 
        \begin{bmatrix}
            2 Q (\bm{q}(t_f) - \bm{q}_f) \\
            \bm{0}
        \end{bmatrix} \\
        \bm{p}_{i-1} 
        &= \bm{p}_{i} + h  \cdot
        \begin{bmatrix}
           \bm{0}_2 & I_2 \\
            \frac{\partial \ddot{\bm{q}}}{\partial \bm{q}}
           &
           \frac{\partial \ddot{\bm{q}}}{\partial \dot{\bm{q}}}
        \end{bmatrix}
        \cdot \bm{p}_{i} \\
        \frac{\partial \ddot{\bm{q}}}{\partial \bm{q}} 
        &=
        \frac{\partial}{\partial \bm{q}} 
        \left( 
            M^{-1}(\bm{q}_i) \cdot ( - C(\bm{q}_i,\dot{\bm{q}}_i) \cdot \dot{\bm{q}}_i 
            - G(\bm{q}_i) + \bm{u}_i ) 
            \right) 
            \\
        \frac{\partial \ddot{\bm{q}}}{\partial \bm{q}} &= 
        \frac{\partial}{\partial \dot{\bm{q}}} \left( 
            M^{-1}(\bm{q}_i) \cdot ( - C(\bm{q}_i,\dot{\bm{q}}_i) 
            \cdot \dot{\bm{q}}_i - G(\bm{q}_i) + \bm{u}_i ) 
            \right)
    \end{align*}
    %
    \begin{itemize}
        \item[2.1)] Finite Differenzen für $\frac{\partial \ddot{\bm{q}}}{\partial x_j}$ $j=1,2,3,4$
        %
        \begin{align*}
            \bm{x}_{plus}  &= \bm{x} + \epsilon \cdot \bm{e}_j
            \quad
            \bm{x}_{minus}  = \bm{x} - \epsilon \cdot \bm{e}_j
            \\
            \ddot{\bm{q}}(\bm{x}) 
            &= M^{-1}(\bm{q}) 
            \cdot ( - C(\bm{q},\dot{\bm{q}}) \cdot \dot{\bm{q}} - G(\bm{q}) + \bm{u} ) \\
            \frac{\partial \ddot{\bm{q}}}{\partial x_j}
            &\approx
            \frac{\ddot{\bm{q}}(\bm{x}_{plus}) - \ddot{\bm{q}}(\bm{x}_{minus})}{2 \epsilon}
        \end{align*}
    \end{itemize}
    Abspeichern Array $\bm{P} = [\bm{p}_0, \bm{p}_1, \ldots, \bm{p}_N]$
    \item[3)] Gradientenverfahren 
    $\bm{u}_{i+1} = \bm{u}_i - \alpha_j \cdot \nabla J(\bm{u}_i)$
    %   %
    \begin{align*}
        \bm{u}_{i+1} &= \bm{u}_i - \alpha_j \cdot 
        \left[
            2 R \cdot (\bm{u}_i - \bm{u}_f) + M^{-1}(\bm{q}_i) \cdot \bm{P}_{\dot{\bm{q}},i}
        \right]
        \quad i = 0, \ldots, N-1    
    \end{align*}
        \begin{itemize}
            \item[3.1)] Berechnung der Schrittweite $\alpha_j$ über Linesearch
            %
            \begin{align*}
                \alpha_{test} &= \beta^m \cdot \alpha_0 \\
                \bm{u}_{test} &= \bm{u}_k - \alpha_{test} \cdot \nabla J(\bm{u}_k) \\
                J_{test} &= J(\bm{u}_{test})  
                \\
                \text{IF } J_{test} &\leq J(\bm{u}_k) - c \cdot \alpha_{test} \cdot \|\nabla J(\bm{u}_k)\|^2: \\
                    \alpha_k &= \alpha_{test} \\
                    \Rightarrow &\text{Break}
            \end{align*}
        \end{itemize}
\end{itemize}

\newpage

\section{Foundations of Reinforcement Learning}

Wir folgen der Notation aus \cite{Plaat_RL}.
% Grundbegriffe R, MDP, Policy, Value-Funktion
\begin{definition}
    Wir definieren für den Markov decision process (MDP) mit 
    Zustandsraum $\mathcal{S}$, 
    Aktionsraum $\mathcal{A}$, 
    Übergangswahrscheinlichkeiten $P:\mathcal{S} \times \mathcal{A} \times \mathcal{S} \to [0,1]$, 
    Belohnungsfunktion $R:\mathcal{S} \times \mathcal{A} \times \mathcal{S} \to \mathbb{R}$ 
    und Diskontfaktor $\gamma \in [0,1)$ die optimale Value-Funktion als
    %
\end{definition}

% Definition der Value-Funktion
\begin{definition}[Value-Funktion]
    Wir definieren die Value-Funktion für eine Policy $\pi$ als
    \begin{align*}
        V^{\pi}(\bm{s}) 
        = \mathsf{E} 
        \left[
            \sum_{i=0}^{\infty} \gamma^i \cdot r_{t+i+1} \mid \bm{s}_t = \bm{s}
        \right]
    \end{align*}
\end{definition}

% Herleitung V(s) = E(r_t+1 + gamma V(s_t+1) | s_t = s)
\begin{align*}
    V^{\pi}(\bm{s}) 
    &= \mathsf{E} 
    \left[
        \sum_{i=0}^{\infty} \gamma^i \cdot r_{t+i+1} \mid \bm{s}_t = \bm{s}
    \right]
    \\
    &= 
    \mathsf{E} 
    \left[
        r_{t+1} \mid \bm{s}_t = \bm{s}
    \right]
    +
    \rot{\gamma} \cdot
    \mathsf{E} 
    \left[
        \sum_{i=0}^{\infty} \gamma^{i} \cdot r_{t+i+\rot{2}} \mid \bm{s}_t = \bm{s}
    \right]
    \\
    &=
    \mathsf{E} 
    \left[
        r_{t+1} \mid \bm{s}_t = \bm{s}
    \right]
    +
    \rot{\gamma} \cdot
    \mathsf{E} 
    \left[
        \underbrace{        
            \mathsf{E}
            \left[
                \sum_{i=0}^{\infty} \gamma^{i} \cdot r_{t+i+\rot{2}} \mid \bm{s}_{t+1}
            \right]
            }_{
                V^{\pi}(\bm{s}_{t+1})
            }
        \mid \bm{s}_t = \bm{s}
    \right]
    \\
    \Rightarrow
    V^{\pi}(\bm{s})
    &=
    \mathsf{E} 
    \left[
        r_{t+1} + \gamma \cdot V^{\pi}(\bm{s}_{t+1}) \mid \bm{s}_t = \bm{s}
    \right] 
\end{align*}

Denn es gilt hier für die $\sigma$-Algebren, 
dass $\sigma(\bm{s}_t ) \subseteq \sigma(\bm{s}_{t+1})$ gilt. 
Explizit ausgeschrieben in Erwartungswerten ergibt sich:

% Herleitung V(s) = E_a~pi [ E_s'~T_a [ R_a(s,s') + gamma V(s') ] ]
\begin{align*}
    V^{\pi}(\bm{s})
    &= \mathsf{E} 
        \left[ 
            r_{t+1} + \gamma V^{\pi}(\bm{s}_{t+1}) \mid \bm{s}_t = \bm{s} 
        \right] \\
    &= \mathsf{E}_{a_t \sim \pi(\cdot|\bm{s})} 
        \left[ 
            \vphantom{\int}
            \mathsf{E} 
            \left[ 
                r_{t+1} + \gamma V^{\pi}(\bm{s}_{t+1}) \mid \bm{s}_t = \bm{s}, a_t = a 
            \right] 
        \right] \\
    &= \mathsf{E}_{a_t \sim \pi(\cdot|\bm{s})} 
    \left[ 
        \vphantom{\int}
        \mathsf{E}_{\bm{s}_{t+1} \sim T_a(\bm{s})} 
        \left[ 
            R_{a_t}(\bm{s}_t, \bm{s}_{t+1}) + \gamma V^{\pi}(\bm{s}_{t+1}) 
        \right] 
    \right] \\
    &= \mathsf{E}_{a \sim \pi(\cdot|\bm{s})} 
        \left[ 
            \vphantom{\int}
            \mathsf{E}_{\bm{s}' \sim T_a(\bm{s})} 
            \left[ 
                R_{a}(\bm{s}, \bm{s}') + \gamma V^{\pi}(\bm{s}') 
            \right] 
        \right]
\end{align*}

% Definition der Q-Funktion
\begin{definition}[Q-Function]
    Wir definieren die Q-Function als
    \begin{align*}
        Q^{\pi}(\bm{s}, \gruen{a}) 
        =
        \mathsf{E}
        \left[
            \vphantom{\int}
            \sum_{i=0}^{\infty} \gamma^i \cdot r_{t+i+1} 
            \mid \bm{s}_t = \bm{s}, \gruen{a_t = a}
        \right]
        \Rightarrow
        V^{\pi}(\bm{s})
        =
        \mathsf{E}_{\gruen{a} \sim \pi(\cdot|\bm{s})}
        \left[
            Q^{\pi}(\bm{s}, \gruen{a})
        \right]
    \end{align*}    
\end{definition}

Herleitung der Bellmann Gleichung für die Q-Funktion:
% Q(s,a) = E_s'~T_a [ R_a(s,s') + gamma E_a'~pi [ Q(s',a') ] ]
% Q(s,a) = E_s'~T_a [ R_a(s,s') + gamma V(s') ]
\begin{align*}
    Q^{\pi}(\bm{s}, a) 
    &= 
    \mathsf{E}
    \left[
        \vphantom{\int}
        \sum_{i=0}^{\infty} \gamma^i \cdot r_{t+i+1} 
        \mid \bm{s}_t = \bm{s}, a_t = a
    \right] \\
    &=
    \mathsf{E}
    \left[  
        \mathsf{E}
        \left[
            r_{t+1} + \sum_{i=1}^{\infty} \gamma^i \cdot r_{t+i+1} 
            \mid \bm{s}_t = \bm{s}, a_t = a, \bm{s}_{t+1} = \bm{s}'
            \right]
            \mid \bm{s}_t = \bm{s}, a_t = a
        \right] \\
        &= 
        \mathsf{E}
        \left[
            r_{t+1} + \gamma \cdot 
            \underbrace{
                \mathsf{E}
                \left[
                    \sum_{i=0}^{\infty} \gamma^{i} 
                    \cdot r_{t+i+2} \mid \bm{s}_{t+1} = \bm{s}'
                \right]
            }_{
                V^{\pi}(\bm{s}_{t+1})
            } 
            \mid \bm{s}_t = \bm{s}, a_t = a
        \right] \\
        &= 
        \mathsf{E}_{\bm{s}' \sim T_a(\bm{s})}
        \left[
            \vphantom{\int}
            R_a(\bm{s}, \bm{s}') + \gamma V^{\pi}(\bm{s}') 
        \right]
        \\
        &= 
        \mathsf{E}_{\bm{s}' \sim T_a(\bm{s})}
        \left[
            \vphantom{\int}
            R_a(\bm{s}, \bm{s}') + \gamma 
            \mathsf{E}_{a' \sim \pi(\cdot|\bm{s}')}
            \left[
                \vphantom{\int}
                Q^{\pi}(\bm{s}', a')
            \right]
        \right]
\end{align*}

    Wobei wir im letzten Schritt die folgende Beziehung genutzt haben:
    % V(s) = E_a [Q(s,a)]
    \begin{align*}
        V^{\pi}(\bm{s})
        &=
        \mathsf{E}_{a \sim \pi(\cdot|\bm{s})}
        \left[
            Q^{\pi}(\bm{s}, a)
        \right]
    \end{align*}

\subsection{Bellman Optimality Equations}

% Bellmann Optimality Equations 
% V* = max_a E_s'~T_a [ R_a(s,s') + gamma V*(s') ]
\begin{satz}[Optimality Equation für $V$] \label{Bellman_Optimality_V}
    Die optimale Value-Funktion $V^*(\bm{s})$ erfüllt:
    \begin{align*}
        V^{*}(\bm{s})
        =
        \max_{a} \mathsf{E}_{\bm{s}' \sim T_a(\bm{s})}
        \left[
            \vphantom{\int}
            R_{a}(\bm{s}, \bm{s}') + \gamma V^{*}(\bm{s}')
        \right]
        \quad
        \pi^*(\bm{s}) 
        =
        \arg\max_{a} \mathsf{E}_{\bm{s}' \sim T_a(\bm{s})}
        \left[
            \vphantom{\int}
            R_{a}(\bm{s}, \bm{s}') + \gamma V^{*}(\bm{s}')
        \right]
    \end{align*}
\end{satz}

\proof{
    Wir setzen $V^* (\bm{s}) = \max_{\pi} V^{\pi}(\bm{s})$. 
    Für jede Policy $\pi$ gilt nun $V^*(\bm{s}') \geq V^{\pi}(\bm{s}')$ und somit:
    % Herleitung Bellman Optimality Equation
    % V*(s) = max_pi V^pi(s) = max_a E_s'~T_a [ R_a(s,s') + gamma V*(s') ]
    \begin{align*}
        V^{*}(\bm{s})
        &= \max_{\pi} 
        \mathsf{E}_{a \sim \pi(\cdot|\bm{s})} 
        \left[ 
            \vphantom{\int}
            \mathsf{E}_{\bm{s}' \sim T_a(\bm{s})} 
            \left[ 
                R_{a}(\bm{s}, \bm{s}') + \gamma \rot{V^{\pi}}(\bm{s}')
            \right] 
        \right] 
        \rot{\leq}\;
        \max_{\pi}
        \mathsf{E}_{a \sim \pi(\cdot|\bm{s})}
        \left[ 
            \vphantom{\int}
            \mathsf{E}_{\bm{s}' \sim T_a(\bm{s})} 
            \left[
                R_{a}(\bm{s}, \bm{s}') + \gamma V^{*}(\bm{s}')
            \right]
        \right]
    \end{align*}

    Das Maximum über alle Policies $\pi$ wird erreicht durch die Policy,
    die in jedem Zustand $\bm{s}$ die Aktion $a$ auswählt, die das Maximum 
    in der inneren Erwartung erreicht. Somit gilt:
    %
    \begin{align*}
        V^{*}(\bm{s})
        &\leq \max_{a} \mathsf{E}_{\bm{s}' \sim T_a(\bm{s})}
        \left[
            \vphantom{\int}
            R_{a}(\bm{s}, \bm{s}') + \gamma V^{*}(\bm{s}')
        \right]
    \end{align*}

    Wir konstruieren nun die greedy Policy $\pi^*$ mittels:
    % Definition pi* = argmax_a E_s'~T_a [ R_a(s,s') + gamma V*(s') ]
    %
    \begin{align*}
        \pi^*(\bm{s}) 
        =
        \arg\max_{a} \mathsf{E}_{\bm{s}' \sim T_a(\bm{s})}
        \left[
            \vphantom{\int}
            R_{a}(\bm{s}, \bm{s}') + \gamma V^{*}(\bm{s}')
        \right]
    \end{align*}

    Für diese Policy $\pi^*$ gilt nun:
    % Herleitung V^{pi*}(s) = max_a E_s'~T_a [ R_a(s,s') + gamma V*(s') ]
    %
    \begin{align*}
        V^{\pi^*}(\bm{s})
        &= 
        \max_{a} \mathsf{E}_{\bm{s}' \sim T_a(\bm{s})}
        \left[
            \vphantom{\int}
            R_{a}(\bm{s}, \bm{s}') + \gamma V^{*}(\bm{s}')
        \right]
        \geq V^{*}(\bm{s})
        \Rightarrow
        V^{\pi^*}(\bm{s}) = V^{*}(\bm{s})
    \end{align*}
}
% Bellmann Optimality Equations for Q-function
% Q*(s,a) = E_s'~T_a [ R_a(s,s') + gamma max_a' Q*(s',a') ]
\begin{satz}[Optimality Equation für Q-Funktion] \label{Bellman_Optimality_Q}
    Die optimale Q-Funktion $Q^*(\bm{s}, a)$ erfüllt:
    %
    \begin{align*}
        Q^{*}(\bm{s}, a) =
        \mathsf{E}_{\bm{s}' \sim T_a(\bm{s})}
        \left[
            \vphantom{\int}
            R_a(\bm{s}, \bm{s}') + \gamma 
            \max_{a'} Q^{*}(\bm{s}', a')
        \right]
        \quad
        \pi^{*}(\bm{s}) = \arg\max_{a} Q^{*}(\bm{s}, a)
    \end{align*}
\end{satz}

\proof{
    Für die optimale Q-Funktion gilt somit
    % Q* = max_pi Q(a,s)
    \begin{align*}
        Q^{*}(\bm{s}, a) 
        =
        \max_{\pi} Q^{\pi}(\bm{s}, a)
        \Rightarrow
        Q^{*}(\bm{s}, a) =
        \mathsf{E}_{\bm{s}' \sim T_a(\bm{s})}
        \left[
            \vphantom{\int}
            R_a(\bm{s}, \bm{s}') + \gamma 
            \max_{a'} Q^{*}(\bm{s}', a')
        \right]
    \end{align*}

    Für die optimale Policy gilt:
    \begin{align*}
        \pi^{*}(\bm{s}) = a^* = \arg\max_{a} Q^{*}(\bm{s}, a)
    \end{align*}
}

\subsection{Value Iteration}

Wir definieren die Abbildung $\mathcal{T}$ als
\begin{align*}
    \mathcal{T}[V](\bm{s}, a)
    &:=
    \max_{a}
    \mathsf{E}_{\bm{s}' \sim T_a(\bm{s})}
    \left[
        \vphantom{\int}
        R_a(\bm{s}, \bm{s}') + \gamma 
        V(\bm{s}', a')
    \right]
\end{align*}

Die Bellmann Optimalitätsgleichung \ref{Bellman_Optimality_V} sagt nun aus, 
dass $V^*$ der Fixpunkt der Abbildung $\mathcal{T}$ ist, 
also $\mathcal{T}[V^*] = V^*$ gilt. 
Die Fixpunktiteration $V_{k+1} = \mathcal{T}[V_k]$ konvergiert somit gegen $V^*$.
%

\proof{
    Wir zeigen zunächst, dass $\mathcal{T}$ eine Kontraktion ist:
    % T(V) - T(U) <= gamma ||V-U||_infty
    \begin{align*}
        \left| \mathcal{T}[V] - \mathcal{T}[U] \right|
        &=
        \left|
            \lila{\max_{a}}
            \mathsf{E}_{\bm{s}' \sim T_a(\bm{s})}
            \left[
                \vphantom{\int}
                R_a(\bm{s}, \bm{s}') + \gamma 
                V(\bm{s}')
            \right]
            -
            \lila{\max_{a}}
            \mathsf{E}_{\bm{s}' \sim T_a(\bm{s})}
            \left[
                \vphantom{\int}
                R_a(\bm{s}, \bm{s}') + \gamma 
                U(\bm{s}')
            \right]
        \right| \\
        &\leq
        \lila{\max_{a}}
        \left|
            \mathsf{E}_{\bm{s}' \sim T_a(\bm{s})}
            \left[
                \vphantom{\int}
                R_a(\bm{s}, \bm{s}') + \gamma 
                V(\bm{s}')
            \right]
            -
            \mathsf{E}_{\bm{s}' \sim T_a(\bm{s})}
            \left[
                \vphantom{\int}
                R_a(\bm{s}, \bm{s}') + \gamma 
                U(\bm{s}')
            \right]
        \right|
        \\
        &=
        \max_{a}
        \left|
            \mathsf{E}_{\bm{s}' \sim T_a(\bm{s})}
            \left[
                \vphantom{\int}
                \gamma
                (V(\bm{s}') - U(\bm{s}'))
            \right]
        \right|
        \\
        &\leq
        \max_{a}
        \mathsf{E}_{\bm{s}' \sim T_a(\bm{s})}
        \left[
            \vphantom{\int}
            \gamma
            \cdot
            \underbrace{
                \rot{\sup_{\bm{s}'} |} V(\bm{s}') - U(\bm{s}') \rot{|}
            }_{
                \|V - U\|_{\infty}
            }
        \right]
        \\
        &=
        \gamma \|V - U\|_{\infty}
    \end{align*}

    Nach dem Fixpunktsatz von Banach \cite{sutton_barto_2018}
    konvergiert die Iteration $V_{k+1} = \mathcal{T}(V_k)$ gegen $V^*$ für $k \to \infty$.
}

\begin{algorithm}
    \caption{Value Iteration}
    \begin{algorithmic}
        \State Initialize $V_0(s)$ arbitrarily for all $s \in \mathcal{S}$
        \State $k \gets 0$
        \Repeat
        \For{each $s \in \mathcal{S}$}
        \State $V_{k+1}(s) \gets \max_{a} 
        \underbrace{
            \mathsf{E}_{s' \sim T_a(s)} 
            \left[ 
                \vphantom{\int}
                R_a(s, s') + \gamma V_k(s') 
            \right]}_{Q_{k}(s,a)}
            =
            \max_a \sum_{s'} T_a(s,s') 
            \left[ 
                \vphantom{\int}
                R_a(s,s') + \gamma V_k(s') 
            \right]
        $
        \EndFor
        \State $k \gets k + 1$
        \Until{$|V_k - V_{k-1}| < \varepsilon$}
        \State \Return $V_k$
    \end{algorithmic}
\end{algorithm}

\newpage

\subsection{Q-Learning}

Wir betrachten die Optimalitätsgleichung für die Q-Funktion \ref{Bellman_Optimality_Q}:
%
\begin{align*}
    Q^{*}(\bm{s}, a) =
    \mathsf{E}_{\bm{s}' \sim T_a(\bm{s})}
    \left[
        \vphantom{\int}
        R_a(\bm{s}, \bm{s}') + \gamma 
        \max_{a'} Q^{*}(\bm{s}', a')
    \right]
\end{align*}

und definieren erneut den Operator $\mathcal{T}$ als
%
\begin{align*}
    \mathcal{T}[Q](\bm{s}, a)
    &:=
    \mathsf{E}_{\bm{s}' \sim T_a(\bm{s})}
    \left[
        \vphantom{\int}
        R_a(\bm{s}, \bm{s}') + \gamma 
        \max_{a'} Q(\bm{s}', a')
    \right]
\end{align*}

und erhalten erneut die Fixpunktgleichung $\mathcal{T}[Q^*] = Q^*$. Die Fixpunktiteration
$Q_{k+1} = \mathcal{T}[Q_k]$ konvergiert somit gegen $Q^*$, 
allerdings ist die Berechnung der Erwartungswerte nicht möglich, 
da wir das Modell $\bm{s}' = T_a(\bm{s})$ nicht kennen.
%

% Hier konkreter Q-Learning Algorithmus
%--------------------------------------------------------------
\begin{algorithm}
    \caption{Q-Learning}
    \begin{algorithmic}
        \State Given Environment, 
        learning rate $\alpha \in (0,1]$, 
        discount factor $\gamma \in [0,1)$,
        exploration rate $\epsilon \in (0,1)$
        \State Initialize $Q(s,a)$ arbitrarily for all $s \in \mathcal{S}, a \in \mathcal{A}$
        \State $Q(\text{terminal}, \cdot) = 0$ \Comment{\lila{Terminale Zustände haben Wert 0}}
        \For{each episode}
            \State Initialize state $s$
            \While{state $s$ is not terminal}
                \State Choose action $a = \epsilon$-greedy$(Q[s,\cdot],\epsilon)$ 
                \Comment{\gruen{Exploration vs. Exploitation}}
                \State $(s',r) = Environment(s,a)$
                \State $Q[s,a] \leftarrow (1-\alpha) \cdot Q[s,a] 
                + \alpha \cdot (r + \gamma \cdot \max_{a'} Q[s',a'])$
                \Comment{\rot{Bootstrapping und TD-Learning}}
                \State $s \leftarrow s'$
            \EndWhile
        \EndFor
    \end{algorithmic}
\end{algorithm}

\proof{
    Wir prüfen die Vorraussetzungen für die Konvergenz von 
    Q-Learning aus \cite{tsitsiklis1992asynchronous} und folgern die 
    Konvergenz gegen $Q^*$ fast sicher.
    %
    \begin{itemize}
        \item Der Beweis, dass $\mathcal{T}$ eine Kontraktion ist,
        verläuft analog zum Beweis für die Value-Funktion.
    %
        \begin{align*}
            \left| \mathcal{T}[Q] - \mathcal{T}[U] \right|
            &=
            \left|
                \mathsf{E}_{\bm{s}' \sim T_a(\bm{s})}
                \left[
                    \vphantom{\int}
                    R_a(\bm{s}, \bm{s}') + \gamma 
                    \max_{a'} Q(\bm{s}', a')
                \right]
                -
                \mathsf{E}_{\bm{s}' \sim T_a(\bm{s})}
                \left[
                    \vphantom{\int}
                    R_a(\bm{s}, \bm{s}') + \gamma 
                    \max_{a'} U(\bm{s}', a')
                \right]
            \right| \\
            &=
            \left|
                \mathsf{E}_{\bm{s}' \sim T_a(\bm{s})}
                \left[
                    \vphantom{\int}
                    \gamma
                    \left(
                        \max_{a'} Q(\bm{s}', a') - \max_{a'} U(\bm{s}', a')
                    \right)
                \right]
            \right| \\
            &\leq
            \gamma
            \mathsf{E}_{\bm{s}' \sim T_a(\bm{s})}
            \left[
                \vphantom{\int}
                \gamma
                \cdot
                \underbrace{
                    \rot{\max_{a'} |} Q(\bm{s}', a') - U(\bm{s}', a') \rot{|}
                }_{
                    \leq \|Q - U\|_{\infty}
                }
            \right]
            \\
            &\leq
            \gamma \|Q - U\|_{\infty}
        \end{align*}
        %
        \item Wir verwenden folgende Update-Regel, um die Q-Funktion zu approximieren:
    %
        \begin{align*}
            Q_{k+1}(\bm{s}, a) 
            &=  Q_k(\bm{s}, a)
            + \alpha \cdot 
            \left[
                \underbrace{
                    R_a(\bm{s},\bm{s}') + \gamma \cdot \max_{a'} Q_k(\bm{s}', a')
                    }_{\text{Sample of }T[Q_k]} 
                - Q_k(\bm{s}, a)
            \right]
            \\
            &=
            Q_k(\bm{s}, a)
            + \alpha \cdot
            \left[
                \rot{\mathcal{T}[Q_k](\bm{s}, a)} - Q_k(\bm{s}, a) +
                \underbrace{
                    R_a(\bm{s},\bm{s}') + \gamma \cdot \max_{a'} Q_k(\bm{s}', a')
                    \rot{- \mathcal{T}[Q_k](\bm{s}, a)}}_{
                        \eta_k
                    }
            \right]
            \\
            &=  Q_k(\bm{s}, a)
            + \alpha \cdot
            \left[
                \vphantom{\int}
                \mathcal{T}[Q_k](\bm{s}, a) - Q_k(\bm{s}, a)
                + \eta_k
            \right]
        \end{align*}
        %
        Für den Erwartungswert des Rauschterms $\eta_k$ gilt:
        %
        \begin{align*}
            \mathsf{E}_{\bm{s}' \sim T_a(\bm{s})}
            \left[
                \eta_k \mid Q_k
            \right]
            &=
            \mathsf{E}_{\bm{s}' \sim T_a(\bm{s})}
            \left[
                R_a(\bm{s},\bm{s}') + \gamma \cdot \max_{a'} Q_k(\bm{s}', a')
                - 
                \underbrace{\mathcal{T}[Q_k](\bm{s}, a)}_{
                    \text{deterministic}
                }
                \mid Q_k
            \right]
            = 0
        \end{align*}
        %
        \item Wir definieren die $\epsilon$-Greedy-Policy:
    %
        \begin{align*}
            \pi_{\epsilon}(a, \bm{s})
            &:=
            \begin{cases}
                1- \epsilon + \frac{\varepsilon}{|\mathcal{A}|} & \text{if } a = \arg\max_{a'} Q[\bm{s}, a'] \\
                \frac{\varepsilon}{|\mathcal{A}|} & \text{else}
            \end{cases}
            =
            (1-\epsilon) \cdot \mathds{1}_{a = \arg\max_{a'} Q[\bm{s}, a']} 
            + \frac{\epsilon}{|A|}
        \end{align*}
\end{itemize}


}

\newpage

\subsection{SARSA}

\begin{algorithm}
    \caption{SARSA}
    \begin{algorithmic}
        \State Given Environment, 
        learning rate $\alpha \in (0,1]$, 
        discount factor $\gamma \in [0,1)$,
        exploration rate $\epsilon \in (0,1)$
        \State $Q(Terminal) = 0$
        \For{each episode}
            \State Initialize state $s$
            \State Choose action $a = \epsilon$-greedy$(Q[s,\cdot],\epsilon)$
            \While{state $s$ is not terminal}
                \State $(s',r) = Environment(s,a)$
                \State $Q[s,a] = (1-\alpha) \cdot Q[s,a] + \alpha \cdot (r + \gamma \cdot Q[s',a'])$
                \State $s = s'$
                \State Choose action $a' = \epsilon$-greedy$(Q[s,\cdot],\epsilon)$
            \EndWhile
        \EndFor
    \end{algorithmic}
\end{algorithm}

\newpage

\section{Reinforcement Learning in Optimal Control}

Wir folgen den Notationen aus \cite{Plaat_RL}.
Für eine gegebene Trajektorie $\tau$ des MDPs gilt:
% Definition der Trajektorie tau
\begin{align*}
    \tau_t^n :=
    \{
        \bm{s}_t, a_t, r_{t}, \bm{s}_{t+1}, \cdots, \bm{s}_{t+n}, a_{t+n}, r_{t+n}, \bm{s}_{t+n+1}
    \} 
    =
    \left\{
        \vphantom{\int}
        \bm{s}_{t+i}, a_{t+i}, r_{t+i}, \bm{s}_{t+i+1} 
    \right\}_{i=0,\ldots,n}
\end{align*}

In einem unendlichen Zeithorizont notieren wir $\tau_t^\infty = \tau_t$. 
Für die Verteilung der Trajektorien gilt:
% Definition der Trajektorie Verteilung p(tau|pi)
\begin{align*}
    p(\tau_0 \mid \rot{\pi})
    = p(\bm{s}_0) \cdot 
    \prod_{t=0}^{\infty} \rot{\pi}(a_t|\bm{s}_t) 
    \cdot T_{a_t}(\bm{s}_t, \bm{s}_{t+1})
\end{align*}     

Für das Zielfunktional definieren wir:
%
% Definition Objective Function J = V^pi(s_0) 
\begin{definition}[Objective Function]
    Wir definieren die Objective Function als
    %
    \begin{align*} 
        J(\rot{\pi}) 
        :=
        V^{\rot{\pi}}(\bm{s}_0)
        =
        \mathsf{E}_{\tau_0 \sim p(\tau_0| \rot{\pi})}
        \left[
            \vphantom{\int}
            R(\tau_0)
        \right]
        \Rightarrow
        \pi^* (a | \bm{s}) = \arg\max_{\pi} V^{\rot{\pi}}(\bm{s}_0)
    \end{align*}
\end{definition}


\subsection{Policy-Based Algorithms}
% Nun Schritt von pi zu pi_theta

Wir betrachten nun nachfolgend eine Familie von parametrisierten Policies
$\pi_{\theta}(a|\bm{s})$ mit den Parametern $\theta \in \mathbb{R}^d$.

\begin{center}
    \begin{tabular}{l |l| l}
        & Unendlichdimensionales Problem & Parametrisiertes Problem \\
        \hline
        Policy 
        & $\pi: S \to p(A)$ 
        & $\theta \rightarrow \pi_{\theta}$ mit $\pi_{\theta}(a|\bm{s})$ 
        \\
        $J(\pi)$ 
        &
        $\displaystyle J(\pi) = \mathsf{E}_{\tau \sim p(\tau \mid \rot{\pi})}
        \left[
            \vphantom{\int}
            R(\tau)
        \right]$
        & 
        $\displaystyle J(\blau{\theta}) 
        = \mathsf{E}_{\tau \sim p(\tau \mid \blau{\theta})}
        \left[
            \vphantom{\int}
            R(\tau)
        \right]$ 
        \\ 
        $p(\tau_0)$ 
        & $ \displaystyle p(\tau_0 \mid \rot{\pi})
            = p(\bm{s}_0) \cdot 
            \prod_{t=0}^{T-1} \rot{\pi}(a_t|\bm{s}_t) 
            \cdot T_{a_t}(\bm{s}_t, \bm{s}_{t+1})$ 
        & $ \displaystyle p(\tau_0 \mid \blau{\theta}) = 
            p(\bm{s}_0) \cdot 
            \prod_{t=0}^{T-1} \pi_{\blau{\theta}}(a_t|\bm{s}_t) 
            \cdot T_{a_t}(\bm{s}_t, \bm{s}_{t+1})$ 
        \\
        $\nabla J$ 
        & $\displaystyle \nabla J(\pi)=$ 
        & $\displaystyle \nabla J(\blau{\theta}) 
        = \mathsf{E}_{\tau \sim p(\tau \mid \blau{\theta})} 
            \left[ 
                R(\tau) \cdot 
                \sum_{t=0}^{T-1} 
                \nabla_{\theta} \log \pi_{\blau{\theta}}(a_t|\bm{s}_t)
            \right]$ \\
\end{tabular}
\end{center}

Für die Gradienten der Objective Function gilt:
% Policy Gradient Theorem
\begin{satz}[Policy Gradient Theorem]
    Für die Gradienten der Objective Function gilt:
    %
    \begin{align*}
    \nabla_{\theta} J(\theta) 
    = \nabla_{\theta} \mathsf{E}_{\tau \sim p(\tau|\theta)} \left[ R(\tau) \right] 
    = \mathsf{E}_{\tau \sim p(\tau|\theta)} \left[ \nabla_{\theta} R(\tau) \right]
    \end{align*}    
\end{satz}

\proof{
    \begin{itemize}
        \item Direkte Ableitung der Objective Function:
        %
            \begin{align*}
                \nabla_{\theta} J(\theta)
                &=
                \nabla_{\theta} \int R(\tau) \cdot p(\tau|\theta) \mathrm{d}\tau \\
                &=
                \int R(\tau) \cdot \nabla_{\theta} p(\tau|\theta) \mathrm{d}\tau \\
                &=
                \int R(\tau) \cdot p(\tau|\theta) \cdot 
                \frac{\nabla_{\theta} p(\tau|\theta)}{p(\tau|\theta)} \mathrm{d}\tau \\
                &=
                \int R(\tau) \cdot p(\tau|\theta) \cdot 
                \nabla_{\theta} \log p(\tau|\theta) \mathrm{d}\tau \\
                &=
                \mathsf{E}_{\tau \sim p(\tau|\theta)} 
                \left[ 
                    R(\tau) \cdot \nabla_{\theta} \log p(\tau|\theta) 
                \right]        
            \end{align*}
            %
            \item  Wir berechnen nun $\nabla_{\theta} \log p(\tau|\theta)$:
    %
                \begin{align*}
                    \nabla_{\theta} \log p(\tau|\theta)
                    &=
                    \nabla_{\theta} \log 
                    \left[
                        p(\bm{s}_0) \cdot 
                        \prod_{t=0}^{T-1} \pi_{\theta}(a_t|\bm{s}_t) 
                        \cdot T_{a_t}(\bm{s}_t, \bm{s}_{t+1})
                    \right] \\
                    &=
                    \nabla_{\theta} 
                    \left[
                        \log p(\bm{s}_0) + 
                        \sum_{t=0}^{T-1} 
                        \log \pi_{\theta}(a_t|\bm{s}_t) +
                        \log T_{a_t}(\bm{s}_t, \bm{s}_{t+1})
                    \right] \\
                    &=
                    \sum_{t=0}^{T-1} 
                    \nabla_{\theta} \log \pi_{\theta}(a_t|\bm{s}_t)
                \end{align*}
            %
            \item  Wir erhalten somit: 
    %           %
                \begin{align*}
                    \nabla_{\theta} J(\theta) 
                    &= 
                    \mathsf{E}_{\tau \sim p(\tau|\theta)} 
                    \left[ 
                        R(\tau) \cdot \nabla_{\theta} \log p(\tau|\theta) 
                    \right]
                    =
                    \mathsf{E}_{\tau \sim p(\tau|\theta)} 
                    \left[ 
                        R(\tau) \cdot 
                        \sum_{t=0}^{T-1} 
                        \nabla_{\theta} \log \pi_{\theta}(a_t|\bm{s}_t)
                    \right]
                \end{align*}
    \end{itemize}
}

\begin{align*}
    R &\leftarrow \sum_{k=t}^{T-1} \gamma^{k-t} r_{k} \\
    \theta &\leftarrow \theta + \alpha 
    \cdot R \cdot \nabla_{\theta} \log \pi_{\theta}(a_t|\bm{s}_t)
\end{align*}

\proof{
    Es gilt $\pi_\epsilon = \pi + \epsilon \cdot h$
    %
    \begin{align*}
        \frac{d}{d \epsilon} J(\pi_\epsilon) \mid_{\epsilon=0}
        &=
        \frac{d}{d \epsilon}
        \mathsf{E}_{\tau \sim p(\tau|\pi_\epsilon)}
        \left[
            R(\tau)
        \right]
        \mid_{\epsilon=0} \\
        &=
        \frac{d}{d \epsilon}
        \int R(\tau) \cdot p(\tau|\pi_\epsilon) \mathrm{d}\tau
        \mid_{\epsilon=0} \\
        &=
        \int R(\tau) \cdot
        \underbrace{
            \frac{d}{d \epsilon} p(\tau|\pi_\epsilon)
        }_{
            p(\tau|\pi_\epsilon) \cdot \frac{d}{d \epsilon} \log p(\tau|\pi_\epsilon)
        } \mathrm{d}\tau
        \mid_{\epsilon=0} \\
        &=
        \int R(\tau) \cdot p(\tau|\pi) \cdot
        \frac{d}{d \epsilon} \log p(\tau|\pi_\epsilon)
        \mid_{\epsilon=0} \mathrm{d}\tau 
    \end{align*}

    Wir berechnen nun $\frac{d}{d \epsilon} \log p(\tau|\pi_\epsilon)$:
    % 
    \begin{align*}
        \frac{d}{d \epsilon} \log p(\tau|\pi_\epsilon)
        &=
        \frac{d}{d \epsilon} \log
        \left[
            p(\bm{s}_0) \cdot 
            \prod_{t=0}^{T-1} \pi_\epsilon(a_t|\bm{s}_t) 
            \cdot T_{a_t}(\bm{s}_t, \bm{s}_{t+1})
        \right] \\
        &=
        \frac{d}{d \epsilon}
        \left[
            \log p(\bm{s}_0) + 
            \sum_{t=0}^{T-1} 
            \log \pi_\epsilon(a_t|\bm{s}_t) +
            \log T_{a_t}(\bm{s}_t, \bm{s}_{t+1})
        \right] \\
        &=
        \sum_{t=0}^{T-1} 
        \frac{d}{d \epsilon} \log 
        \left( 
            \vphantom{int}
            \pi(a_t|\bm{s}_t) + \epsilon \cdot h(a_t|\bm{s}_t)
        \right) 
        \\
        &=
        \sum_{t=0}^{T-1} 
        \frac{h(a_t|\bm{s}_t)}{\pi(a_t|\bm{s}_t) + \epsilon \cdot h(a_t|\bm{s}_t)} 
        \overset{\epsilon = 0}{=}
        \sum_{t=0}^{T-1} 
        \frac{h(a_t|\bm{s}_t)}{\pi(a_t|\bm{s}_t)} 
    \end{align*}

    Wir erhalten somit:
    %
    \begin{align*}
        \frac{d}{d \epsilon} J(\pi_\epsilon) \mid_{\epsilon=0}
        &=
        \int R(\tau) \cdot p(\tau|\pi) \cdot
        \sum_{t=0}^{T-1} 
        \frac{h(a_t|\bm{s}_t)}{\pi(a_t|\bm{s}_t)} 
        \mathrm{d}\tau \\
        &=
        \mathsf{E}_{\tau \sim p(\tau|\pi)}
        \left[
            R(\tau) \cdot 
            \sum_{t=0}^{T-1} 
            \frac{h(a_t|\bm{s}_t)}{\pi(a_t|\bm{s}_t)} 
        \right]
        \\
        &=
        \sum_{t=0}^{T-1}
        \mathsf{E}_{\tau \sim p(\tau|\pi)}
        \left[
            R(\tau) \cdot 
            \frac{h(a_t|\bm{s}_t)}{\pi(a_t|\bm{s}_t)} 
        \right]
    \end{align*}

    Für jeden festen Zeitschritt $t$ gilt somit:
    %
    \begin{align*}
        \mathsf{E}_{\tau \sim p(\tau|\pi)}
        \left[
            R(\tau) \cdot 
            \frac{h(a_t|\bm{s}_t)}{\pi(a_t|\bm{s}_t)} 
        \right]
        &=
        \mathsf{E}_{\bm{s}_t,a_t}
        \left[
            \mathsf{E}_{\tau \sim p(\tau|\bm{s}_t,a_t)}
            \left[
                R(\tau) \cdot 
                \frac{h(a_t|\bm{s}_t)}{\pi(a_t|\bm{s}_t)} 
            \right]
        \right]
        \\
        &=
        \mathsf{E}_{\bm{s}_t,a_t}
        \left[
            \frac{h(a_t|\bm{s}_t)}{\pi(a_t|\pi, \bm{s}_t)}
            \cdot
            \underbrace{
                \mathsf{E}_{\tau \sim p(\tau | \pi)}
                \left[
                    R(\tau) \mid \bm{s}_t,a_t
                \right]
            }_{
                = Q^{\pi}(\bm{s}_t,a_t)
            }
        \right]
    \end{align*}
}

\subsection{Deep Deterministic Policy Gradient}

\subsection{Two-Link-Revolute Manipulator}

\newpage

\printbibliography


\end{document}
