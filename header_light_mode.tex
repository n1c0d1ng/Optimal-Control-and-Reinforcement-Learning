% !TeX root = ./Roboter.tex
\documentclass[a4paper,12pt, oneside]{scrartcl}
\usepackage[ngerman]{babel}
\usepackage[utf8]{inputenc}
\setlength{\parindent}{0em}
\usepackage{eurosym}
\usepackage{amsmath}
\usepackage{amssymb}
\usepackage{dsfont}
\usepackage{polynom}
\usepackage{graphicx}
\usepackage{caption}
\usepackage{tikz}
\usepackage[a4paper,portrait,left=1.0cm,right=1.0cm,top=2cm,bottom=2cm]{geometry}
\usepackage{pgf} % Zur Einbindung der PGF Files
\usepackage{bm}

% Light Mode Customization
\usepackage{xcolor}
%\definecolor{backgroundgray}{RGB}{255,255,255}  % Weiß als Seitenhintergrund
\definecolor{textwhite}{RGB}{0,0,0}            % Schwarz als globale Schriftfarbe
\definecolor{boxgray}{RGB}{245,245,245}        % Sehr helles Grau für Box-Hintergründe
\definecolor{boxframe}{RGB}{200,200,200}       % Helles Grau für Box-Rahmen
\pagecolor{boxgray}  % Hintergrund auf Weiß
\color{textwhite}           % Text auf Schwarz

% Eigene Farben definiert (nun optimiert für hellen Hintergrund)
\definecolor{rot}{RGB}{204,0,0}
\definecolor{blau}{RGB}{0,102,204}
\definecolor{orange}{RGB}{204,85,0}
\definecolor{gruen}{RGB}{0,153,0}
\definecolor{dunkelgruen}{RGB}{0,153,0}
\definecolor{dunkelblau}{RGB}{0,128,128}
\definecolor{backcolor}{RGB}{235,245,255}
\definecolor{frontcolor}{RGB}{0,153,0}
\definecolor{hellblau}{RGB}{180,210,235}
\definecolor{lila}{RGB}{153,0,153}
\definecolor{gelb}{RGB}{204,153,0}
\definecolor{dunkelrot}{RGB}{153,0,51}
\definecolor{dunkelgelb}{RGB}{153,153,0}

% Section colors (Headings)
%\usepackage{sectsty}
%\sectionfont{\color{textwhite}}
%\subsectionfont{\color{textwhite}}
%\subsubsectionfont{\color{textwhite}}

% Box customization mit mdframed
\usepackage{mdframed}
\mdfdefinestyle{mystyle}{
    leftmargin=0pt,
    linecolor=boxframe,
    linewidth=1pt,
    backgroundcolor=boxgray,
    fontcolor=textwhite,
    %frametitlefont={\color{gruen}\bfseries}
}

% Custom theorem environments
\newmdtheoremenv[style=mystyle]{satz}{\color{rot}Satz}[section] % linecolor=rot
\newmdtheoremenv[style=mystyle]{definition}[satz]{\color{gelb}Definition}
\newmdtheoremenv[style=mystyle,leftmargin=-10pt]{beispiel}[satz]{\color{lila}Beispiel}
\newmdtheoremenv[style=mystyle,leftmargin=-10pt]{bem}[satz]{Anmerkung}
\newmdtheoremenv[style=mystyle,leftmargin=-10pt]{ex}{Aufgabe} % \color{gruen}

% Proof- und Lösungs-Kommandos
\newcommand{\proof}[1]{
    \begingroup
    \textit{\hfill \\ \gruen{\textbf{Beweis:}} }#1 \hfill $\Box$
    \endgroup
}
\newcommand{\loesung}[1]{
    \begingroup
    \renewcommand{\textsc}[1]{{\rmfamily\scshape##1}}
    \renewcommand{\emph}[1]{{\normalfont##1}}
    \addtokomafont{subsection}{\color{gruen}}
    \addtokomafont{subsubsection}{\color{gruen}}
    \addtokomafont{caption}{\color{gruen}}
    \addtokomafont{captionlabel}{\color{gruen}}
    \sffamily\slshape\color{textwhite}
    \textbf{\hfill \\ Lösung: }#1 
    \endgroup
}

% Listings für Python
\usepackage[german,tworuled]{algorithm2e}
\usepackage{listings}
\lstset{
    commentstyle=\color{gruen},
    keywordstyle=\color{blau},
    language=Python,
    backgroundcolor=\color{boxgray},
    basicstyle=\ttfamily\color{textwhite},
    stringstyle=\color{rot},
    tabsize=2,
    literate={@}{{@}}1,
    title=\lstname
}

% Hyperref Settings
\usepackage[pdfstartview=Fit,plainpages=false,colorlinks=true,linkcolor=gruen,linktocpage=true,hyperfootnotes=false,bookmarksopen=false]{hyperref}

% Eigene Kommandos
\newcommand{\e}[1]{\text{e}^{#1}}
\newcommand{\rot}[2][rot]{\textcolor{#1}{#2}}
\newcommand{\gruen}[2][gruen]{\textcolor{#1}{#2}}
\newcommand{\frontcolor}[2][frontcolor]{\textcolor{#1}{#2}}
\newcommand{\gelb}[2][gelb]{\textcolor{#1}{#2}}
\newcommand{\orange}[2][orange]{\textcolor{#1}{#2}}
\newcommand{\blau}[2][blau]{\textcolor{#1}{#2}}
\newcommand{\hellblau}[2][hellblau]{\textcolor{#1}{#2}}
\newcommand{\lila}[2][lila]{\textcolor{#1}{#2}}
\newcommand{\dunkelrot}[2][dunkelrot]{\textcolor{#1}{#2}}
\newcommand{\dunkelgelb}[2][dunkelgelb]{\textcolor{#1}{#2}}

%%%%%%%%%%%%%%%%%%%%%%%%%%%%%%%%%%%%%%%%%%%%%%%%%%%%%%%%%%%%%%%%%%%%%%%
\newcommand{\norm}[1]{\left\lVert#1\right\lVert}
\newcommand{\dv}{\thinspace \mathrm{d}v}
\newcommand{\dw}{\thinspace \mathrm{d}w}
\newcommand{\dx}{\thinspace \mathrm{d}x}
\newcommand{\dy}{\thinspace \mathrm{d}y}
\newcommand{\dt}{\thinspace \mathrm{d}t}
\newcommand{\dr}{\thinspace \mathrm{d}r}
\newcommand{\ds}{\thinspace \mathrm{d}s}
\newcommand{\du}{\thinspace \mathrm{d}u}
\newcommand{\dz}{\thinspace \mathrm{d}z}
\newcommand{\dW}{\thinspace \mathrm{d}W}
\newcommand{\dX}{\thinspace \mathrm{d}X}
\newcommand{\dP}{\thinspace \mathrm{d}P}
\DeclareMathOperator*{\esssup}{ess\,sup}
\DeclareMathOperator*{\argmin}{arg\,min}
%% adjust title page to be centered
% \typearea[18mm]{11}