% !TeX root = ./Roboter.tex
\documentclass[a4paper,12pt, oneside]{scrartcl}
\usepackage[ngerman]{babel}
\usepackage[utf8]{inputenc}
\setlength{\parindent}{0em}
\usepackage{eurosym}
\usepackage{amsmath}
\usepackage{amssymb}
\usepackage{dsfont}
\usepackage{polynom}
\usepackage{graphicx}
\usepackage{caption}
\usepackage{tikz}
\usepackage[a4paper,portrait,left=1.0cm,right=1.0cm,top=2cm,bottom=2cm]{geometry}
\usepackage{pgf} % Zur Einbindung der PGF Files
\usepackage{bm}

% Light Mode Customization
\usepackage{xcolor}
%\definecolor{backgroundgray}{RGB}{255,255,255}  % Weiß als Seitenhintergrund
\definecolor{textwhite}{RGB}{0,0,0}            % Schwarz als globale Schriftfarbe
\definecolor{boxgray}{RGB}{245,245,245}        % Sehr helles Grau für Box-Hintergründe
\definecolor{boxframe}{RGB}{200,200,200}       % Helles Grau für Box-Rahmen
\pagecolor{boxgray}  % Hintergrund auf Weiß
\color{textwhite}           % Text auf Schwarz

% Eigene Farben definiert (nun optimiert für hellen Hintergrund)
\definecolor{rot}{RGB}{204,0,0}
\definecolor{blau}{RGB}{0,102,204}
\definecolor{orange}{RGB}{204,85,0}
\definecolor{gruen}{RGB}{0,153,0}
\definecolor{dunkelgruen}{RGB}{0,153,0}
\definecolor{dunkelblau}{RGB}{0,128,128}
\definecolor{backcolor}{RGB}{235,245,255}
\definecolor{frontcolor}{RGB}{0,153,0}
\definecolor{hellblau}{RGB}{180,210,235}
\definecolor{lila}{RGB}{153,0,153}
\definecolor{gelb}{RGB}{204,153,0}
\definecolor{dunkelrot}{RGB}{153,0,51}
\definecolor{dunkelgelb}{RGB}{153,153,0}

% Section colors (Headings)
%\usepackage{sectsty}
%\sectionfont{\color{textwhite}}
%\subsectionfont{\color{textwhite}}
%\subsubsectionfont{\color{textwhite}}

% Box customization mit mdframed
\usepackage{mdframed}
\mdfdefinestyle{mystyle}{
    leftmargin=0pt,
    linecolor=boxframe,
    linewidth=1pt,
    backgroundcolor=boxgray,
    fontcolor=textwhite,
    %frametitlefont={\color{gruen}\bfseries}
}

% Custom theorem environments
\newmdtheoremenv[style=mystyle]{satz}{\color{rot}Satz}[section] % linecolor=rot
\newmdtheoremenv[style=mystyle]{definition}[satz]{\color{gelb}Definition}
\newmdtheoremenv[style=mystyle,leftmargin=-10pt]{beispiel}[satz]{\color{lila}Beispiel}
\newmdtheoremenv[style=mystyle,leftmargin=-10pt]{bem}[satz]{Anmerkung}
\newmdtheoremenv[style=mystyle,leftmargin=-10pt]{ex}{Aufgabe} % \color{gruen}

% Proof- und Lösungs-Kommandos
\newcommand{\proof}[1]{
    \begingroup
    \textit{\hfill \\ \gruen{\textbf{Beweis:}} }#1 \hfill $\Box$
    \endgroup
}
\newcommand{\loesung}[1]{
    \begingroup
    \renewcommand{\textsc}[1]{{\rmfamily\scshape##1}}
    \renewcommand{\emph}[1]{{\normalfont##1}}
    \addtokomafont{subsection}{\color{gruen}}
    \addtokomafont{subsubsection}{\color{gruen}}
    \addtokomafont{caption}{\color{gruen}}
    \addtokomafont{captionlabel}{\color{gruen}}
    \sffamily\slshape\color{textwhite}
    \textbf{\hfill \\ Lösung: }#1 
    \endgroup
}

% Listings für Python
\usepackage[german,tworuled]{algorithm2e}
\usepackage{listings}
\lstset{
    commentstyle=\color{gruen},
    keywordstyle=\color{blau},
    language=Python,
    backgroundcolor=\color{boxgray},
    basicstyle=\ttfamily\color{textwhite},
    stringstyle=\color{rot},
    tabsize=2,
    literate={@}{{@}}1,
    title=\lstname
}

% Hyperref Settings
\usepackage[pdfstartview=Fit,plainpages=false,colorlinks=true,linkcolor=gruen,linktocpage=true,hyperfootnotes=false,bookmarksopen=false]{hyperref}

% Eigene Kommandos
\newcommand{\e}[1]{\text{e}^{#1}}
\newcommand{\rot}[2][rot]{\textcolor{#1}{#2}}
\newcommand{\gruen}[2][gruen]{\textcolor{#1}{#2}}
\newcommand{\frontcolor}[2][frontcolor]{\textcolor{#1}{#2}}
\newcommand{\gelb}[2][gelb]{\textcolor{#1}{#2}}
\newcommand{\orange}[2][orange]{\textcolor{#1}{#2}}
\newcommand{\blau}[2][blau]{\textcolor{#1}{#2}}
\newcommand{\hellblau}[2][hellblau]{\textcolor{#1}{#2}}
\newcommand{\lila}[2][lila]{\textcolor{#1}{#2}}
\newcommand{\dunkelrot}[2][dunkelrot]{\textcolor{#1}{#2}}
\newcommand{\dunkelgelb}[2][dunkelgelb]{\textcolor{#1}{#2}}

%%%%%%%%%%%%%%%%%%%%%%%%%%%%%%%%%%%%%%%%%%%%%%%%%%%%%%%%%%%%%%%%%%%%%%%
\newcommand{\norm}[1]{\left\lVert#1\right\lVert}
\newcommand{\dv}{\thinspace \mathrm{d}v}
\newcommand{\dw}{\thinspace \mathrm{d}w}
\newcommand{\dx}{\thinspace \mathrm{d}x}
\newcommand{\dy}{\thinspace \mathrm{d}y}
\newcommand{\dt}{\thinspace \mathrm{d}t}
\newcommand{\dr}{\thinspace \mathrm{d}r}
\newcommand{\ds}{\thinspace \mathrm{d}s}
\newcommand{\du}{\thinspace \mathrm{d}u}
\newcommand{\dz}{\thinspace \mathrm{d}z}
\newcommand{\dW}{\thinspace \mathrm{d}W}
\newcommand{\dX}{\thinspace \mathrm{d}X}
\newcommand{\dP}{\thinspace \mathrm{d}P}
\DeclareMathOperator*{\esssup}{ess\,sup}
\DeclareMathOperator*{\argmin}{arg\,min}
%% adjust title page to be centered
% \typearea[18mm]{11}
\usepackage{listings}
\definecolor{codebg}{rgb}{0.92,0.92,0.92}
\definecolor{keywordcolor}{rgb}{0.2,0.2,0.7}
\definecolor{stringcolor}{rgb}{0.8,0.1,0.1}
\definecolor{commentcolor}{rgb}{0.4,0.4,0.4}


\lstset{
  backgroundcolor=\color{codebg},
  %frame=single,
  %numbers=left,
  %numberstyle=\tiny\color{gray},
  basicstyle=\ttfamily\small,
  keywordstyle=\color{keywordcolor}\bfseries,
  stringstyle=\color{stringcolor},
  commentstyle=\color{gruen}\itshape,
  language=Python,
  tabsize=2,
  showstringspaces=false
}



\begin{document}

\begin{titlepage}

\newcommand{\TitleLineI}{Modellierung eines Roboters} 
\newcommand{\Author}{Nicolas Schäfer}
\newcommand{\SubmissionDate}{Saarbrücken, \today}

\thispagestyle{empty}



%\begin{center}
	%\includegraphics[scale=0.3]{cover.jpg}
%	\includegraphics[scale=0.3]{cover2.jpg}
%\end{center}


\rule{\textwidth}{0.4pt}

\begin{center}
	\begin{Large}
		Modellierung eines Roboters
	\end{Large}
\end{center}

\vspace{2.0cm}
\begin{center}
		\begin{tabular}{c}
			\Author \\
			 \SubmissionDate 
			%%Uncomment the following line for a third title line 
%			\TitleLineIII \\
		\end{tabular}
\end{center}
\vspace{1.5cm}

\end{titlepage}


\pagenumbering{roman}

\tableofcontents

\newpage

\pagenumbering{arabic}

\raggedbottom

\section{Modellierung}


\begin{align*}
\underbrace{f - mg}_{\textsf{Kräftebilanz}} = \underbrace{m x''}_{\textsf{Newtons 2nd Law}} &= \frac{\mathrm{d}}{\dt} \left[m \rot{x'(t)} \right] \\
f - \frac{\mathrm{\partial}}{\partial x} \underbrace{\left[ mg x(t) \right]}_{\textsf{Potenzielle Energie }V} &= \frac{\mathrm{d}}{\dt} \rot{\frac{\partial}{\partial x'}} \underbrace{\left[m \rot{ \frac{1}{2} x'(t)^2} \right]}_{\textsf{kinetische Energie } T} \\
f \dunkelgelb{- \frac{\partial V}{\partial x}} = \frac{\mathrm{d}}{\dt} \rot{\frac{\partial T}{\partial x'}}
&\Rightarrow
f = \frac{\mathrm{d}}{\dt} \rot{\frac{\partial L}{\partial x'}} - \dunkelgelb{\frac{\partial L}{\partial x}} 
\end{align*}

Wobei $L = T - V = \frac{1}{2}m x'(t)^2 -mg x(t)$ gilt.
Für generalisierte Koordinaten $\textbf{q} = \theta \in \mathbb{R}^n$
%
\begin{align*}
\frac{d}{dt} \left( \frac{\partial T}{\partial \theta_{i}'}  \right) - \frac{\partial T}{\partial \theta_i}  + \frac{\partial V}{\partial \theta_i} = \tau_i \quad i=1, \cdots , n
\end{align*}

\subsection{Two Link Revolute Manipulator}

% Graphik des Roboterarms
%--------------------------------------------------------------------------------------------------------
\begin{center}
\begin{tikzpicture}[scale = 1.75, every node/.style={transform shape}]
    % Koordinatensystem
    %\draw[very thin, gray] (-5,-5) grid (5,5); % Raster
    %\draw[->] (-5,0) -- (5,0); % x-Achse
    %\draw[->] (0,-5) -- (0,5); % y-Achse
    
    % Achsen und Hilfslinien
    \draw[white, dashed, ->] (0.0,-1.0) -- (3.5,-1.0);
    \draw[white, dashed, ->] (0.0,-1.0) -- (0.0,3.0);
    
    % Schwerpunkte
    \draw[gelb, dashed] (1.0,-0.5) -- (1.0,-1.0);
    \draw[gelb, dashed] (1.0,-0.5) -- (0,-0.5);
    \node at (-0.2, -0.5) {\tiny $\gelb{y_1}$};  % Punktbeschriftung
    \node at (1, -1.2) {\tiny $\gelb{x_1}$}; 
    \draw[orange, dashed] (2.5,1) -- (0.0,1);
    \draw[orange, dashed] (2.5,1) -- (2.5,-1);
    \node at (-0.2, 1.0) {\tiny $\orange{y_2}$};  % Punktbeschriftung
    \node at (2.5, -1.2) {\tiny $\orange{x_2}$}; 
    
    % Erste Zeichnung (Original)
    \fill[hellblau] (-4,-1) circle (0.25);  % Füllt das Objekt mit hellblau
    \draw[white, line width = 0.2mm] (-4,-1) circle (0.25);  % Weißer Rand um das Objekt
    
    \draw[white, line width = 0.2mm] (-4.25,-1.4) rectangle (-3.75,-1);  % Weißer Rand um das Rechteck
    \fill[hellblau] (-4.25,-1.4) rectangle (-3.75,-1);  % Füllt das Rechteck mit hellblau
    \draw[thick, white] (-4.5,-1.4) -- (-3.5,-1.4);
    
    % Grüne Linie angepasst (Endpunkt auf (-2.0, 0.0))
    \draw[thick, gruen] (-4,-1) -- (-2.0,0.0);
    
    % Rote Linie angepasst (von (-2.0, 0.0) nach (-1.0, 2.0))
    \draw[thick, dunkelrot] (-2.0,0.0) -- (-1.0,2.0);
    
    % Glieder
    \fill[blau] (-2.0,0.0) circle (0.1);
    \draw[white, thin] (-2.0,0.0) circle (0.1);  % Weißer Rand um das Objekt
    \fill[blau] (-4.0,-1.0) circle (0.1);
    \draw[white, thin] (-4.0,-1.0) circle (0.1);  % Weißer Rand um das Objekt
    
	% Beschriftungen
	\node at (-3, -0.75) {\tiny $\gruen{l_1}$};  % 
	\node at (-1.5, 1.5) {\tiny $\dunkelrot{l_2}$};  % 
	
	% Schwerpunkte 
	\fill[white] (-3.0,-0.5) circle (0.05);
	\fill[white] (-1.5,1) circle (0.05);
	
	\draw[white, <->] (-1.85,0.15) -- (-1.5,0.85);
	\draw[white, <->] (-3.85,-0.85) -- (-3.1,-0.45);
	
	% Beschriftungen
	\node at (-3.5, -0.4) {\tiny $a_1$};  % 
	\node at (-1.5, 0.5) {\tiny $a_2$};  % 
    
    % Zweite Zeichnung (verschoben um 4 Einheiten entlang der x-Achse)
    \fill[hellblau] (0,-1) circle (0.25);  % Füllt das Objekt mit hellblau
    \draw[white, line width = 0.2mm] (0,-1) circle (0.25);  % Weißer Rand um das Objekt
    
    \draw[white, line width = 0.2mm] (-0.25,-1.4) rectangle (0.25,-1);  % Weißer Rand um das Rechteck
    \fill[hellblau] (-0.25,-1.4) rectangle (0.25,-1);  % Füllt das Rechteck mit hellblau
    \draw[thick, white] (-0.5,-1.4) -- (0.5,-1.4);
    
    % Grüne Linie (verschoben)
    \draw[thick, gruen] (0,-1) -- (2.0,0.0);
    \draw[thick, gruen, dotted] (0,-1) -- (3.5,0.75);
    
    % Rote Linie (verschoben)
    \draw[thick, dunkelrot] (2.0,0.0) -- (3.0,2.0);
    
    % Glieder
    \fill[blau] (2.0,0.0) circle (0.1);
    \draw[white, thin] (2.0,0.0) circle (0.1);  % Weißer Rand um das Objekt
    \fill[blau] (0.0,-1.0) circle (0.1);
    \draw[white, thin] (0.0,-1.0) circle (0.1);  % Weißer Rand um das Objekt
    \fill[blau] (3.0,2.0) circle (0.1);
    \draw[white, thin] (3.0,2.0) circle (0.1);  % Weißer Rand um das Objekt
    

    
    % Winkel
    \draw [gruen, thick] (1,-1)  arc[start angle=0,end angle=45,radius=0.5cm];
	\node at (0.75, -0.85) {\tiny $\gruen{\theta_1}$};  % Winkelbeschriftung
	
	\draw [dunkelrot, thick] (3,0.5)  arc[start angle=0,end angle=80,radius=0.5cm];
	\node at (2.7, 0.6) {\tiny $\dunkelrot{\theta_2}$};  % Winkelbeschriftung
\end{tikzpicture}
\end{center}
%--------------------------------------------------------------------------------------------------------

\begin{align*}
\gelb{x_1} = a_1 \cdot \cos(\gruen{\theta_1}) \quad \gelb{y_1} = a_1 \cdot \sin(\gruen{\theta_1}) 
\end{align*}

Gelenk 2 befindet sich in  $(l_1 \cos(\gruen{\theta_1}),\;l_1 \sin(\gruen{\theta_1}))$. Der Schwerpunkt von Link~2 ist \emph{zusätzlich} noch \(a_2\) vom Gelenk~2 entfernt
und besitzt eine absolute Winkelausrichtung \(\theta_1 + \theta_2\). Daher:
%
\begin{align*}
\orange{x_2} &= l_1 \cdot \cos(\gruen{\theta_1}) + a_2 \cos(\gruen{\theta_1} + \dunkelrot{\theta_2}) \\
\orange{y_2} &= l_1 \cdot \sin(\gruen{\theta_1}) + a_2 \sin(\gruen{\theta_1} + \dunkelrot{\theta_2})
\end{align*}


Die Euler-Lagrange Gleichung des Systems
%
\begin{align*}
D(\theta)
\theta''
+ 
C(\theta, \theta')
\theta'
+
G(\theta)
=
\tau
\end{align*}


\subsubsection{Herleitung der kinetischen Energie}

Für die translatorische Energie von Link $1$ gilt:
%
\begin{align*}
T_{1,trans} 
= \frac{m_1}{2} \cdot (\gelb{x_{1}'(t)}^2 + y_{1}'(t)^2) 
= \frac{m_1}{2} \cdot (\gelb{( - a_1 \sin(\theta_1) \theta_{1}' )}^2 + (a_1 \cos(\theta_1) \theta_{1}')^2)  
= \frac{m_1}{2} \cdot a_1^2  \cdot \gruen{\theta_{1}'(t)}^2 
\end{align*}

Es gilt $T_1 = T_{1,trans} + T_{1, rot}$ und somit

\begin{align*}
T_1 
= \frac{m_1}{2} a_1^2 \gruen{\theta_{1}'(t)}^2 + \frac{I_1}{2} \gruen{\theta_{1}'(t)}^2 
= \frac{m_1 a_1^2 + I_1}{2} \gruen{\theta_{1}'(t)}^2
\end{align*}

Analog für die Energie an Link $2$ gilt:

\begin{align*}
T_2
&= \underbrace{\frac{m_2}{2} \cdot (\orange{x_{2}'(t)}^2 + y_{2}'(t)^2)}_{\textsc{translatorisch}} + \underbrace{\frac{I_2}{2} \cdot (\gruen{\theta_{1}'(t)} + \dunkelrot{\theta_{2}'(t)})^2}_{\textsc{rotorisch}}
\end{align*}

\begin{align*}
T(\gruen{\theta_{1}}, \gruen{\theta_{1}'} , \dunkelrot{\theta_{2}}, \dunkelrot{\theta_{2}'} )
=
\frac{1}{2} 
\begin{pmatrix}
\gruen{\theta_{1}'} & \dunkelrot{\theta_{2}'}
\end{pmatrix} 
D(\gruen{\theta_{1}},  \dunkelrot{\theta_{2}})
\begin{pmatrix} \gruen{\theta_{1}'} \\ \dunkelrot{\theta_{2}'} \end{pmatrix} 
=
\frac{1}{2}  D_{\gruen{1 1}} \gruen{\theta_{1}'}(t)^2 + D_{\gruen{1}\dunkelrot{2}} \gruen{\theta_{1}'} \dunkelrot{\theta_{2}'} + \frac{1}{2} D_{\dunkelrot{22}} \dunkelrot{\theta_{2}'}(t)^2
\end{align*}


\begin{align*}
x_{2}'(t) 
&=  -l_1 \sin(\gruen{\theta_1}) \gruen{\theta_{1}'} - a_2 \sin(\gruen{\theta_1} + \dunkelrot{\theta_2}) (\gruen{\theta_{1}} ' + \dunkelrot{\theta_2} ') \\
y_{2}'(t)
&=
l_1 \cos(\gruen{\theta_{1}}) \gruen{\theta_{1}'} + a_2 \cos(\gruen{\theta_{1}} + \dunkelrot{\theta_2}) (\gruen{\theta_{1}} ' + \dunkelrot{\theta_2} ') 
\end{align*}

%-------------------------------------------------------------------------------------------------
\begin{align*}
x_{2}'(t)^2 + y_{2}'(t)^2 
&=
\left[ \vphantom{\int} -l_1 \sin(\gruen{\theta_1}) \gruen{\theta_{1}'} - a_2 \sin(\gruen{\theta_1} + \dunkelrot{\theta_2}) (\gruen{\theta_{1}} ' + \dunkelrot{\theta_2} ') \right]^2
+
\left[\vphantom{\int} l_1 \cos(\gruen{\theta_{1}}) \gruen{\theta_{1}'} + a_2 \cos(\gruen{\theta_{1}} + \dunkelrot{\theta_2}) (\gruen{\theta_{1}} ' + \dunkelrot{\theta_2} ') \right]^2
\\
&=
l_1^2 \sin(\gruen{\theta_1})^2 \gruen{\theta_{1}'}^2 +2 l_1 a_2 \sin(\gruen{\theta_1}) \sin(\gruen{\theta_1} + \dunkelrot{\theta_2}) \gruen{\theta_{1}'} (\gruen{\theta_{1}} ' + \dunkelrot{\theta_{2}'} ) 
+ \underbrace{a_2^2 \sin(\gruen{\theta_1} + \dunkelrot{\theta_2})^2 (\gruen{\theta_{1}} ' + \dunkelrot{\theta_2} ')^2}_{a_2^2 \sin(\gruen{\theta_1} + \dunkelrot{\theta_2})^2 \cdot (\gruen{\theta_{1}'}^2 + 2\gruen{\theta_{1}'} \dunkelrot{\theta_{2}'} + \dunkelrot{\theta_{2}'}^2  )}  \\
&\quad +
l_1^2 \cos(\gruen{\theta_{1}})^2 \gruen{\theta_{1}'}^2 
+ 2 l_1 a_2 \cos(\gruen{\theta_{1}})  \cos(\gruen{\theta_{1}} + \dunkelrot{\theta_2}) \gruen{\theta_{1}'} (\gruen{\theta_{1}} ' + \dunkelrot{\theta_2} ')
+ \underbrace{a_2^2 \cos(\gruen{\theta_{1}} + \dunkelrot{\theta_2})^2 (\gruen{\theta_{1}} ' + \dunkelrot{\theta_2} ')^2}_{a_2^2 \cos(\gruen{\theta_1} + \dunkelrot{\theta_2})^2 \cdot (\gruen{\theta_{1}'}^2 + 2\gruen{\theta_{1}'} \dunkelrot{\theta_{2}'} + \dunkelrot{\theta_{2}'}^2  )} \\
&=
\left[ \vphantom{\int} \underbrace{l_1^2 \sin(\gruen{\theta_1})^2  + l_1^2 \cos(\gruen{\theta_{1}})^2}_{= l_1^2} + \underbrace{a_2^2 \sin(\gruen{\theta_1} + \dunkelrot{\theta_2})^2 + a_2^2 \cos(\gruen{\theta_1} + \dunkelrot{\theta_2})^2}_{= a_2^2} \right] \gruen{\theta_{1}'}^2 \\
&\quad + 2 l_1 a_2 \gruen{\theta_{1}'} \cdot 
\left[ \vphantom{\int}  \underbrace{\sin(\gruen{\theta_1}) \sin(\gruen{\theta_1} + \dunkelrot{\theta_2}) +   \cos(\gruen{\theta_{1}})  \cos(\gruen{\theta_{1}} + \dunkelrot{\theta_2})}_{\cos(\dunkelrot{\theta_{2}})} \right]
 (\gruen{\theta_{1}} ' + \dunkelrot{\theta_2} ') \\
&\quad + 2 \cdot 
\left[  \vphantom{\int}  \underbrace{a_2^2 \sin(\gruen{\theta_1} + \dunkelrot{\theta_2})^2  + a_2^2 \cos(\gruen{\theta_1} + \dunkelrot{\theta_2})^2}_{= a_2^2}  \right] 
\gruen{\theta_{1}'} \dunkelrot{\theta_{2}'}  
+ \left[  \vphantom{\int}  \underbrace{a_2^2 \sin(\gruen{\theta_1} + \dunkelrot{\theta_2})^2  + a_2^2 \cos(\gruen{\theta_1} + \dunkelrot{\theta_2})^2}_{= a_2^2}  \right] \dunkelrot{\theta_{2}'}^2
\\
&=
\left[\vphantom{\int} l_1^2 +a_2^2 + 2 l_1 a_2  \cos(\dunkelrot{\theta_{2}}) \right] \gruen{\theta_{1}'}^2 
+ \left[ \vphantom{\int} 2a_2^2 + 2 l_1 a_2  \cos(\dunkelrot{\theta_{2}}) \right] \gruen{\theta_{1}'} \dunkelrot{\theta_{2}'} + a_2^2 \dunkelrot{\theta_{2}'}^2
\end{align*}

%------------------------------------------------------------------------------------------------------
Für die kinetische Energie gilt somit
%
\begin{align*}
T 
&= 
T_1  + T_2 
\\
&=
\frac{m_1 a_1^2 + I_1}{2} \gruen{\theta_{1}'}^2 + \frac{m_2}{2} \cdot (x_{2}'(t)^2 + y_{2}'(t)^2 ) + \frac{I_2}{2} \cdot \underbrace{(\gruen{\theta_{1}'} + \dunkelrot{\theta_{2}'})^2}_{\gruen{\theta_{1}'}^2 + 2\gruen{\theta_{1}'}\dunkelrot{\theta_{2}'} + \dunkelrot{\theta_{2}'}^2} \\
&=
\left[\vphantom{\int}  \frac{m_1 a_1^2 + I_1 + m_2 \cdot (l_1^2 +a_2^2 + 2 l_1 a_2  \cos(\dunkelrot{\theta_{2}}) ) + I_2}{2}  \right] 
\gruen{\theta_{1}'}^2 
+ \left[ \vphantom{\int}  \frac{m_2 \cdot (2a_2^2 + 2 l_1 a_2  \cos(\dunkelrot{\theta_{2}})) + 2I_2}{2} \right] 
\gruen{\theta_{1}'} \dunkelrot{\theta_{2}'} \\
&\quad + \left[\vphantom{\int} \frac{m_2 a_2^2 + I_2}{2}  \right] \dunkelrot{\theta_{2}'}^2 \\
&=
\frac{1}{2} 
\begin{pmatrix}
\gruen{\theta_{1}'} & \dunkelrot{\theta_{2}'}
\end{pmatrix} 
\underbrace{\begin{pmatrix}
m_1 a_1^2 + I_1 + m_2 \cdot (l_1^2 +a_2^2 + 2 l_1 a_2  \cos(\dunkelrot{\theta_{2}}) ) + I_2 &  m_2 \cdot (a_2^2 +  l_1 a_2  \cos(\dunkelrot{\theta_{2}})) + I_2 \\
m_2 \cdot (a_2^2 +  l_1 a_2  \cos(\dunkelrot{\theta_{2}})) + I_2	& m_2 a_2^2 + I_2
\end{pmatrix}}_{D(\gruen{\theta_{1}},\dunkelrot{\theta_{2}}) = D(\dunkelrot{\theta_{2}})}
\begin{pmatrix} \gruen{\theta_{1}'} \\ \dunkelrot{\theta_{2}'} \end{pmatrix} 
\\
&=
\frac{1}{2}  D_{\gruen{1 1}} \gruen{\theta_{1}'}(t)^2 + D_{\gruen{1}\dunkelrot{2}} \gruen{\theta_{1}'} \dunkelrot{\theta_{2}'} + \frac{1}{2} D_{\dunkelrot{22}} \dunkelrot{\theta_{2}'}(t)^2
\end{align*}


\subsubsection{Aufstellen der Coriolis-Matrix}

Es gilt 


\begin{align*}
\frac{\partial T}{\partial \gruen{\theta_{1}}}
&= 
\frac{1}{2}  \frac{\partial D_{11}}{\partial \gruen{\theta_{1}}} \gruen{\theta_{1}'}^2 
+ \frac{\partial D_{12}}{\partial \gruen{\theta_{1}}} \gruen{\theta_{1}'} \dunkelrot{\theta_{2}'} 
+ \frac{1}{2} \frac{\partial D_{22}}{\partial \gruen{\theta_{1}}} \dunkelrot{\theta_{2}'}(t)^2
\end{align*}

\begin{align*}
\frac{\partial T}{\partial \dunkelrot{\theta_{2}}}
&= 
\frac{1}{2} \frac{\partial D_{11}}{\partial \dunkelrot{\theta_{2}}} \gruen{\theta_{1}'}^2
+ \frac{\partial D_{12}}{\partial \dunkelrot{\theta_{2}}} \gruen{\theta_{1}'} \dunkelrot{\theta_{2}'} 
+ \frac{1}{2} \frac{\partial D_{22}}{\partial \dunkelrot{\theta_{2}}} \dunkelrot{\theta_{2}'}(t)^2
\end{align*}

\begin{align*}
\frac{\partial T}{\partial \gruen{\theta_{1}'}} 
&= D_{\gruen{1 1}} \gruen{\theta_{1}'} + D_{\gruen{1}\dunkelrot{2}}  \dunkelrot{\theta_{2}'}
\end{align*}
%   

\begin{align*}
\frac{\partial T}{\partial \dunkelrot{\theta_{2}'}} 
&=
D_{\gruen{1}\dunkelrot{2}} \gruen{\theta_{1}'} + D_{\dunkelrot{22}} \dunkelrot{\theta_{2}'}
\end{align*}

Für die zeitlichen Ableitungen der Größen gilt 
%
\begin{align*}
\frac{\mathrm{d}}{\dt} \frac{\partial T}{\partial \gruen{\theta_{1}'}} 
&=
\underbrace{\frac{\mathrm{d}}{\dt} D_{\gruen{1 1}}}_{\frac{\partial D_{11}}{\partial \theta_{1}} \theta_{1}' + \frac{\partial D_{11}}{\partial \theta_{2}} \theta_{2}'} \cdot \gruen{\theta_{1}'} + D_{\gruen{1 1}} \gruen{\theta_{1}''} + \underbrace{\frac{\mathrm{d}}{\dt} D_{\gruen{1}\dunkelrot{2}}}_{\frac{\partial D_{12}}{\partial \theta_{1}} \theta_{1}' + \frac{\partial D_{12}}{\partial \theta_{2}} \theta_{2}'} \cdot \dunkelrot{\theta_{2}'} + D_{\gruen{1}\dunkelrot{2}} \dunkelrot{\theta_{2}''} 
\\
&= \frac{\partial D_{11}}{\partial \gruen{\theta_{1}}} \gruen{\theta_{1}'}^2
 + \frac{\partial D_{11}}{\partial \dunkelrot{\theta_{2}}} \gruen{\theta_{1}'} \dunkelrot{ \theta_{2}'} 
 +  D_{\gruen{1 1}} \gruen{\theta_{1}''} 
 + \frac{\partial D_{12}}{\partial \gruen{\theta_{1}}} \gruen{\theta_{1}'} \dunkelrot{\theta_{2}'} 
 + \frac{\partial D_{12}}{\partial \dunkelrot{\theta_{2}}} \dunkelrot{\theta_{2}'}^2 
 + D_{\gruen{1}\dunkelrot{2}} \dunkelrot{\theta_{2}''} \\
&=
\frac{\partial D_{11}}{\partial \gruen{\theta_{1}}} \gruen{\theta_{1}'}^2
+ \left[ \frac{\partial D_{11}}{\partial \dunkelrot{\theta_{2}}} + \frac{\partial D_{12}}{\partial \gruen{\theta_{1}}} \right] \gruen{\theta_{1}'} \dunkelrot{ \theta_{2}'} 
+ \frac{\partial D_{12}}{\partial \dunkelrot{\theta_{2}}} \dunkelrot{\theta_{2}'}^2 
 +  D_{\gruen{1 1}} \gruen{\theta_{1}''} 
  + D_{\gruen{1}\dunkelrot{2}} \dunkelrot{\theta_{2}''}
\end{align*}

\begin{align*}
\frac{\mathrm{d}}{\dt} \frac{\partial T}{\partial \dunkelrot{\theta_{2}'}} 
&=
\underbrace{\frac{\mathrm{d}}{\dt} D_{\gruen{1}\dunkelrot{2}}}_{\frac{\partial D_{12}}{\partial \theta_{1}} \theta_{1}' + \frac{\partial D_{12}}{\partial \theta_{2}} \theta_{2}'} \cdot \gruen{\theta_{1}'} + D_{\gruen{1}\dunkelrot{2}} \gruen{\theta_{1}''} + \underbrace{\frac{\mathrm{d}}{\dt} D_{\dunkelrot{22}}}_{\frac{\partial D_{22}}{\partial \theta_{1}} \theta_{1}' + \frac{\partial D_{22}}{\partial \theta_{2}} \theta_{2}'} \cdot \dunkelrot{\theta_{2}'} + D_{\dunkelrot{22}} \dunkelrot{\theta_{2}''} \\
&= 
\frac{\partial D_{12}}{\partial \theta_{1}}  \gruen{\theta_{1}'}^2 
+ \frac{\partial D_{12}}{\partial \theta_{2}} \gruen{\theta_{1}'} \dunkelrot{\theta_{2}'} 
+ D_{\gruen{1}\dunkelrot{2}} \gruen{\theta_{1}''}
+ \frac{\partial D_{22}}{\partial \theta_{1}} \gruen{\theta_{1}'} \dunkelrot{\theta_{2}'} 
+ \frac{\partial D_{22}}{\partial \theta_{2}}  \dunkelrot{\theta_{2}'}^2 
+ D_{\dunkelrot{22}} \dunkelrot{\theta_{2}''}
\end{align*}

Es gilt 
%
\begin{align*}
\frac{\mathrm{d}}{\dt} \frac{\partial T}{\partial \gruen{\theta_{1}'}}  - \frac{\partial T}{\partial \gruen{\theta_{1}}}
&=
\left[ \frac{\partial D_{11}}{\partial \gruen{\theta_{1}}} - \frac{1}{2}  \frac{\partial D_{11}}{\partial \gruen{\theta_{1}}} \right] \gruen{\theta_{1}'}^2
+ \left[ \frac{\partial D_{11}}{\partial \dunkelrot{\theta_{2}}} + \frac{\partial D_{12}}{\partial \gruen{\theta_{1}}} -  \frac{\partial D_{12}}{\partial \gruen{\theta_{1}}} \right] \gruen{\theta_{1}'} \dunkelrot{ \theta_{2}'} 
+ \left[\frac{\partial D_{12}}{\partial \dunkelrot{\theta_{2}}} - \frac{1}{2} \frac{\partial D_{22}}{\partial \gruen{\theta_{1}}} \right] \dunkelrot{\theta_{2}'}^2 \\
& \quad
 +  D_{\gruen{1 1}} \gruen{\theta_{1}''} 
  + D_{\gruen{1}\dunkelrot{2}} \dunkelrot{\theta_{2}''} \\
&=
\frac{1}{2}  \frac{\partial D_{11}}{\partial \gruen{\theta_{1}}}  \gruen{\theta_{1}'}^2
+ \frac{\partial D_{11}}{\partial \dunkelrot{\theta_{2}}} \gruen{\theta_{1}'} \dunkelrot{ \theta_{2}'} 
+ \left[\frac{\partial D_{12}}{\partial \dunkelrot{\theta_{2}}} - \frac{1}{2} \frac{\partial D_{22}}{\partial \gruen{\theta_{1}}} \right] \dunkelrot{\theta_{2}'}^2
 +  D_{\gruen{1 1}} \gruen{\theta_{1}''} 
  + D_{\gruen{1}\dunkelrot{2}} \dunkelrot{\theta_{2}''} \\
&=
\underbrace{\left[ \frac{1}{2}  \frac{\partial D_{11}}{\partial \gruen{\theta_{1}}}  \gruen{\theta_{1}'} + \frac{\partial D_{11}}{\partial \dunkelrot{\theta_{2}}} \dunkelrot{ \theta_{2}'} \right] }_{C_{\gruen{11}}}  \gruen{\theta_{1}'} 
+ 
\underbrace{\left[\frac{\partial D_{12}}{\partial \dunkelrot{\theta_{2}}} \dunkelrot{\theta_{2}'} - \frac{1}{2} \frac{\partial D_{22}}{\partial \gruen{\theta_{1}}} \dunkelrot{\theta_{2}'} \right]}_{C_{\gruen{1}\dunkelrot{2}}} 
 \dunkelrot{\theta_{2}'}
 +  D_{\gruen{1 1}} \gruen{\theta_{1}''} 
  + D_{\gruen{1}\dunkelrot{2}} \dunkelrot{\theta_{2}''} 
\end{align*}

\begin{align*}
\frac{\mathrm{d}}{\dt} \frac{\partial T}{\partial \dunkelrot{\theta_{2}'}} - \frac{\partial T}{\partial \dunkelrot{\theta_{2}}}
&= 
\left[ \frac{\partial D_{12}}{\partial \gruen{\theta_{1}}} - \frac{1}{2} \frac{\partial D_{11}}{\partial \dunkelrot{\theta_{2}}} \right] \gruen{\theta_{1}'}^2 
+ \left[ \frac{\partial D_{12}}{\partial \dunkelrot{\theta_{2}}} + \frac{\partial D_{22}}{\partial \gruen{\theta_{1}}} - \frac{\partial D_{12}}{\partial \dunkelrot{\theta_{2}}} \right] \gruen{\theta_{1}'} \dunkelrot{\theta_{2}'} 
+ \left[ \frac{\partial D_{22}}{\partial \dunkelrot{\theta_{2}}} - \frac{1}{2} \frac{\partial D_{22}}{\partial \dunkelrot{\theta_{2}}} \right]  \dunkelrot{\theta_{2}'}^2 \\
& \quad + D_{\gruen{1}\dunkelrot{2}} \gruen{\theta_{1}''} + D_{\dunkelrot{22}} \dunkelrot{\theta_{2}''} \\
&=
\underbrace{\left[ \frac{\partial D_{12}}{\partial \gruen{\theta_{1}}} \gruen{\theta_{1}'} - \frac{1}{2} \frac{\partial D_{11}}{\partial \dunkelrot{\theta_{2}}} \gruen{\theta_{1}'} \right]}_{C_{\dunkelrot{2}\gruen{1}}} \gruen{\theta_{1}'}
+ \underbrace{\left[ \frac{1}{2} \frac{\partial D_{22}}{\partial \dunkelrot{\theta_{2}}}  \dunkelrot{\theta_{2}'} + \frac{\partial D_{22}}{\partial \gruen{\theta_{1}}} \gruen{\theta_{1}'} \right]}_{C_{\dunkelrot{22}}}  \dunkelrot{\theta_{2}'}
+ D_{\gruen{1}\dunkelrot{2}} \gruen{\theta_{1}''} + D_{\dunkelrot{22}} \dunkelrot{\theta_{2}''}
\end{align*}

Für die Einträge der Matrix $C$ gilt
%
\begin{align*}
C_{11} 
&=
\frac{1}{2}  \underbrace{\frac{\partial D_{11}}{\partial \gruen{\theta_{1}}} }_{= 0} \gruen{\theta_{1}'} + \underbrace{\frac{\partial D_{11}}{\partial \dunkelrot{\theta_{2}}}}_{ =  -2m_2 l_1 a_2 \sin(\dunkelrot{\theta_2})} \dunkelrot{ \theta_{2}'}
=
-2m_2 l_1 a_2 \sin(\dunkelrot{\theta_2}) \dunkelrot{ \theta_{2}'} 
\\
C_{12}
&=
\underbrace{\frac{\partial D_{12}}{\partial \dunkelrot{\theta_{2}}}}_{= - m_2 l_1 a_2 \sin(\dunkelrot{\theta_2})} \dunkelrot{\theta_{2}'} - \frac{1}{2} \underbrace{\frac{\partial D_{22}}{\partial \gruen{\theta_{1}}}}_{ = 0} \dunkelrot{\theta_{2}'}
=
- m_2 l_1 a_2 \sin(\dunkelrot{\theta_2}) \dunkelrot{\theta_{2}'}
\\
C_{21} 
&= \underbrace{\frac{\partial D_{12}}{\partial \gruen{\theta_{1}}}}_{= 0} \gruen{\theta_{1}'} - \frac{1}{2} \underbrace{\frac{\partial D_{11}}{\partial \dunkelrot{\theta_{2}}}}_{= -2m_2 l_1 a_2 \sin(\dunkelrot{\theta_2}) } \gruen{\theta_{1}'}
=
m_2 l_1 a_2 \sin(\dunkelrot{\theta_2})\gruen{\theta_{1}'}
\\
C_{22} 
&=
\frac{1}{2} \underbrace{\frac{\partial D_{22}}{\partial \dunkelrot{\theta_{2}}}}_{ = 0} \dunkelrot{\theta_{2}'} + \underbrace{ \frac{\partial D_{22}}{\partial \gruen{\theta_{1}}}}_{ = 0} \gruen{\theta_{1}'}
= 0
\end{align*}
%

Es gilt somit 
%
\begin{align*}
C \cdot \theta'(t)
=
\begin{pmatrix}
-2m_2 l_1 a_2 \sin(\dunkelrot{\theta_2}) \dunkelrot{ \theta_{2}'} 
&
- m_2 l_1 a_2 \sin(\dunkelrot{\theta_2}) \dunkelrot{\theta_{2}'} 
\\
m_2 l_1 a_2 \sin(\dunkelrot{\theta_2})\gruen{\theta_{1}'}
&
0
\end{pmatrix}
\begin{pmatrix}
\gruen{\theta_{1}'} \\
\dunkelrot{\theta_{2}'} 
\end{pmatrix}
\end{align*}

\subsubsection{Gravitationsterme}

Wählen wir \(y=0\) als Nullhöhe, so ist
%
\begin{align*}
V &= m_1 \cdot g \cdot \gelb{y_1} + m_2 \cdot g \cdot \orange{y_2} \\
&= m_1  g  a_1 \sin(\gruen{\theta_1}) + m_2  g \left( l_1 \sin(\gruen{\theta_1}) + a_2 \sin(\gruen{\theta_1} + \dunkelrot{\theta_2}) \right)
\end{align*}

Für die partiellen Ableitungen erhalten wir
%
\begin{align*}
\frac{\partial V}{\partial \gruen{\theta_1}} &= m_1  g  a_1 \cos(\gruen{\theta_1}) + m_2  g \left( l_1 \cos(\gruen{\theta_1}) + a_2 \cos(\gruen{\theta_1} + \dunkelrot{\theta_2}) \right) \\
\frac{\partial V}{\partial \dunkelrot{\theta_2}} &=  m_2  g \left( l_1 \cos(\gruen{\theta_1}) + a_2 \cos(\gruen{\theta_1} + \dunkelrot{\theta_2}) \right)
\end{align*}


\subsubsection{Simulation gegebener Steuerung}

\begin{itemize}
\item Diskretisierung des Zeitintervalls $[0,1]$ und der Funktion mittels $N \in \mathbb{N}$
%
\begin{align*}
h = \frac{1}{N} \quad \left(\vphantom{\begin{pmatrix} \tau_{1}^{i} \\ \tau_{2}^{i} \end{pmatrix}} u_i \right)_{i=0}^{N} = \begin{pmatrix} \tau_{1}^{i} \\ \tau_{2}^{i} \end{pmatrix}_{i=0}^N
\end{align*}

\item 
%
\begin{align*}
x(t)
= 
\begin{pmatrix}
\gruen{\theta_1} \\
\dunkelrot{\theta_2} \\
\gruen{\theta_1}' \\
\dunkelrot{\theta_2}'
\end{pmatrix}
\quad
u(t)
=
\begin{pmatrix}
\tau_1 \\
\tau_2
\end{pmatrix}
\quad
\begin{pmatrix}
\gruen{\theta_1}'' \\ \dunkelrot{\theta_2}''
\end{pmatrix}
=
D^{-1} \left[ u - C(x) \begin{pmatrix} \gruen{\theta_1}' \\ \dunkelrot{\theta_2}'  \end{pmatrix}  - G(\gruen{\theta_1}, \dunkelrot{\theta_2}) \right]
\end{align*}

\item Dynamisches System
%
\begin{align*}
\underbrace{\begin{pmatrix}
\gruen{\theta_1}' \\
\dunkelrot{\theta_2}' \\
\gruen{\theta_1}'' \\
\dunkelrot{\theta_2}''
\end{pmatrix}}_{x'(t)}
=
\underbrace{\begin{pmatrix}
\gruen{\theta_1}' \\
\dunkelrot{\theta_2}'  \\
D^{-1} \left[ u - C(x) \begin{pmatrix} \gruen{\theta_1}' \\ \dunkelrot{\theta_2}'  \end{pmatrix}  - G(\gruen{\theta_1}, \dunkelrot{\theta_2}) \right]_1 \\
D^{-1} \left[ u - C(x) \begin{pmatrix} \gruen{\theta_1}' \\ \dunkelrot{\theta_2}'  \end{pmatrix}  - G(\gruen{\theta_1}, \dunkelrot{\theta_2}) \right]_2
\end{pmatrix}}_{f(x,u)}
\Rightarrow
x_{i+1} = x_{i} + h \cdot f(x_{i} , u_{i})
\end{align*}
\end{itemize}
\newpage

\section{Optimale Steuerungsprobleme}




\begin{definition}[Optimales Steuerungsproblem] \label{def_OCP}
Gegeben seien
%
\begin{itemize}
    \item Systemdynamik:
	%
	\begin{align*}
	x'(t) = f(x(t), u(t), t), \quad x(t_0) = x_0
	\end{align*}
	%
	wobei $x(t) \in \mathbb{R}^n$ der Zustandsvektor und $u(t) \in U \subset \mathbb{R}^m$ die Steuerung ist. 
    \item Zielfunktional:
	%
	\begin{align*}
	J(u) = \Phi(x(t_f), t_f) + \int_{t_0}^{t_f} L(t,x(t), u(t)) \dt
	\end{align*}
	%
    \item Ziel: Finde die Steuerung $u(t)$, die $J(u)$ minimiert, also
    %
    \begin{align*}
    J(\rot{u^*}) = \min_u J(u) \Rightarrow \rot{u^*} = \argmin_u J(u)
    \end{align*}
\end{itemize}

\textbf{Zielfunktional:}
%
\begin{align*}
\min J[u] = \Phi(t_f,x(t_{f})) + \int_{t_{0}}^{t_{f}} \varphi(t,x,u) \dt
\end{align*}

\textbf{Systemdynamik:}
%
\begin{align*}
x'(t) = f(t,x(t),u(t)) \quad x(t_0) = x_0
\end{align*}


Es gilt mittels Kettenregel:
%
\begin{align*}
\left(\frac{\mathrm{d} \Phi}{\dt} = \Phi_{x} \cdot x' + \Phi_{t} \right)
\end{align*}

\end{definition}


\begin{itemize}
\item Augmentierte Lagrange Funktion
%
\begin{align*}
J(u) = 
\int_{t_{0}}^{t_{f}} \underbrace{\varphi(t,x,u) + \Phi_{x} \cdot x' + \Phi_{t} + p \cdot (x' -f(x,u))}_{= L(t,x,x',u)} \dt 
\end{align*}

\item Für den Kozustand gilt $L_{x'} = p(t)$

\item Optimalitätsbedingung  mit $H(t,x,u,p) = x' \cdot p - L$
%
\begin{align*}
x'(t) = H_p \quad p'(t) = - H_x \quad \rot{\frac{\partial L}{\partial u} = 0} 
\end{align*}

\item Diese Bedingungen geben uns die notwendigen Optimalitätsbedingungen für das optimale Steuerungsproblem
\end{itemize}

\begin{satz}[Minimumprinzip] \label{MP}
Sei  $\textbf{H} = L(t,x,u) + p(t)^T \cdot f(t,x,u)$ die Hamilton-Funktion
\begin{itemize}
\item  Es gilt $u^{*} = \textsf{argmin } \textbf{H}(x^{*},u,p^{*})$  
%
\begin{align*}
p'(t) = - H_x   \quad \rot{\textbf{H}_u = 0} \quad x'(t) = H_p
\end{align*}

\item Weiter gilt die Transversalitätsbedingung 
%
\begin{align*}
 \Phi_{x}(t=t_f) = p(t=t_f) 
\end{align*}
%
\item Für freie Endzeit $t_f$ gilt 
%
\begin{align*}
\textbf{H}(t=t_f) + \Phi_t (t=t_f) = 0 \Rightarrow \underbrace{\textbf{H}(x,u,p) = 0}_{ \textsf{für } \textbf{H}_t = 0}
\end{align*}

\end{itemize}
\end{satz}

\begin{beispiel}[Testbeispiel] \label{Testbeispiel}
Löse das folgende optimale Steuerungsproblem
%
\begin{align*}
\min_u \int_0^{1} x(t) + u(t)^2 \dt \quad x'(t) = x(t) + u(t) + 1, \quad x(0) = 0.
\end{align*}

Wir definieren die Hamiltonfunktion $\textbf{H} = x +u^2 + p \cdot (x+u +1)$. Das Minimumprinzip liefert
%
\begin{align*}
p'(t) = -\textbf{H}_x = -1-p \quad \textbf{H}_u = 0 = 2u + p \Rightarrow u = - \frac{p}{2}
\end{align*}

Die adjungierte Gleichung ist trennbar 
%
\begin{align*}
\frac{\mathrm{d}p}{\dt} = -1 - p \Rightarrow \underbrace{\int \frac{\mathrm{d}p}{1+p}}_{\log(p+1)} = \underbrace{\int -1 \dt}_{= -t + K} \Rightarrow p(t) = -1 + C \cdot e^{-t}, \quad C = e^K
\end{align*}

Für die optimale Steuerung gilt somit $u(t) = - \frac{1}{2} p(t) = \frac{1}{2} - \frac{C}{2} \cdot e^{-t}$. Den optimalen Zustand berechnen wir aus der Systemdynamik
%
\begin{align*}
x'(t) =  x(t) + \underbrace{\frac{1}{2} - \frac{C}{2} \cdot e^{-t}}_{= u(t)} + 1 \Rightarrow x'(t) = \rot{1} \cdot x(t) + \blau{\frac{3}{2} - \frac{C}{2} e^{-t}}
\end{align*}

Das ist eine lineare inhomogene Gleichung. Es gilt $A(t) =\int_0^t \rot{1} \ds = t$ und somit
%
\begin{align*}
x(t) 
&= x(0) \cdot e^{\rot{A(t)}} + e^{\rot{A(t)}} \cdot \int_0^t \left( \blau{\frac{3}{2} - \frac{C}{2} e^{-s}} \right) \cdot e^{\rot{-A(s)}} \ds \\
&= e^{t} \cdot \int_0^t \frac{3}{2} \cdot e^{-s} - \frac{C}{2} e^{-s} \cdot e^{-s} \ds \\
&= e^t \cdot \left[ -\frac{3}{2} e^{-s} + \frac{C}{4} e^{-2s} \right]_0^t \\
&=- \frac{3}{2} + \frac{C}{4} e^{-t} + \frac{3}{2} e^{t} - e^{t} \frac{C}{4}
\end{align*}

Da $x(1)$ frei ist und kein Endkostenterm vorliegt gilt $p(1) = 0$. 
%
\begin{align*}
p(1) = -1 + C \cdot e^{-1} = 0 \Rightarrow C = e. 
\end{align*}

Wir fassen nun die Resultate zusammen:
%
\begin{align*}
p(t) &= -1 + e^{-t+1} \\
u(t) &=  \frac{1}{2} \cdot \left(1- e^{-t+1} \right) \\
x(t) &= - \frac{3}{2} + \frac{ e^{-t+1}}{4} + \frac{3}{2} e^{t} - \frac{e^{t+1}}{4}  
\end{align*}
\end{beispiel}

%
\begin{beispiel}[Minimierung des Treibstoffverbrauchs] \label{Min_Treibstoff_fixed}
Ein Fahrzeug startet bei $x(0) = 0$ mit $v(0) = 0$ und soll zum festen Zeitpunkt $t_f = 2$ die Position $x(2) = 1$ erreichen. Das Kostenfunktional lautet:
%
\begin{align*}
J[u] = \int_0^{2}  \frac{1}{2} u^2(t)   dt
\end{align*}

Die Dynamik des Fahrzeugs ist:
%
\begin{align*}
x'(t) = v(t), \quad v'(t) = u(t) \Rightarrow  
\frac{\mathrm{d}}{\dt} 
\orange{\begin{pmatrix}
x(t) \\
v(t)
\end{pmatrix}}
=
\dunkelrot{
\begin{pmatrix}
v(t) \\
u(t)
\end{pmatrix}}
\Rightarrow
\frac{\mathrm{d}}{\dt} \orange{y(t)} = \dunkelrot{f(y,u)}
\end{align*}

\begin{itemize}
\item Aufstellen der Hamilton-Funktion mit $p(t) = \begin{pmatrix} p_x \\ p_v \end{pmatrix}$
%
\begin{align*}
H = L + \textbf{p}(t)^T \cdot \textbf{f}(\textbf{y},u)  = \frac{1}{2} u(t)^2  + p_x (t) v(t) + p_v (t) u(t) 
\end{align*}

\item Formulierung des Minimumprinzips:
%
\begin{align*}
\frac{\partial H}{\partial u} = 0 \Rightarrow \gruen{-p_v(t) = u^*(t)}, \quad -p'(t) = \nabla_{\orange{y = (x,v)}} H \Rightarrow -p_{x}'(t) = 0,  -p_{v}'(t) = p_x (t)
\end{align*}

\item Integration der Gleichungen 
%
\begin{align*}
p_{x}(t) = C  ,  \gruen{p_{v}(t) = -C \cdot t + K}
\end{align*}

\item Integration der Systemdynamik unter Nutzung der Anfangswerte $\rot{x(0)=v(0) = 0}$
%
\begin{align*}
\gruen{u(t)} &\gruen{= C t - k } \\
v(t) + \rot{v(0)} &= \int_0^t \orange{v'(s)} \ds = \int_0^t \orange{u(s)} \ds =  \frac{C}{2} t^2 - k t  \\
x(t) + \rot{x(0)} &= \int_0^t \blau{x'(s)} \ds = \int_0^t \blau{v(s)} \ds = \frac{C}{6} t^3 - \dfrac{k}{2} t^2 
\end{align*}

\item Anwendung der Endwerte $x(2) = 1$ und $v(2) = 0$
%
\begin{align*}
0 &= \frac{C}{2} \cdot 4 - k \cdot 2 \Rightarrow k = C \\
1 &= \frac{C}{6} \cdot 8 - \frac{C}{2} \cdot 4 \Rightarrow 1 = \left( \frac{4C}{3} - 2C \right) = -\frac{2C}{3} \Rightarrow C = -\frac{3}{2} \\
u(t) &= \left( -\frac{3}{2} \right) t - \left( -\frac{3}{2} \right) = -\frac{3}{2} t + \frac{3}{2}
\end{align*}
\end{itemize}
\end{beispiel}

\subsection{Linear-Quadratische Probleme}

\begin{satz}[Riccati-Regler] \label{LQR_Riccati}
Die Lösung des linear-quadratischen Problems
%
\begin{align*}
\min J(u) &=  \int_0^T   \frac{q}{2} \cdot x^2(t) + \frac{r}{2} \cdot u^2(t) dt + \frac{s}{2} x(T)^2 \\
\textsf{s.t.} \quad x'(t) &= a \cdot x(t) + b \cdot u(t), \quad x(0) = x_0
\end{align*}

mit $a,b \in \mathbb{R}$ und $q,r,s \geq 0$ ist gegeben durch die Riccati-Gleichung
%
\begin{align*}
K'(t) = - 2a K(t)  + \frac{b^2}{r} K(t)^2  - q, \quad K(T) = s
\end{align*}

und die resultierenden Lösungen
%
\begin{align*}
x^*(t) = x_0 \exp\left( \int_0^t a - \frac{b^2}{r} K(s) \ds \right) \quad u^*(t) = -\frac{b}{r} K(t) x(t) 
\end{align*}
\end{satz}


\proof{ 
Wir definieren die Hamilton-Funktion:

\begin{align*}
H = \frac{1}{2} q x(t)^2 + \frac{1}{2} r u(t)^2 + p(t) \left( a x(t) + b u(t) \right)
\end{align*}

Dann liefert das Minimumprinzip die notwendigen Bedingungen

\begin{align*}
\frac{\partial H}{\partial u} &= 0 \Rightarrow u^*(t) = -\frac{b}{r} p(t) \quad
p'(t) = -\frac{\partial H}{\partial x} \Rightarrow p'(t) = - q x(t) - a p(t)
\end{align*}

Weiter gilt die Transversalitätsbedingung $p(T) = s \cdot x(T)$. Wir nehmen an, dass eine Feedbacksteuerung vorliegt, d.h. 

\begin{align*}
p(t) = K(t) x(t) \quad p(T) = K(T)x(T) \Rightarrow K(T)= s
\end{align*}

wobei $K(t)$ eine skalare Funktion ist. Ableiten von $p$ nach $t$ liefert

\begin{align*}
p'(t) = K'(t) x(t) + K(t)x'(t)
\end{align*}

Für die Zustandsgleichung gilt mit $u^*(t) = -\frac{b}{r} p(t)$

\begin{align*}
x'(t) = a x(t) + b u^*(t) = a x(t) - \frac{b^2}{r} K(t) x(t) = \left( a - \frac{b^2}{r} K(t) \right) x(t).
\end{align*}

Wir setzen nun $p'(t) = -H_x$ und unseren Ausdruck für $x'(t)$ ein, so gilt

\begin{align*}
\underbrace{- q x(t) - a p(t)}_{= p'(t)} &= K'(t) x(t) + K(t) \underbrace{\left( a - \frac{b^2}{r} K(t) \right) x(t)}_{= x'(t)} \\
\Leftrightarrow - q x(t) - a K(t) x(t) &= K'(t) x(t) + a K(t) x(t) - \frac{b^2}{r} K(t)^2 x(t) \\
\Leftrightarrow K'(t) x(t) &= - 2a K(t) x(t) + \frac{b^2}{r} K(t)^2 x(t) - q x(t)
\end{align*}

Unter der Annahme $x(t) \neq 0$ erhalten wir die Riccati-Differentialgleichung:

\begin{align*}
K'(t) = - 2a K(t)  + \frac{b^2}{r} K(t)^2  - q, \quad K(T) = s.
\end{align*}

Wir erhalten $x(t)$ als Lösung der folgenden Differentialgleichung

\begin{align*}
x'(t) =  \left( a - \frac{b^2}{r} K(t) \right) x(t) \Rightarrow x(t) = x(0) \cdot \exp\left( \int_0^t a - \frac{b^2}{r} K(s) \ds \right).
\end{align*}

Die optimale Steuerung $u^*(t)$ ergibt sich aus $u^*(t) = -\frac{b}{r} p(t) = -\frac{b}{r} K(t)x(t)$.
}


\begin{beispiel}[Berechnung des Riccati-Reglers] \label{LQR_ex2}
Wir betrachten erneut ein Beispiel. In diesem Beispiel gilt für die Parameter

\begin{align*}
a = 0, b = 1, q =1 , r= 1, T=1, s =0, x_0 = 1.
\end{align*}

Die Riccati-Gleichung lautet somit

\begin{align*}
K'(t) = K(t)^2 - 1, \quad K(1) = 0
\end{align*}

Die Gleichung ist trennbar und es gilt

\begin{align*}
\frac{\mathrm{d}K}{\dt} = K^2 - 1 \Rightarrow \int \frac{1}{K^2 - 1} \mathrm{d}K  = \int 1 \dt
\end{align*}

Wir nutzen die Partialbruchzerlegung für $\frac{1}{K^2 -1}$ und erhalten

\begin{align*}
\frac{1}{K^2 -1} = \frac{1}{(K+1)(K-1)} = \frac{A}{K-1} + \frac{B}{K+1} \Rightarrow 1 = \underbrace{A (K+1) + B(K-1) }_{K \cdot (A+B) + A -B}
\end{align*}

Der Koeffizientenvergleich liefert $A+B = 0$ und $A-B =1$, mit der Lösung $A = \frac{1}{2}$ und $B= -\frac{1}{2}$. Wir erhalten für unser Integral

\begin{align*}
\int \frac{1}{K^2 - 1} \mathrm{d}K  &= \int 1 \dt \\
\int \frac{1}{2} \frac{1}{K-1} - \frac{1}{2} \frac{1}{K+1} \mathrm{d}K  &= t + \frac{C}{2} \\
\frac{1}{2} \cdot \left(\underbrace{ \log(K-1) - \log(K+1)}_{\log\left( \frac{K-1}{K+1} \right)} \right) &= t+ \frac{C}{2} \\
\frac{K-1}{K+1} &= \exp\left(2t + C \right) = \underbrace{e^C}_{:= A} \cdot e^{2t} \\
K(t) &= \frac{1+ Ae^{2t}}{1- Ae^{2t}} 
\end{align*}

Es gilt weiter $K(1) = 0$ und somit $A = -\frac{1}{e^2}$ damit ist die Lösung der Riccati-Gleichung gegeben durch

\begin{align*}
K(t) = \frac{1- e^{2(t-1)}}{1+e^{2(t-1)}}.
\end{align*}

Für den optimalen Zustand gilt 

\begin{align*}
x(t) = 1 \cdot \exp\left( \int_0^t -K(s) \ds \right) = \exp\left( \int_0^t -\frac{1- e^{2(s-1)}}{1+e^{2(s-1)}} \ds \right)
\end{align*}

Wir nutzen die Substitution $z = e^{2(s-1)}$ dann gilt für das Integral 

\begin{align*}
\frac{\dz}{\ds} = 2 \cdot \underbrace{e^{2(s-1)}}_{ = z} \Rightarrow \int - \frac{1-z}{1+z} \cdot \frac{\dz}{2z} = -\frac{1}{2} \int \frac{1-z}{z(1+z)} \dz
\end{align*}

Wir verwenden erneut eine Partialbruchzerlegung

\begin{align*}
 \frac{1-z}{z(1+z)} = \frac{A}{z} + \frac{B}{z+1} \Rightarrow 1-z = \underbrace{A(1+z) + Bz}_{= (A+B)z + A}
\end{align*}

Ein Koeffizientenvergleich liefert $A= 1$ und $A+B = -1$ und somit $B= -2$. Für das Integral gilt somit

\begin{align*}
-\frac{1}{2} \int \frac{1-z}{z(1+z)} \dz 
&= -\frac{1}{2} \int  \frac{1}{z} - \frac{2}{z+1} \dz = -\frac{1}{2} \left[ \log(z) -2\log(z+1) \right] + C
\end{align*}

Rücksubstitution von $z(t) = e^{2(t-1)}$ liefert

\begin{align*}
x(t) = \exp\left[ -(t-1) + \log(e^{2(t-1)}+1)  + C \right]
\end{align*}

Wir nutzen $x(0) = 1$ und bestimmen die Konstante $C$. 

\begin{align*}
x(0) &= \exp\left( -(0-1) + \log(e^{2(0-1)}+1)  + C \right) = 1 
\end{align*}

Wir erhalten somit die Bedingung

\begin{align*}
-(0-1) + \log(e^{2(0-1)}+1)  + C = 0 \Rightarrow C = -1 - \log(1+e^{-2})
\end{align*}

Der optimale Zustand ist somit gegeben als 

\begin{align*}
x(t) = \exp\left[ -t+2 + \log\left(\frac{1+e^{2(t-1)}}{1+e^{-2}}\right) \right] = \underbrace{e^{-t+2} \cdot \frac{1+ e^{2(t-1)}}{1+e^{-2}}}_{= \frac{e^2}{1+e^{-2}} e^{-t} + \frac{1}{1+e^{-2}} e^{t}}
\end{align*}

Für die optimale Steuerung gilt $u^*(t)= -K(t) x(t)$ und somit
% \bcancel{\textcolor{red!50}{1+ e^{2(t-1)}}}
\begin{align*}
u(t) &=  - \frac{1- e^{2(t-1)}}{\rot{1+e^{2(t-1)}}} \cdot e^{-t+2} \cdot \frac{\rot{1+ e^{2(t-1)}}}{1+e^{-2}} \\
&=  \frac{e^{-t+2}}{1+e^{-2}} \cdot (-1 + e^{2(t-1)}) \\
&= - \frac{e^2}{1+e^{-2}} e^{-t} + \frac{1}{1+e^{-2}} e^{t} 
\end{align*}

Dies entspricht den Lösungen aus dem Beispiel.
\end{beispiel}


\newpage

\section{Numerische Lösungsverfahren}


\subsection{Gradientenverfahren}

\begin{align*}
u(t) + \epsilon \cdot \eta_u (t) \quad x(t) + \epsilon \cdot \eta_x (t), \eta_x (0) = 0
\end{align*}

Um die Zustandsgleichung als Nebenbedingung zu integrieren, führen wir einen Lagrange-Multiplikator \( p(t) \) ein. Das erweiterte Zielfunktional lautet dann:

\[
J^{\text{aux}}(u, x, p) = \int_{0}^{T} \left[ L(x(t), u(t), t) + p(t) \left( f(x(t), u(t)) - \dot{x}(t) \right) \right] \mathrm{d}t + \Phi(x(T))
\]


Wir betrachten kleine Variationen der Steuerungsfunktion und des Zustands:



wobei \( \varepsilon \) ein kleiner Parameter ist.



Die Variation von \( J^{\text{aux}} \) bezüglich \( \varepsilon \) bei \( \varepsilon = 0 \) ergibt:

\[
\delta J^{\text{aux}} = \int_{0}^{T} \left[ \frac{\partial L}{\partial x} \delta x(t) + \frac{\partial L}{\partial u} \delta u(t) + p(t) \left( \frac{\partial f}{\partial x} \delta x(t) + \frac{\partial f}{\partial u} \delta u(t) - \delta \dot{x}(t) \right) \right] \mathrm{d}t + \frac{\partial \Phi}{\partial x}(x(T)) \delta x(T) = 0
\]


Der Term mit \( \delta \dot{x}(t) \) wird mittels Integration durch Teile umgeformt:

\[
\int_{0}^{T} p(t) (-\delta \dot{x}(t)) \, \mathrm{d}t = \left[ -p(t) \delta x(t) \right]_{0}^{T} + \int_{0}^{T} \dot{p}(t) \delta x(t) \, \mathrm{d}t
\]

Da \( \delta x(0) = 0 \) ist (weil der Anfangszustand festgelegt ist), vereinfacht sich der Randterm zu:

\[
-p(T) \delta x(T) + \int_{0}^{T} \dot{p}(t) \delta x(t) \, \mathrm{d}t
\]


Setzen wir die Ergebnisse in die Variation des erweiterten Zielfunktionals ein:

\[
\int_{0}^{T} \left[ \left( \frac{\partial L}{\partial x} + p(t) \frac{\partial f}{\partial x} + \dot{p}(t) \right) \delta x(t) + \left( \frac{\partial L}{\partial u} + p(t) \frac{\partial f}{\partial u} \right) \delta u(t) \right] \mathrm{d}t + \left( \frac{\partial \Phi}{\partial x} - p(T) \right) \delta x(T) = 0
\]

Für die Gleichung muss jeder Koeffizient der unabhängigen Variation Null sein. Daher ergeben sich die adjungierten Gleichungen:

\[
\begin{cases}
\dot{p}(t) = -\frac{\partial L}{\partial x} - p(t) \frac{\partial f}{\partial x} \\
p(T) = \frac{\partial \Phi}{\partial x}(x(T))
\end{cases}
\]



Der Gradient des Zielfunktionals \( J(u) \) ist gegeben durch:

\[
\frac{\delta J}{\delta u(t)} = \frac{\partial L}{\partial u} + p(t) \frac{\partial f}{\partial u}
\]

\subsubsection{Sensitivity Approach}

Erweitertes Zielfunktional

\begin{align*}
J^{aux}(u) 	&= \int_0^T \varphi(t,x,u) + p(t) \cdot (f(t,x,u) - x'(t)) \dt + \Phi(x(T)) \\
			&= \int_0^T \underbrace{\varphi(t,x,u) + p(t) \cdot f(t,x,u)}_{= H(t)} \dt \underbrace{-  \int_0^T p(t) x'(t) \dt}_{ = - \left[p(t) x(t) \right]_0^T + \int_0^T p'(t) x(t) \dt} + \Phi(x(T)) \\
			&= \int_0^T H(t) + p'(t)x(t) \dt - \left[p(t) x(t) \right]_0^T + \Phi(x(T)) \\
			&= \int_0^T H(t) + p'(t)x(t) \dt - p(T) x(T) +  \gelb{p(0) x(0)} + \Phi(x(T))
\end{align*}

Der Term $\gelb{p(0) x(0)}$ ist unabhängig von $u$ und fällt bei der Optimierung weg. Bilden der Gateau Ableitung in $u$ entlang $h$ liefert  

\begin{align*}
J'(u,h) &= \int_0^T H_x [t] \cdot S(t)  + H_u [t] \cdot h(t)  + p'(t)S(t)  \dt - p(T) S(T)  + \Phi_x (x(T)) S(1) \\
		&= \int_0^T ( \rot{\underbrace{H_x [t] + p'(t)}_{ = 0}}) \cdot S(t)  + H_u [t] \cdot h(t)  \dt  + ( \lila{\underbrace{\Phi_x (x(T))- p(T)}_{= 0}} ) \cdot S(T)  \\
		&= \int_0^T H_u [t] \cdot h(t)
\end{align*}

Wir erhalten somit $J'(u) = H_u [t]$

\newpage

\subsubsection{Praktische Implementierung}

\begin{beispiel} \label{ex_gradient_1}
Wir betrachten das Steuerungsproblem 
%
\begin{align*}
\min_u \int_0^{1} x(t) + u(t)^2 \dt \quad x'(t) = x(t) + u(t) + 1, \quad \orange{x(0) = 0}.
\end{align*}

welches in einer Aufgabe analytisch gelöst ist. Für den Gradienten gilt
%
\begin{align*}
\nabla J(u) = H_u = \frac{\partial}{\partial u} ( x(t) + u(t)^2 + p(t) \cdot  (x(t) + u(t) + 1) ) = 2u(t) +p(t)
\end{align*} 

Wir verwenden das explizite Euler-Verfahren zur Diskretisierung der Gleichungen aus dem Minimumprinzip: 
%
\begin{itemize}
\item Diskretisierung des Zeitintervalls $[0,1]$ und der Funktionen mittels $\blau{N} \in \mathbb{N}$
%
\begin{align*}
h = \frac{1}{N} \quad \textbf{u}^k = \left(u_i^k \right)_{i=0}^{N}, \quad  \textbf{x}^k = \left( x^{k}_{i} \right)_{i=0}^{N}, \quad \textbf{t}^k = \left( i \cdot h\right)_{i=0}^{N},
\end{align*} 


\item Armijo Suchrichtung $\gruen{\alpha^{(k)}}$:  Für $0 \leq m < \blau{M=10}$,  $\blau{c= 10^{-4},\beta  = 0.5} \in (0,1)$
%
\begin{align*}
u^{\textsf{neu}} 	&= u^k + \beta^m  \cdot d^{(k)} \\
\rot{J(u^k)} 				&= \int_0^1 x^{k}(t) + (u^{k}(t))^2 \dt \rot{\approx h \cdot \sum_{i=0}^{N} x^{k}_{i} + (u^{k}_{i})^2}\\
\rot{J(u^{neu})}			&= \int_0^1 x^{neu}(t) + (u^{neu}(t))^2 \dt \rot{\approx h \cdot \sum_{i=0}^{N} x^{neu}_{i} + (u^{neu}_{i})^2} \\
		J(u^{neu}) &\leq J(u^{k}) \dunkelrot{-} c \cdot \beta^m \cdot \dunkelrot{h \cdot \sum_{i=0}^{N} (2 \cdot u^{(k)}_{i}  + p^{(k)}_{i})(2 \cdot u^{(k)}_{i}  + p^{(k)}_{i})} \\
\Leftrightarrow \rot{J(u^{neu})}		&\leq \rot{J(u^{k})}  \dunkelrot{- \norm{d^{(k)})}^2} \cdot c \cdot \beta^m \Rightarrow \gruen{\alpha^{(k)} = \beta^m} 
\end{align*}

\item Gradientenverfahren: Für $0 \leq \dunkelgelb{k} < MaxIter$, $\blau{\epsilon = 10^{-6}}$:
%
	\begin{itemize}
	\item Forward Integration $x'(t) = x(t) +u(t) +1$
	%
	\begin{align*}
	x^{\dunkelgelb{k}}_{i+1} = x^{\dunkelgelb{k}}_{i} + h \cdot (x^{(\dunkelgelb{k})}_{i} + u^{(\dunkelgelb{k})}_{i} +1) \quad \orange{x^{(k)}_{0} = 0}
	\end{align*}

	\item Backward Integration $p'(t) = -( 1 + p(t) )$
	%
	\begin{align*}
	p^{\dunkelgelb{k}}_{i} = p^{\dunkelgelb{k}}_{i+1} + h \cdot (1 + p^{\dunkelgelb{k}}_{i+1}) \quad \orange{p^{(k)}_{N} = \partial_x \Phi(x(1)) = 0}
	\end{align*}

	\item Berechnung des Gradienten $\nabla J = H_u =2u + p$ 
	%
	\begin{align*}
	d^{(\dunkelgelb{k})}  = -(2 \cdot u^{(\dunkelgelb{k})}  + p^{(\dunkelgelb{k})})
	\end{align*}
	
	\item Berechnung der Schrittweite $\gruen{\alpha^{(k)}}$:
	%
	\begin{align*}
	u^{(\dunkelgelb{k+1})} &= u^{(\dunkelgelb{k})} + \gruen{\alpha^{(k)}} \cdot d^{(\dunkelgelb{k})}  \\
	\norm{u^{(\dunkelgelb{k+1})} - u^{(\dunkelgelb{k})}} &< \blau{\epsilon} \Rightarrow u^{\rot{\textsf{final}}} = u^{(\dunkelgelb{k+1})},  \textsf{ sonst } \dunkelgelb{k \to k+1}
	\end{align*}
	%
	\end{itemize}
%
\end{itemize}


%\includegraphics[width=\textwidth]{ex3.1.plot}
\end{beispiel}

\begin{figure}[h] % H für Positionierung der Grafik
    \centering
    \includegraphics[width=\textwidth]{ex3.1.plot.pdf}
    \captionsetup{labelformat=empty}  % Entfernt "Abbildung 1"
    \caption{Zustand $x$ und Steuerung $u$ mit den Iterationen $\blau{100},\orange{500}$ und der \rot{analytischen Lösung}}
    \label{fig:plot} % Referenz, falls du auf die Grafik verweisen möchtest
\end{figure}

\newpage


\subsection{Shooting Methods}

Das Ziel ist es, den Anfangswert \( p(t_0) \) so anzupassen, dass die Endbedingung \( p(t_f) = \frac{\partial \Phi}{\partial x}(x(t_f)) \) erfüllt wird. Dieses Verfahren wird mithilfe des Schießverfahrens gelöst. Wir definieren eine Residuenfunktion \( R(p(t_0)) \), die die Abweichung zwischen dem berechneten Endwert \( p(t_f; p(t_0)) \) und dem gewünschten Endwert \( \frac{\partial \Phi}{\partial x}(x(t_f)) \) darstellt:

\begin{equation}
    R(p(t_0)) = p(t_f; p(t_0)) - \frac{\partial \Phi}{\partial x}(x(t_f; p(t_0)))
\end{equation}

Das Newton-Verfahren zur Lösung von \( R(p(t_0)) = 0 \) liefert den Update-Schritt:

\begin{equation}
    p^{(k+1)}(t_0) = p^{(k)}(t_0) - \frac{R(p^{(k)}(t_0))}{R'(p^{(k)}(t_0))}
\end{equation}

Die Jacobi-Matrix \( R'(p(t_0)) \) ergibt sich durch Ableiten der Residuenfunktion:

\begin{equation}
    R'(p(t_0)) = \frac{\partial R}{\partial p(t_0)} = S_p(t_f) - \frac{\partial^2 \Phi}{\partial x^2} S_x(t_f),
\end{equation}

wobei \( S_x(t) = \frac{\partial x(t)}{\partial p(t_0)} \) und \( S_p(t) = \frac{\partial p(t)}{\partial p(t_0)} \) die Sensitivitäten sind.

Wir leiten die Zustandsgleichung und die adjungierte Gleichung nach \( p(t_0) \) ab, um die Sensitivitäten \( S_x(t) \) und \( S_p(t) \) zu berechnen.

\begin{itemize}
\item Durch Ableiten der Zustandsgleichung nach \( p(t_0) \) erhalten wir die Gleichung für \( S_x(t) \):

\begin{equation}
    \frac{d}{dt} S_x(t) = \frac{\partial f}{\partial x} S_x(t) + \frac{\partial f}{\partial u} S_u(t),
\end{equation}

mit der Anfangsbedingung \( S_x(t_0) = 0 \).


\item Durch Ableiten der adjungierten Gleichung nach \( p(t_0) \) erhalten wir die Gleichung für \( S_p(t) \):

\begin{equation}
    \frac{d}{dt} S_p(t) = -\frac{\partial^2 H}{\partial x^2} S_x(t) - \frac{\partial^2 H}{\partial x \partial u} S_u(t),
\end{equation}

mit der Anfangsbedingung \( S_p(t_0) = I \), wobei \( I \) die Einheitsmatrix ist.

\item Falls die Steuerung \( u \) implizit von \( p(t_0) \) abhängt, bestimmen wir \( S_u(t) = \frac{\partial u}{\partial p(t_0)} \) aus der Optimalitätsbedingung \( \frac{\partial H}{\partial u} = 0 \):

\begin{equation}
    \frac{\partial^2 H}{\partial u \partial x} S_x(t) + \frac{\partial^2 H}{\partial u^2} S_u(t) + \frac{\partial^2 H}{\partial u \partial p} S_p(t) = 0
\end{equation}

Diese Gleichung kann nach \( S_u(t) \) aufgelöst werden, sofern \( \frac{\partial^2 H}{\partial u^2} \) invertierbar ist:

\begin{equation}
    S_u(t) = -\left( \frac{\partial^2 H}{\partial u^2} \right)^{-1} \left( \frac{\partial^2 H}{\partial u \partial x} S_x(t) + \frac{\partial f}{\partial u}^\top S_p(t) \right)
\end{equation}
\end{itemize}


% Beispiel aus Gerdts p.198
\begin{beispiel}
Betrachten wir das folgende konkrete Beispiel:

\begin{align*}
\min & \frac{1}{2} (x(1) - 1)^2 + \frac{1}{2} \int_0^1 \left( u(t)^2 + x(t)^3 \right) \, dt \\
\textsf{s.t }  x'(t) &=   u(t) - 15 \exp(-2t), \quad x(0) = 4
\end{align*}

Der adjungierte Zustand ist gegeben durch

\begin{align*}
p'(t) &= -\frac{3}{2} x(t)^2 \quad p(1) = 5(x(1) - 1)
\end{align*}



\begin{itemize}
\item Resultierende Sensitivitätsgleichungen für \( S_x \) und \( S_p \)

In diesem Beispiel ergibt sich die Steuerung \( u(t) = -p(t) \) aus der Optimalitätsbedingung. Durch Einsetzen in die Zustandsgleichung erhalten wir das gekoppelte System:

\begin{align*}
    \dot{x}(t) &= -p(t) - 15 \exp(-2t), \quad x(0) = 4 \\
    \dot{p}(t) &= -\frac{3}{2} x(t)^2, \quad p(1) = 5(x(1) - 1)
\end{align*}

Die Sensitivitätsgleichungen für \( S_x(t) \) und \( S_p(t) \) lauten:

\begin{align*}
    \frac{d}{dt} S_x(t) &= -S_p(t), \quad S_x(0) = 0 \\
    \frac{d}{dt} S_p(t) &= -3 x(t) S_x(t), \quad S_p(0) = 1
\end{align*}

\item Die Jacobi-Matrix ergibt sich zu:

\begin{equation}
    R'(p(0)) = S_p(1) - 5 \cdot S_x(1)
\end{equation}
\end{itemize}



Das Newton-Verfahren zur Anpassung von \( p(0) \) sieht folgendermaßen aus:

\begin{enumerate}
    \item \textbf{Initialisierung}: Wählen Sie eine Startschätzung \( p^{(0)}(0) \).
    
    \item \textbf{Iterative Berechnung}:
    
    \begin{enumerate}
        \item \textbf{Vorwärtsintegration der Zustandsgleichungen}: Integrieren Sie die Gleichungen für \( x(t) \) und \( p(t) \) mit \( p^{(k)}(0) \) von \( t = 0 \) bis \( t = 1 \).
        
        \item \textbf{Berechnung des Residuals}: Berechnen Sie das Residuum
        \[
        R(p^{(k)}(0)) = p(1; p^{(k)}(0)) - 5(x(1; p^{(k)}(0)) - 1)
        \]
        
        \item \textbf{Integration der Sensitivitätsgleichungen}: Integrieren Sie die Sensitivitätsgleichungen für \( S_x(t) \) und \( S_p(t) \) von \( t = 0 \) bis \( t = 1 \), um \( S_x(1) \) und \( S_p(1) \) zu berechnen.
        
        \item \textbf{Berechnung der Ableitung \( R'(p^{(k)}(0)) \)}:
        \[
        R'(p^{(k)}(0)) = S_p(1) - 5 \cdot S_x(1)
        \]
        
        \item \textbf{Aktualisierung mittels Newton-Schritt}:
        \[
        p^{(k+1)}(0) = p^{(k)}(0) - \frac{R(p^{(k)}(0))}{R'(p^{(k)}(0))}
        \]
        
        \item \textbf{Konvergenzkriterium}: Falls \( |R(p^{(k+1)}(0))| < \epsilon \) (Toleranz), beenden Sie die Iteration. Falls nicht, setzen Sie \( k = k + 1 \) und wiederholen ab Schritt (a).
    \end{enumerate}
\end{enumerate}
\end{beispiel}

\subsection{Riccati Regler}

Es gilt 
%
\begin{align*}
F(K) := Q - KPK + KA + A^T K \Rightarrow F(K) = 0
\end{align*}

% Wichtige Grundbegriffe aus LA
%----------------------------------------------------------------------------
\begin{definition}
Sei $X \in \mathbb{R}^{n \times n}$ dann definieren wir 
%
\begin{align*}
\textsf{vec}(X) = 
\begin{pmatrix}
X_{11} & X_{21} & \cdots & X_{n1} & X_{12} & X_{22} & \cdots X_{nn}  
\end{pmatrix}^T 
\in 
\mathbb{R}^{n^2}
\end{align*}
%
Seien $A \in \mathbb{R}^{n \times m}, B \in \mathbb{R}^{p \times r}$, dann definieren wir 
%
\begin{align*}
A \otimes B =
\begin{pmatrix}
A_{11} B 	& \cdots 	& A_{1m} B 	\\
A_{12} B 	& \hdots 	& A_{2m} B	\\
\vdots		& \ddots	& \vdots	\\
A_{n1} B	& \cdots	& A_{nm} B
\end{pmatrix}
\in 
\mathbb{R}^{mp \times nr}
\end{align*}
\end{definition}
%----------------------------------------------------------------------------

Es gilt die Formel
%
\begin{align*}
\textsf{vec}(MXN) = (N^T \otimes M) \textsf{vec}(X)  
\end{align*}

Betrachten wir den Fall $M= 1$ und $N=A$ bzw. $N = 1$ und $M=A^T$ erhalten wir mit $X=K$ 
%
\begin{align*}
\textsf{vec}(KA)= (A^T \otimes 1) \textsf{vec}(K) \quad \textsf{vec}(A^T K) = (1 \otimes A^T) \textsf{vec}(K)
\end{align*}

Anwendung auf unser Problem liefert
%
\begin{align*}
\underbrace{\textsf{vec}(F(K))}_{f(x = \textsf{vec}(K))} = \underbrace{\textsf{vec}(Q)}_{q} - \underbrace{\textsf{vec}(KPK)}_{g(x = \textsf{vec}(K))} + \underbrace{\textsf{vec}(KA)}_{(A^T \otimes 1) \textsf{vec}(K)} + \underbrace{\textsf{vec}(A^T K)}_{(1 \otimes A^T) \textsf{vec}(K)}
\end{align*}

Durch Einführung von $x = \textsf{vec}(K) \in \mathbb{R}^{n^2}$ erhalten wir 
%
\begin{align*}
f(x) = q - g(x)+ (A^T \otimes 1)x + (1 \otimes A^T)x
\end{align*}

Da $(KP)^T = KP$ gilt 
% 
\begin{align*}
g(\textsf{vec}(K)) = \textsf{vec}(KPK) = (1 \otimes KP) \textsf{vec}(K) 
\end{align*}

% Hier arbeit 14.07
%-----------------------------------------------------------------------------------

Wir betrachten die Variation $K(\epsilon) = K + \epsilon \cdot \bar{K}$ und die Funktion $h:\mathbb{R} \to \mathbb{R}^{n^2}$ mit
%
\begin{align*}
h(\epsilon) 
:= g(\textsf{vec}(K(\epsilon))) 
&= \textsf{vec}(K(\epsilon) P K(\epsilon)) \\
&= \textsf{vec}((K + \epsilon \cdot \bar{K}) P (K + \epsilon \cdot \bar{K})) \\
&= \textsf{vec}(KPK + \epsilon KP \bar{K} + \epsilon \bar{K} PK + \epsilon^2 \bar{K} P \bar{K})
\end{align*}

Ableitung nach $\epsilon$ und Auswertung in $\epsilon = 0$ ergibt
%
\begin{align*}
h'(0) = \frac{d}{d\epsilon} g(\textsf{vec}(K + \epsilon \bar{K})) \bigg|_{\epsilon = 0} = \textsf{vec}(KP\bar{K} + \bar{K}PK)
= \textsf{vec}(KP\bar{K}) + \textsf{vec}(\bar{K}PK)
\end{align*}

Umschreiben als Matrix-Vektor-Produkt
%
\begin{align*}
\textsf{vec}(KP \cdot \bar{K} \cdot 1) = (1 \otimes KP) \textsf{vec}(\bar{K}) \quad \textsf{vec}(1 \cdot \bar{K} \cdot PK) = ((PK)^T \otimes 1) \textsf{vec}(\bar{K})
\end{align*}

Wir erhalten als Ableitung 
%
\begin{align*}
g'(x) = (1 \otimes KP) + ((PK)^T \otimes 1) \Rightarrow f'(x) = - (1 \otimes KP) - ((PK)^T \otimes 1) + (A^T \otimes 1) + (1 \otimes A^T)
\end{align*}

\subsubsection{Implementierung}

\begin{minipage}{0.65\textwidth}
\begin{lstlisting}
def vec(X):
    return X.reshape(-1, order='F')

def unvec(v, n):
    return v.reshape((n, n), order='F')
\end{lstlisting}
\end{minipage}
%
\begin{minipage}{0.35\textwidth}
Hier normaler text
\end{minipage}

\begin{minipage}{0.65\textwidth}
\begin{lstlisting}
# Newton Kleinman Verfahren 
def solv_CARE(A,B,R,Q,tol=1e-8,max_iter=50):
	n = A.shape[0]
	I = np.eye(n)
	q = vec(Q)
	L = np.kron(I, A.T) + np.kron(A.T, I)
	P = B @ np.linalg.solve(R, B.T)
	# Startwert K0 hier Einheitsmatrix
	K = I
	x = vec(K)	
	
	for i in range(max_iter):
		X = K @ P #KP
		vec_KPK = np.kron(I,X) @ x
		f = q - vec_KPK + L @ x 
		Dg = np.kron(I,X)+np.kron(X,I) 
		Df = L - Dg
		dx = np.linalg.solve(Df, -f)
		x_new = x + dx
		# Abbruch
		if np.linalg.norm(dx) / np.linalg.norm(x_new) < tol:
			x=x_new
			break
        
        x = x_new
        K = unvec(x, n)
	return K
\end{lstlisting}
\end{minipage}
%
\begin{minipage}{0.35\textwidth}
\rot{Erklärung des Codes}

\blau{Stabilität der Startlösung hier Gershgorin kreise und Stabilität erklären}

Für $Y$ mit $RY=B^T \Rightarrow Y = B^{-1}B^T$ und $B \cdot Y = P$
\end{minipage}
%




%-----------------------------------------------------------------------------------
\newpage

%------------------------------------------------------------------------------------------------------------
\section{Hamilton-Jacobi-Bellmann Gleichung}


Wir betrachten ein Steuerungsproblem und definieren die Value-Function $V$ mittels
%
\begin{align*}
V(x(t),\dunkelgelb{t}) := \min_{\dunkelrot{u}} \left\{ \int_{\dunkelgelb{t}}^{t_f} L(s,x(s),\dunkelrot{u}(s)) \ds + \Phi(x(t_f)) \right\}
\end{align*}

\begin{satz}[Hamilton-Jacobi-Bellman Gleichung]
Die Value-Function erfüllt
%
\begin{align*}
0 = \frac{\partial V(x, t)}{\partial t} + \min_{u} \left[ L(t,x, u) + \frac{\partial V(x,t)}{\partial x} \cdot f(t,x,u) \right], \quad x \in \mathbb{R}^n, \, t_0 \leq t \leq t_f
\end{align*}
%
mit den Randbedingungen
%
\begin{align*}
V(x, \lila{t_f}) = \Phi(x(\lila{t_f})) \quad V(x, \dunkelgelb{t_0} ) = \min_u \int_{\dunkelgelb{t_0}}^{t_f} L(t,x(t),u(t)) \dt + \Phi(x(t_f)) 
\end{align*}
\end{satz}

%------------------------------------------------------------------------------------------------------------
\proof{ 

Die Value-Function $V(x,t)$ beschreibt die minimalen Kosten von Zustand $x$ zum Endzeitpunkt $t_f$.  Für $s \in [t,t_f]$ gilt
%
\begin{itemize}
\item Systemdynamik
%
\begin{align*}
x'(s) = f(x(s), u(s)), \quad x(t) = x
\end{align*}

\item Kostenfunktional
%
\begin{align*}
J(u) = \int_t^{t_f} L(s,x(s),u(s)) \ds + \Phi(x(t_f))
\end{align*}
\end{itemize}

Angenommen, wir wechseln zum Zeitpunkt $t+h$ zu einer optimalen Steuerung, dann gilt für unser Zielfunktional 
%
\begin{align*}
J(u) 
&= \int_{t}^{t+h} L(s,x(s),u(s)) \ds + \dunkelrot{\int_{t+h}^{t_f} L(s,x(s),u(s)) \ds + \Phi(x(t_f)) } \\
&= \int_{t}^{t+h} L(s,x(s),u(s)) \ds + \dunkelrot{V(x(t+h),t+h)} 
\end{align*}

Da $V(x(t), t)$ die minimalen Kosten vom Zustand $x$ zur Zeit $t$ angibt, gilt:
%
\begin{align*}
V(x(t),t) 
&\leq \int_{t}^{t+h} L(s,x(s),u(s)) \ds + \dunkelrot{V(x(t+h),t+h)} \\
\Rightarrow 0 &\leq \int_{t}^{t+h} L(s,x(s),u(s)) \ds + \dunkelrot{V(x(t+h),t+h)} - V(x(t),t) 
\end{align*}

Im Grenzübergang $h \to 0$ erhalten wir
%
\begin{align*}
0 &\leq \gruen{\lim_{h \to 0}  \frac{1}{h}  \int_{t}^{t+h} L(s,x(s),u(s)) \ds} + \dunkelrot{\lim_{h \to 0} \frac{ V(x(t+h),t+h) - V(x,t) }{h}} \\
&= \gruen{ L(t,x(t),u(t))} + \dunkelrot{\frac{\mathrm{d}}{\dt} V(x(t),t)} \\
&= \gruen{ L(t,x(t),u(t))} + \dunkelrot{\frac{\partial V}{\partial t}  + \frac{\partial V}{\partial x} \cdot x'(t)} \\
&= \gruen{ L(t,x(t),u(t))} + \dunkelrot{\frac{\partial V}{\partial t}  + \frac{\partial V}{\partial x} \cdot f(t,x(t), u(t))}
\end{align*}

Durch Minimierung bezüglich $u$ erhalten wir die Gleichheit und somit
%
\begin{align*}
0 &= \frac{\partial V}{\partial t} + \orange{\min_{u} \left[L(t,x(t),u(t)) +  \frac{\partial V}{\partial x} \cdot f(t,x(t), u(t))   \right]} \\
&= \frac{\partial V}{\partial t} + \orange{\textsf{H}\left(t,x(t) , \frac{\partial V}{\partial x}(t,x(t)) \right)}
\end{align*}
}
%------------------------------------------------------------------------------------------------------------


\subsection{Herleitung des Minimumprinzips}

Berechnung der totalen Ableitung der Value-Function $V$ entlang der Charakteristik $x(t)$:
%
\begin{align*}
\frac{dV}{dt} &= \blau{\frac{\partial V}{\partial x}} \frac{dx}{dt} + \lila{\frac{\partial V}{\partial t}} = \blau{p(x,t)} \cdot x'(t) \lila{- H\left( t,x , \frac{\partial V}{\partial x} \right)} \\
\end{align*}

Berechnung der totalen Ableitung von $p = \frac{\partial V}{\partial x}$:
%
\begin{align*}
\frac{dp}{dt} &= \frac{\mathrm{d}}{\dt} \left( \frac{\partial V}{\partial x} \right) =  \frac{\partial^2 V}{\partial x^2} \cdot x'(t) + \rot{\frac{\partial V}{\partial x \partial t}}
\end{align*}

Differenzieren der HJB bezüglich $x$ liefert
%
\begin{align*}
0 = \frac{\mathrm{d}}{\dx} \left[ \frac{\partial V}{\partial t} + H\left(t,x, \orange{ \frac{\partial V}{\partial x}} \right) \right] = \rot{\frac{\partial V}{\partial t \partial x}} + \frac{\partial H}{\partial x} + \frac{\partial H}{\partial \orange{p}} \cdot \frac{\partial^2 V}{\partial x^2}
\end{align*}

Einsetzen der Gleichung liefert:
%
\begin{align*}
\frac{dp}{dt} =  \frac{\partial^2 V}{\partial x^2} \cdot x'(t) + \rot{\left(- \frac{\partial H}{\partial x} - \frac{\partial H}{\partial p} \cdot \frac{\partial^2 V}{\partial x^2} \right)}
= \frac{\partial^2 V}{\partial x^2} \cdot \gruen{\left(x'(t) - \frac{\partial H}{\partial p} \right)} - \frac{\partial H}{\partial x}
\end{align*}

Wir erhalten somit die bekannten, notwendigen Optimalitätsbedingungen:
%
\begin{align*}
p'(t) = -H_x , \quad \gruen{x'(t) = f(t,x,u^*)}, \quad u^* = \textsf{argmin} \left[ L(t,x,u)+ p(t,x)f(t,x,u) \right]
\end{align*}

%-----------------------------------------------------------------------------------------------------------
\subsection{Anwendung auf Linear-Quadratische Probleme}

\subsubsection{Finite time problem}

Gegeben sei das Steuerungsproblem 
%
\begin{align*}
\min_u &  \int_{\dunkelgelb{t_0}}^{t_f} x^T Q x + u^T R u \dt + \underbrace{(x(t_f)- x_f)^T S (x(t_f)- x_f)}_{x(t_f)^T S x(t_f) -2x_{f}^T S x(t_f) +x_{f}^{T} S x_f} 
= V(\dunkelgelb{t_0},x(\dunkelgelb{t_0})) + x_{f}^{T} S x_f
\end{align*}

unter der Nebenbedingung $x'(t) = Ax+Bu$ und $x(t_0) = x_0$. Der Term $x_{f}^{T} S x_f$ ist unabhängig von $u$ und kann bei der Minimierung weggelassen werden.

% Aufbau der Value Funktion und HJB
%----------------------------------------------------------------------------------------------
%\begin{itemize}
%\item Als Ansatz für die Value-Funktion wählen wir 
%
\begin{align*}
V(x,t) 	&= x^T K(t) x + 2 s(t)^T x +r(t) \quad V(x,t_f) = x(t_f)^T K(t_f) x(t_f) + 2 s(t_f)^T x(t_f) + r(t_f)
\end{align*}

Wir erhalten für die Endwerte
%
\begin{align*}
K(t_f) = S, \quad s(t_f) = - S x_f \quad r(t_f) = 0
\end{align*}

%\item Für die partiellen Ableitungen gilt:
%
\begin{align*}
V_t = x^T K'(t) x + 2 s'(t)^T x +r'(t) \quad \rot{V_x = 2K(t)x +2s(t)} 
\end{align*}


%\item Minimierung bezüglich $u$ liefert 
%
\begin{align*}
\frac{\partial}{\partial u} \left[ x^T Q x + u^T R u + V_x^T \cdot (Ax+Bu) \right] &= 2Ru + B^T \cdot V_x \Rightarrow u^{*}(t) = - \frac{1}{2} R^{-1} B^T \rot{V_x}
\end{align*}

%\item Da $R^T = R$ gilt, folgt $(R^{-1})^T = R^{-1}$ und $K^T = K$, da Hesse-Matrix von $V$. %   weiter mit $u^{*}$
%

% Hier weitermachen und ausmultiplizieren
%%%%%%%%%%%%%%%%%%%%%%%%%%%%%%%%%%%%%%%%%%%%%%%%%%%%%%%%%%
\begin{align*}
(u^{*})^T R u^{*} 
&=  \left(- \frac{1}{2} R^{-1} B^T V_x \right)^T R \left(- \frac{1}{2} R^{-1} B^T V_x \right) \\
&= \frac{1}{4} \cdot \left( R^{-1} B^T \rot{V_x} \right)^T R \left( R^{-1} B^T \rot{V_x} \right) \\
&= \frac{1}{4} \cdot \left(\rot{2} R^{-1} B^T \rot{(K(t)x + s(t))} \right)^T R \left( \rot{2} R^{-1} B^T \rot{(K(t)x + s(t))} \right) \\
&=  (K(t)x + s(t))^T \orange{B R^{-1}} \gruen{ R R^{-1}} \orange{B^T} (K(t)x + s(t)) \\
&=    < K(t)x + s(t) , \orange{P} (2K(t)x + s(t)) >  \\
&=  < K(t)x, P K(t)x > +  \underbrace{< K(t)x, Ps(t) >}_{< s(t),  \orange{P^{T}} K(t)x  >} + < s(t), P K(t)x  > + <s(t), P s(t) >  \\
&= x^T K(t)^T P K(t)x + 2 \cdot s(t)^T P K(t) x +  s(t)^T P s(t)
\end{align*}

mit $P = B R^{-1} B^T$ und $P^T = P$.  
%

\begin{align*}
V_x^{T}  \left( - \frac{1}{2} B R^{-1} B^T V_x \right)
&= (2K(t)x +2s(t))^T \left( - \frac{1}{2} B R^{-1} B^T (2K(t)x +2s(t)) \right) \\
&= -2 \cdot (K(t)x + s(t))^T P (K(t)x + s(t)) \\
&= -2 \cdot x^T K(t)^T P K(t)x -4 \cdot s(t)^T P K(t)x -2 \cdot s(t)^T P s(t) 
\end{align*}

\begin{align*}
V_x^T Ax &= 2 (K(t)x + s(t))^T A x = 2 x^T K(t)^T A x + 2 s(t)^T Ax = x^T (K(t)^T A + A^T K(t) ) x + 2 s(t)^T Ax
\end{align*}


\begin{align*}
- V_t 	&= \min_u \left[ x^T Q x + u^T R u + V_x^T \cdot (Ax+Bu) \right] 
\end{align*}



\begin{align*}
- V_t 								&=  x^T Q x + (u^{*})^T R u^{*}  + V_x^T \cdot \left( Ax- \frac{1}{2} B R^{-1} B^T V_x \right) 
\\
\dunkelrot{-} x^T \dunkelrot{K'(t)} x - 2 \blau{s'(t)}^T x  -\orange{r'(t)} 	&= x^T (\dunkelrot{Q-KPK+ KA + A^T K}) x - 2 \cdot (\blau{(P K -A)^T s(t)})^T x - \orange{ s(t)^T P s(t)}
\end{align*}

\begin{align*}
K'(t) &= -Q+K(t) B R^{-1} B^T K(t) - K(t) A - A^T K(t) \quad K(t_f) = S \\
s'(t) &= (K(t) B R^{-1} B^T  - A^T) s(t) \quad s(t_f) = -S x_f  \\
r'(t) &= s(t)^T B R^{-1} B^T s(t) \quad r(t_f) = 0
\end{align*}

\subsubsection{infinite time problem}

Im Grenzfall $t \to \infty$ betrachten wir das Steuerungsproblem
%
\begin{align*}
\min_u &  \int_{\dunkelgelb{t_0}}^{\infty} (x-x_f)^T Q (x-x_f)  + u^T R u \dt 
= V(x(\dunkelgelb{t_0}))
\end{align*}

unter der Nebenbedingung $x'(t) = Ax+Bu$ und $x(t_0) = x_0$. Als Ansatz für die Value Funktion wählen wir erneut 
%
\begin{align*}
V(x) = x^T K x + 2s^T x + r \Rightarrow V_x = 2Kx + 2s
\end{align*}

Es gilt erneut 
%
\begin{align*}
\frac{\partial}{\partial u} \left((x-x_f)^T Q (x-x_f)  + u^T R u  + V_x^T (Ax+Bu) \right) = 2Ru + B^T V_x  = 0 \Rightarrow u^{*} = -\frac{1}{2} R^{-1} B^T V_x 
\end{align*}


\begin{align*}
(x-x_f)^T Q (x-x_f) 
&= x^T Q x - 2 x_f^T Q x + x_f^T Q x_f
\end{align*}

\begin{align*}
0 &= x^T ( Q - KPK + KA + A^T K ) x -2 ((PK-A)^T s + Qx_f)^T x - s^T (P) s + x_f^T Q x_f
\end{align*}

%----------------------------------------------------------------------------------------------
\section{Paper Lin}

\begin{align*}
    M(q) \ddot{q} + C(q,\dot{q}) \dot{q} + G(q) = \tau
\end{align*}

Wir wählen den Zustand $x = \begin{pmatrix} q \\ \dot{q} \end{pmatrix}$ und die Nominalterme

\begin{align*}
    M_0(q), \quad C_0(q,\dot{q}), \quad G_0(q)
\end{align*}

Auflösen nach $\ddot{q}$ 

\begin{align*}
    \ddot{q} = M^{-1}(q) \cdot [ \tau - \underbrace{C(q,\dot{q}) \dot{q} + G(q)}_{N(q, \dot{q})} ]
\end{align*}

Wir setzen nun 

\begin{align*}
    u := M_0^{-1}(q) \cdot [ \tau - N_0(q, \dot{q}) ] 
    \Rightarrow
    \tau = M_0(q) u + N_0(q, \dot{q})
\end{align*}

Setze $\tau = M_0 u + N_0$ in die echte Dynamik ein:
\begin{align*}
    \ddot{q} 
    &= M^{-1} \cdot [ M_0 u + N_0 - N] \\
    &= M^{-1} M_0 u + M^{-1} (N_0 - N) \\
    &= u + \underbrace{(M^{-1} M_0- I)}_{h(x)} u + \underbrace{M^{-1} (N_0 - N)}_{f(x)}
\end{align*}

Für den Zustandsraum $x$ erhalten wir folgendes System:
\begin{align*}
    \dot{x} 
    = \begin{pmatrix} \dot{q} \\ u + h(x) u + f(x) \end{pmatrix} 
    = \underbrace{\begin{pmatrix} 0 & I \\ 0 & 0 \end{pmatrix}}_{A} x + \underbrace{\begin{pmatrix} 0 \\ I \end{pmatrix}}_{B} u + \begin{pmatrix} 0 \\ h(x) u + f(x) \end{pmatrix}
\end{align*}

Unter vernachlässigung $\ddot{q} = u$ erhalten wir das lineare System 
\begin{align*}
    \dot{x} = A x + B u
\end{align*}

\begin{satz}
    Die Lösung des optimalen Steuerungsproblem:
    \begin{align*}
        \min_u \int_0^{\infty} x^T Q x + u^T R u \dt 
        \quad 
        \dot{x} = 
        \underbrace{\begin{pmatrix} 0 & I \\ 0 & 0 \end{pmatrix}}_{A} x 
        + \underbrace{\begin{pmatrix} 0 \\ I \end{pmatrix}}_{B} u
    \end{align*}
    ist gegeben durch
    %
    \begin{align*}
        u^{*} = -R^{-1} B^T K x  \quad \text{mit} \quad Q - KPK + KA + A^T K = 0
    \end{align*}

    und für die Momente gilt 
    %
    \begin{align*}
        \tau^{*} = M_0(q) u^{*} + N_0(q, \dot{q}) = -M_0(q) R^{-1} B^T K x + N_0(q, \dot{q})
    \end{align*}
\end{satz}

\usetikzlibrary{positioning}


\begin{tikzpicture}[>=latex]
    \node[draw= boxframe,anchor = north west ,align=left] (a) at (-2,0) {
        Ursprungsproblem:\\
        $\begin{aligned}
            \min_u & \int_0^\infty q^T Q q + \tau^T R \tau \\
            	ext{s.t. } & M(q) \ddot{q} + C(q,\dot{q}) \dot{q} + G(q) = \tau;
        \end{aligned}$
    };
    \node[draw = rot, anchor = north west, align=left] (b) at ([xshift=1cm]a.north east) {
        LQR:\\
        %1. Wahl von Nominaltermen $M_0, C_0, G_0$\\
        $\begin{aligned}
            \min_u \int_0^{\infty} x^T Q x + u^T R u \dt 
            \quad 
            \dot{x} = 
            \underbrace{\begin{pmatrix} 0 & I_2 \\ 0 & 0 \end{pmatrix}}_{A} x 
            + \underbrace{\begin{pmatrix} 0 \\ I_2 \end{pmatrix}}_{B} u
        \end{aligned}$
    };
    \node[draw = rot, below=of b, align=center] (c) {
        Lösung der CARE für $n=2$ \\
        $\begin{aligned}
            Q - K \underbrace{B R^{-1} B^T}_{P} K + KA + A^T K &= 0 
        \end{aligned}$
        };
    \node[draw = rot, below=of c, align=center] (d) {Closed-Loop Regelung:\\
        $\begin{aligned}
            u^{*} = -R^{-1} B^T K x
        \end{aligned}$
        };
    \node[draw = boxframe, anchor=south, align=left] (e) at ([yshift=-6cm]a.south) {
        Lösung Originalsystem \\
        $\begin{aligned}
            \tau^{*} &= M_0(q) u^{*} + N_0(q, \dot{q}) \\
            M_0 &= M(q=0)  \\
            N_0 &= N(q, \dot{q}) = C_0(q, \dot{q}) \dot{q} + G_0(q)
        \end{aligned}$
        };

    \draw[->, color = boxframe] (a) -- (b);
    \draw[->, color = boxframe] (b) -- (c);
    \draw[->, color = boxframe] (c) -- (d);
    \draw[->, color = boxframe] (d.west) -- ++(-3.65cm,0cm);
    \draw[->, color = gruen] (a) -- (e);
\end{tikzpicture}

\proof{
    Für die Value Funktion erhalten wir 
    %
    \begin{align*}
        \frac{\mathrm{d}}{\dt} V(x(t)) &= 
        \underbrace{\nabla V(x)}_{2K x(t)} 
        \cdot 
        \underbrace{x'(t)}_{Ax(t) + Bu(t)} \\
        &= 2 x(t)^T K 
        \left[ 
            Ax(t) + B \underbrace{ - R^{-1} B^T K x(t)}_{u(t)} 
        \right] \\
        &= x(t)^T 
        \left[ 
            \underbrace{K A + A^T K - 2 K B R^{-1} B^T K}_{-Q - K B R^{-1} B^T K} 
        \right] 
        x(t) \\
        &= - x(t)^T Q x(t) - \underbrace{x(t)^T K B}_{x(t)^T K B R^{-1} R} \underbrace{R^{-1} B^T K x(t)}_{-u^*(t)} \\
        &= - x(t)^T Q x(t) - \underbrace{x(t)^T K B R^{-1} R}_{u^{*}(t)^T R} u^*(t) \\
        &= - x(t)^T Q x(t) - u^{*}(t)^T R u^{*}(t)
    \end{align*}

    Wir erhalten somit $\frac{\mathrm{d}}{\dt} V(x(t)) \leq 0$ mit Gleichheit genau dann wenn $x(t) = u^{*}(t) = 0$. 
    Damit ist die Value Funktion $V(x(t))$ streng monoton fallend und 
    somit ist das optimale Steuerungssignal $u^{*}(t)$ stabil.

    Verwenden wir die Value Funktion auf die reale Dynamik, so gilt

    \begin{align*}
        \frac{\mathrm{d}}{\dt} V(x(t)) 
        &= \nabla V(x) \cdot x'(t) \\
        &= 2 x(t)^T K 
        \left[ 
            Ax(t) + B u^{*}(t) + \begin{pmatrix} 0 \\ h(x(t)) u^{*}(t) + f(x(t)) \end{pmatrix}
        \right] \\
        &= 2 x(t)^T K 
        \left[ 
            Ax(t) + B u^{*}(t) + B h(x(t)) u^{*}(t) + B f(x(t))
        \right] \\
        &= \underbrace{ 2 x(t)^T K (Ax+Bu^{*})}_{(1)} + \underbrace{2 x(t)^T K B (h(x) u^{*} + f(x))}_{(2)}
    \end{align*}

    wobei wir $B = \begin{pmatrix} 0 \\ I_2 \end{pmatrix}$ verwendet haben. 
    Aus $u^* = -R^{-1} B^T K x(t)$ folgt $B^T K x(t) = -R u^{*}$ und wir erhalten für $(2)$

    \begin{align*}
        2 x(t)^T K B (h(x) u^{*} + f(x)) 
        &= 2 \left[\underbrace{B^T K x(t)}_{-R u^{*}} \right]^T (h(x) u^{*} + f(x)) \\
        &= -2 (u^{*})^T R (h(x) u^{*} + f(x)) \\
        &= -2 (u^{*})^T R h(x) u^{*} - 2 (u^{*})^T R f(x)
    \end{align*}

    Für $h$ positiv definit gilt:
    \begin{align*}
        -(u^{*})^T R h(x) u^{*} \leq 0
    \end{align*}

    Für jedes $\epsilon > 0$ gilt:
    \begin{align*}
        -2 (u^{*})^T R f(x) 
        &\leq 2 \|u^{*}\| \|R\| \|f(x)\| \\
        &\leq \epsilon \|u^{*}\|^2 + \frac{1}{\epsilon} \|R\|^2 \|f(x)\|^2 \\
        &= \epsilon (u^{*})^T u^{*} + \frac{1}{\epsilon} \|R\|^2 \|f(x)\|^2 \\
        &\leq \epsilon (u^{*})^T R u^{*} + \frac{1}{\epsilon} \|R\|^2 \|f(x)\|^2
    \end{align*}

    Wir wählen als Schranke $f^T R f \leq x^T P x$
}

\end{document}
